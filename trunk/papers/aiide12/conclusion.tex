\section{Conclusion}
\label{sec:conclusion}
We report the first known results of TRANSIT~\cite{bast06} route-finding to grid-based benchmarks
from video games.  We find that on such domains the basic algorithm is impacted, in a strongly
negative way, by the presence of uniform-cost path symmetries.  To address this, we give a new
general symmetry breaking technique involving the random perturbation of edges in the input graph
with small $\epsilon$-costs. We prove this technique is optimality preserving and show that it can
reduce TRANSIT's memory overhead by several factors and improve performance by up to two orders.  We
undertake an extensive empirical analysis of TRANSIT on a range of popular grid-based pathfinding
benchmarks taken from video games and give a first comparison of TRANSIT with CPDs~\cite{botea11}.
We find the two have complementary strengths and identify a class of problems to which TRANSIT
appears better suited: distance queries involving start and goal locations that are not in close
proximity.
An obvious direction for future work appears to be combining TRANSIT and CPDs.  Another possibility
is to reduce the number of nodes in the TRANSIT network; for example through the application of
recent graph partitioning schemes~\cite{natural_cuts} or the application of systematic symmetry
breaking in the manner of Jump Point Search~\cite{harabor11b}.  Finally we believe it may be
possible to use a subset of the TRANSIT network as an accurate memory heuristic for A* search.
