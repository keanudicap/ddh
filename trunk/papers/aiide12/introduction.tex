\section{Introduction}
\label{sec:introduction}
The AI and Game Development communities have devoted much attention to the study
of both exact and approximate techniques that speed up forward state-space search algorithms
such as Dijkstra and A*.
These efforts range from: \textbf{(i)} abstraction-based near-optimal techniques such as~\cite{botea04,sturtevant07}
\textbf{(ii)} precomputation algorithms for improving heuristic estimates; for example~\cite{sturtevant09,goldenberg10}
\textbf{(iii)} online and offline pruning and symmetry breaking methods such as~\cite{bjornsson06,pochter09,harabor11b} and
\textbf{(iv)} compressing the entire set of All-Pairs data as in ~\cite{botea11}.
Almost all involve a speed vs. memory tradeoff and typically deliver improvements in the
range of one or (as in the case of \cite{botea11}) two orders of magnitude.

In addition to computing shortest paths it is sometimes desirable to efficiently
calculate the distance between two units on a map or the distance to an object
of interest.  Support for such \emph{distance queries} is found in popular game
libraries, including Umbra 3~\footnote{\url{http://www.umbrasoftware.com/}},
where they are used for optimizing game logic and driving scripted events.
Distance queries may also be useful for higher level AI; e.g. as described
in~\cite{champandard09}.

TRANSIT~\cite{bast06} is a well known and influential technique that supports both shortest
path and distance queries.  Developed in the Algorithmics community for the
purpose of routing on road networks, TRANSIT can be described as an optimality
preserving abstraction technique whose primary advantages are typically modest
memory requirements and the ability to eliminate almost entirely the need for
state-space search.  For example, on the 24 million node US road network,
TRANSIT requires less than 500MB of storage and answers 99\% of all distance
queries without any state-space search in just 12 $\mu$s \footnote{The
remaining 1\% are short local queries solved using state-space search that is
reported to take 5 milliseconds on average.}.

TRANSIT's performance is due to a natual property of most road
networks: low \emph{highway dimension}~\cite{abraham10}.
Intuitively, a graph has low highway dimension if, for all nodes inside an area
of radius $r$, there are a small set of vertices that cover all shortest paths
of length longer than $k \cdot r$, for some value $k$.  It is unclear to what extent this characteristic
applies to grid maps. Previous work~\cite{goldberg06} has indicated that, for
randomly-weighted grids at least, this property is unlikely to hold.
However a recent study~\cite{sturtevant2012benchmarks} suggests that for certain
popular grid benchmarks, particularly those drawn from video games, the opposite
may be true.

Our contribution is as follows:
We give a first detailed evaluation of TRANSIT on popular grid domains from the AI literature.
We find in the first instance that TRANSIT is strongly and negatively impacted,
in terms of running time but also memory, by uniform-cost path symmetries that
are inherent to many grid-based domains but not road networks.
We address this with a new general technique for breaking symmetries using
small additive $\epsilon$-costs to perturb the weights of edges in the search graph.
Our enhancements reduce the number of nodes in the TRANSIT network by up to 4 times
and yield running times up to \emph{4 orders of magnitude} faster than A* search.
We also compare TRANSIT with CPDs~\cite{botea11}: a recent and very
fast database-driven pathfinding approach. Our results indicate the two algorithms
have complementary strengths and we suggest an approach by which they could be combined.
However, we also identify a class of problems to which TRANSIT appears uniquely well suited.
In these cases we report up to two orders speed improvement vs CPDs using a comparable
amount of memory.

