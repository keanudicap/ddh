\section{Experimental Setup}
We evaluated the performance of AA* and HAA* on a set of 120 octile-based maps, ranging in size from 50x50 to 320x320, which we borrowed from a popular roleplaying game.
The same maps were used in previous games-related research (e.g., \cite{botea04}).
In their default configuration, the maps only featured one type of traversable terrain interspersed with hard obstacles. 
We therefore created five derivative sets (making for a total of 720 maps) where each traversable tile on every map had one of $\lbrace 10\%, 20\%, 30\%, 40\%, 50\% \rbrace$ probability of being converted into a second type of traversable terrain. Octiles with this terrain type are called \emph{soft obstacles}, as they are not traversable by all agents.
This allowed us to evaluate the algorithms in environments with both soft and hard obstacles.
\par \indent
For each map we generated 100 experiments by randomly creating valid problems between arbitrarily chosen pairs of locations and some random capability.
We used two agent sizes in each experiment: small (occupying one tile) and large (occupying four tiles) resulting in 144,000 problem instances (720x200) overall.
All experiments were conducted on a 2.4GHz Intel Core 2 Duo processor with 2GB RAM running OSX 10.5.2.
To implement both planners we used the University of Alberta's freely available pathfinding library, HOG (\url{www.cs.ualberta.ca/~nathanst/hog.html}). 
