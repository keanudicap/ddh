\section{Introduction}
\label{sec:introduction}

Symmetry in grid-based pathfinding manifests itself when we consider a
path, traditionally defined as an ordered sequence of nodes, as consisting
instead of an ordered sequence of vectors.
Each vector $\vec{d}$ is associated with 
one of the eight allowable movement directions (up, down, left, right etc.)
and has a magnitude of either 1 or $\sqrt 2$, depending on whether 
it represents a straight or a diagonal move.
Such a formulation allows us to see that 
many paths on a grid map, which share the same start and end node but which pass through 
different intermediate nodes, are often just symmetric permutations of each other; i.e.
they are identical save for the order in which the individual moves occur.

In the presence of symmetry, A* unnecessarily considers permutations 
of all shortest paths: from the start node to 
\emph{every expanded node}.
Jump Point Search~\cite{harabor11b} is a simple but highly effective strategy that
eliminates many such symmetries. 
JPS is fast, optimal, requires zero preprocessing, has zero memory overhead
and appears orthogonal to recent pathfinding algorithms; e.g.~\cite{bjornsson06,pochter10,goldenberg10}. 
We first adapt JPS to grid domains where ``corner-cutting'' movement is not allowed. 
Then, we introduce JPS+: a new search method which reformulates an input graph into a 
symmetry-reduced equivalent that can be 
searched much faster.
