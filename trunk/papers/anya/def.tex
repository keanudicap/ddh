\section{Preliminaries}
A \emph{grid} is a planar subdivision consisting of $W \times H$ square cells.
Each cell is either \emph{traversable} or \emph{non-traversable}
\footnote{Such cells are sometimes called obstacles.}.
%It is common
%in such settings for each tile to represent a unique location in the grid and to
%assume there is an implicit vertex located at its centre~\cite{yap02}.  In this
%paper we follow~\cite{nash07} and adopt a slightly different model: 
The set of discrete points that define the perimeter of each cell are the locations
of the grid. Each such point $p$ has a unique coordinate $(x, y)$
where $x \in \{0,\dots,W\}$ and $y \in \{0,\dots,H\}$.
% refers to an explicit vertex 
Edges in the grid can be interpreted as continuous intervals connecting two 
discrete points. Each edge is common to at most two adjacent cells. Each discrete 
point is common to at most four adjacent cells.
%We now define some terminology for discussing discrete points:
An \emph{intersection} is any discrete point that is common to
four adjacent cells. A \emph{corner} is any intersection where three of the
adjacent cells are traversable and one is not. 
Two points are \emph{visible} from one another if there exists a 
straight-line path connecting them which does not pass through the interior of any
 non-traversable cell.
%A \emph{corner} is an intersection 
%such that only one of its four squares is non-traversable.  
%Intersection $i'$ is \emph{visible} from intersection $i$ 
%iff the segment $(i,i')$ crosses only traversable tiles 
%or borders non-traversable tiles and traversable tiles.  
%Without loss of generality, 
%we assume that if the segment $(i,i')$ crosses a corner 
%(besides $i$ and $i'$), 
%then intersection $i'$ is not visible from intersection $i$.  

Using the above we define an \emph{any-angle path} $\pi$ as a sequence of discrete points 
$\langle p_1,\dots,p_k \rangle$ where each $p_{i}$ is visible from both $p_{i-1}$
and $p_{i+1}$.
The \emph{length} of path $\pi$ 
is the cumulative straight-line distance between every successive
pair of discrete points along the path, 
i.e., $d(p_1,p_2) + d(p_2,p_3) + \dots + d(p_{k-1},p_k)$, 
where $d((x,y), (x',y'))= \sqrt{(x-x')^2 + (y-y')^2}$ 
is a uniform Euclidean distance metric.
We will say $p_i \in \pi$ is a \emph{turning point} if the segments
$(p_{i-1}, p_i)$ and $(p_i, p_{i+1})$ form an angle.
\\
\begin{lemm}
\label{lemma::corner}
  Given two discrete points $p$ and $p'$, 
  any turning point in the optimal any-angle path between $p$ and $p'$ 
  is also a corner point.
\end{lemm}

\begin{proof}
{
  Assume an optimal any-angle path $\pi = (p_1,\dots,p_k)$ 
  that includes a turning point $p_l$ ($l \not\in \{0,k\}$) 
  which is not a corner.  
  We will prove that $\pi$ is suboptimal 
  which, by contradiction, will prove the lemma.  
  If $p_{l+1}$ is visible from $p_{l-1}$, 
  then  $\pi \setminus p_l$ is a path 
  which is strictly shorter than $\pi$.  
  Hence $\pi$ is suboptimal.  
  If $p_{l+1}$ is not visible from $p_{l-1}$, 
  then let $p'_{l}$ be a point from the segment $(p_l,p_{l+1})$ 
  that (i) is visible from $p_{l-1}$ 
  and (ii) is different from $p_l$.  
  Because $p_l$ is not a corner, 
  then such a $p'_l$ exists.  
  Moreover, the subpath $\langle p_{l-1}, p'_{l}, p_{l+1} \rangle$ is 
  strictly shorter than $\langle p_{l-1}, p_{l}, p_{l+1}\rangle$. 
  Hence $\pi$ is suboptimal.
  This case is is illustrated in Figure~\ref{fig::corner}  
  where the point $p'_l$ is chosen as close as possible from the corner $c$.  
}
\end{proof}


\begin{figure}[tb]
  \begin{center}
    \begin{tikzpicture}

\creategrid{6}{6}

\drawobstacle{2}{5}
\drawobstacle{2}{4}
\drawobstacle{2}{3}
\drawobstacle{3}{3}
\drawobstacle{3}{2}
\drawobstacle{3}{1}
\drawobstacle{4}{3}
\drawobstacle{4}{2}
\drawobstacle{4}{1}

\coordinate (plm1)   at (5,0);
\coordinate (pl)     at (2,1);
\coordinate (plp1)   at (1,5);
\coordinate (redpathdirection) at (0,2.4); % Actually, further away
\coordinate (actualcorner) at (3,1); 
\path[name path=uppath] (pl) -- (plp1);
\path[name path=shortcut] (plm1) -- (redpathdirection);
\draw[name intersections={of=uppath and shortcut, by=plp'}];

\draw[red, line width=2pt]   (plm1) -- (plp');
\draw[blue, line width=2pt]  (plm1) -- (pl) -- (plp');
\draw[green, line width=2pt] (plp1) -- (plp');

\draw (plm1) +(0.4,0.2) node {$p_{l-1}$};
\draw (pl) +(-0.3,-0.2)   node {$p_{l}$};
\draw (plp1) +(-0.3,0.2) node {$p_{l+1}$};
\draw (plp') +(-0.3,0) node {$p'_l$};
%\draw (actualcorner) +(-0.3,0.3) node {$c$};

\end{tikzpicture}

  \end{center}
  \caption{Illustration of Lemma~\ref{lemma::corner}.
  A similar observation has been previously made for geodesic paths \cite{mitchell87}.  }
  \label{fig::corner}
\end{figure}


% EOF
