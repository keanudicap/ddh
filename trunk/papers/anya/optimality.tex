\section{Correctness and Optimality}
We begin by connecting the notions of \emph{taut} and \emph{optimal}.
\begin{lemm}
\label{lemma::opt_is_taut}
Every optimal path $\pi$ is also taut.
\end{lemm}
\begin{proof}
By contradiction. If $\pi$ is not taut we can apply 
Lemma~\ref{lemma::corner} to derive a path $\pi'$ which is 
shorter than $\pi$. 
%The existance of any such $\pi'$ contradicts the 
%optimality of $\pi$.
\end{proof}
Lemma~\ref{lemma::opt_is_taut} tells us that,
during optimal search, the only successors of a node which
are interesting are those reachable by a local path that is taut.
\begin{lemm}
\label{lemma::taut_successors}
Every point $p'$ in an interval $I$ is reached by a taut local
path of the form $\langle r, p, p' \rangle$ where
r is the root of a search node $(I, r)$ and $p \in I$.
%For every point $p'$ there exists a taut local path from 
%For every search node $(I, r)$ and
%taut local path $\langle r, p, p' \rangle$ from $r$, through some 
%$p \in I$ and to a point $p'$: there exists
%a successor $(I', r')$ s.t. $p' \in I'$.
\end{lemm}
\begin{proof}
%% NB: The full proof needs to be an inductive argument
%% for the visible and non-visible cases.
%% We have to show that every point can be reached by a taut local
%% path and that the union of taut local paths is a complete cover.
%% Don't forget to point out that the start node has 0 observable successors.
Sketch. There are two cases to consider depending on whether or not $p'$ is
visible from $r$ through $I$.  If $p'$ is visible from $r$ through $I$ then it
belongs to one of the observable successors of $(I, r)$. Otherwise $p'$ belongs
to one of the non-observable successors of $(I, r)$.  The only points $p'$ not
found in any successor interval are those where the local path from $r$ through 
$I$ is not taut. Refer once more to Definition~\ref{defi::successors} and
Figure~\ref{fig::successors}.
\end{proof}
From Lemma~\ref{lemma::taut_successors} we conclude that Anya is complete: for
any search node $(I, r)$ every point on a taut local path from $r$ through
$I$ is reached through some successor $(I', r')$.  As we only have
positive cost transitions, from a node to each successor, we can be certain that
if a path exists it will eventually be returned. We now show that if Anya returns
a path that path is cost optimal.
\begin{theorem}
For any arbitrary pair of discrete points $s$ and $t$, which are connected,
Anya always finds a path $\pi = \langle s, \ldots, t\rangle$ which is
cost optimal.
\end{theorem}
\begin{proof}
For any search node $(I, r)$ we must show: (i) the cost of the any-angle path 
from the start point to each root point is optimal; (ii) the $f$-value
of the representative point from each interval is minimum; (iii)
$f$-values along any given path are monotonically increasing.
%Optimality requires that $f$-costs are monotonically increasing along any 
%given path. Thus for any search node $m = (I_m, r_m)$ whose representative
%point is $p_m \in I_m$ and any arbitrary 
%successor $(I_n, r_n)$ whose representative point is $p_n \in I_n$ we 
%will need to show that $f(p_m) \leq f(p_n)$.
%
%where $p_m$ and $p_n$ are the representative points 
%The $f$-value of each node is computed using Equation~\ref{eq::fvalue}
%and requires selecting from each interval a single representative point 
%$p_m \in I_m$ and $p_n \in I_n$. To satisfy the monotone condition we
%need to show that $f(p_m) \leq f(p_n)$.
%
%If $n$ is an observable successor then $p_n$ is visible from $r_m$
%and both $p_m$ and $p_n$ appear on the straight line $r_m \rightarrow t$.
%The length of this line is a least lower bound on the cost of the optimal 
%path and using Equation~\ref{eq::fvalue} we have $f(p_m) = f(p_n)$.
%
%
%where $f(m) = f(p_m)$ and $p_m \in I_m$ is a representative point 
%chosen from amongst all points in the interval $I_m$.
%We require that for each successor $(I_n, r_n)$ the representative point 
%$p_n \in I_n$ has an associated $f$-value s.t: $f(p_m) \leq f(p_m)$.
%Each representative point is selected according to Lemma~\ref{lemma:fvalue}:
%
%If $p_n$ is visible from $r_m$ then $p_n$ 
%
%
%
%
%From all points in the interval $I_m$ we choose a representative
%$p_m$  
% 
%From the set f
%and any arbitrary successor $(I_n, r_n)$.
%Since $r_m \not \in I_m$ and $I_m \cap I_n = \emptyset$ it follows 
%that the cost
%
%To see this consider any taut local path: 
%$\langle r_m, p_m \in I_m, p_n \in I_n\rangle$. the cost of the path
%from one node to the next is greater than zero since (i)
%$r_m \not \in I_m$ and $I_m \cap I_n = \emptyset$.
%
%we do not allow negative cost transitions and (ii) $r_m \not \in I_m$ 
%By Definition~\ref{defi::searchnode} $r_m \not \in I_m$;
%When expanding a search node $(I_m, r_m)$ we require that every path
%to each successor $(I_n, r_n)$ begin at $r_m$ and pass through some 
%$p_m \in I_m$ before k
%through some $p_m \in I_m$ 
%
%Optimality requires that, during search, costs are monotonically increasing
%along any given path 
%We begin by showing $f$-values computed by Anya are monotonically increasing.
%This property is necessary to prove optimal
%Monotonicity is a necessary property of 
%When expanding a node $n = (I_n, r_n)$ Anya considers at the same time
%all local paths from $r_n$ to each $p_n \in I_n$. 
%From Lemma~\ref{lemm::taut_successors} every such path is taut.
%By 
%
%all points $p_n \in I_n$. 
%At each expansion we have found an optimal path to the root $r_m$.
%
%Each time Anya expands a node $n = (I_n, r_n)$ it has found an optimal
%path from the source to the root point $r_n$. 
%
%bla blah
%Conclude Anya is consistent.
%	
%
%The local path $\langle r_m, p_m\rangle 
%a set of points Each search node represents a set of points 9
%
%
%Sketch only. Suppose $\pi$ is not optimal. There must be another 
%path $\pi'$ which is shorter. Since $\pi'$ is shorter $\ldots$.
%\\
%Each time Anya expands a node $m = (I_m, r_m)$ it generates $k$ successors
%each of which can be observable or non-observable.
%From each successor $n = (I_n, r_n)$ a representative point $p_n$ is chosen s.t 
%there is a taut local path $\langle r_m, p_m, p_n\rangle$
%where $p_m \in I_m$ and $p_n \in I_n$.
%Since we do not have any negative or zero-cost transitions we have
%$\ldots$. Thus the $f$-value
%of each successor $m$ is monotonically increasing wrt the $f$-value of 
%its parent $n$. 
%From Lemma~\ref{lemm::fvalue} we also have that $forall p'' \in I'$ 
%$f(p') \leq f(p'')$.
\end{proof}

