This section now presents our search algorithm, 
dubbed {\bf anya}, 
for any-angle pathfinding.  

\begin{algorithm}[ht!]
  \begin{algorithmic}
\STATE {\bf input}: Graph $G$, source $s$, target $t$
\STATE $open := \{s\}$
\LOOP
  \STATE $i := pop(open)$
  \IF{$i = t$}
    \STATE {\bf return} $i$
  \ENDIF
  \IF{$i$ is a node}
    \FORALL{$i' \in$ successors\_of\_corner($i$)}
      \STATE add\_edge($i$,$i'$)
    \ENDFOR
  \ELSE %\COMMENT{$i$ is an interval}
    \STATE {\bf let} $i = \langle [x_{\min},x_{\max}],y,p\rangle$.
    \STATE $i_1 :=$ move($i$)
    \STATE $cs :=$ corners\_of($i_1$)
    \COMMENT{Including target.}
    \STATE $is :=$ split($i_1,cs$)
    \FORALL{$c' \in cs$}
      \IF{$c'$ is visible from $p$}
        \STATE add\_edge($i$,$c'$)
      \ENDIF
    \ENDFOR
    \FORALL{$i' \in is$}
      \STATE {\bf let} $i' = \langle [x'_{\min},x'_{\max}],y',p\rangle$.
      \IF{$\langle \frac{x'_{\min}+x'_{\max}}{2},y'\rangle$ is visible from $p$}
        \STATE add\_edge($i$,$i'$)
      \ENDIF
    \ENDFOR
  \ENDIF
\ENDLOOP
\end{algorithmic}


  \caption{Procedure {\bf anya}, an any-angle pathfinding algorithm}
  \label{algo::anya}
\end{algorithm}

\begin{algorithm}[ht!]
  \begin{algorithmic}
\STATE {\bf input}: node $c$
\RETURN  $\{
           \langle [0,W], y_c, c, \uparrow\rangle,
           \langle [0,W], y_c, c, \downarrow\rangle
         \}$  
\end{algorithmic}

  \caption{Procedure $expand\_node(c)$}
  \label{algo::expandnode}
\end{algorithm}

The main procedure is presented in Algorithm~\ref{algo::anya}.  
The algorithm is a variant of $A^*$ 
with a specific successors generation.  
An interval can be generated from a corner $c$ 
(or the starting node $s$) 
through the procedure $expand\_node(c)$, 
or as the move of an existing interval.  
When an interval $i$ is expanded, 
it is splited as presented in the previous section 
via method $split\_epsilon(i)$.  
Finally, notice that the procedure $open.insert(I)$, 
where $I$ is a set of intervals, 
computes the $f$ value of each interval of $I$, 
and deals with ordering the intervals 
(including ignoring those that have already been expanded 
or those with existing smaller $f$ value).  

The $expand\_node(c)$ procedure is presented 
in Algorithm~\ref{algo::expandnode}.  
From the current node, 
the two intervals are created 
that stretch horizontally as far possible, 
at ordinate $y_c$, with origin $c$, and heading 
in vertical (North or South) direction.  

\begin{algorithm}[ht!]
  \begin{algorithmic}
\STATE {\bf input}: interval $\langle [x_{\min},x_{\max}], y, p, dir\rangle$
\STATE $I := \emptyset$, $C := \emptyset$
\STATE $x_1 := -1; x_2 := -1$
\STATE $obs := obstacle?(x_{\min}-1,y+dir)$
\IF{$\neg obs$}
\STATE $x_1 := x_{\min}$
\ENDIF
\FOR{$current\_x := x_{\min} ; current\_x \le x_{\max} ; current\_x++$}
  \STATE 
\ENDFOR
%
\WHILE{$\langle x_1, x_2\rangle := next\_interval([x_{\min},x_{\max}],y)$}
  \STATE $I := I \cup \{\langle [x_1,x_2], y, p, dir \rangle\}$
  \FORALL{$k \in \{1,2\}$}
    \IF{square from $\langle x_i,y\rangle$ 
      with horizontal direction $x_i - x$ and vertical direction $dir$ 
      is not an obstacle}
      \STATE $C := C \cup \{\langle x_i,y\rangle\}$
    \ENDIF
  \ENDFOR
\ENDWHILE
\RETURN $\langle I, C \rangle$
\end{algorithmic}

  \caption{Procedure $split\_epsilon(c)$}
  \label{algo::splitepsilon}
\end{algorithm}

$next\_interval([x_{\min},x_{\max}],y)$

$move(i)$

%EOF
