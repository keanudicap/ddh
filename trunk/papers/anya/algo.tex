\section{Algorithm}
%The best online approaches for any-angle pathfinding consider 
%only the set of discrete points that define the grid.
%Since this strategy is insufficient to guarantee optimality 
%we will consider sets of discrete and intermediate points
%by taking them together as an interval.

\begin{defi}
\label{defi::interval}
A \emph{grid interval} $I$ is a closed set of contiguous pairwise
visible points whose elements are either:
(i) a single discrete point which is the start location or the goal location;
(ii) a single corner point;
(iii) a set of traversable points drawn from the intermediate and discrete 
non-corner points along any row of the grid.
\end{defi}
We will look for optimal any-angle paths in the search space formed by
intervals appearing along the rows of the grid.
\begin{defi}
A \emph{search node} $n = (I, r)$ is a tuple where $I$ is an interval and $r$
is a \emph{root} point chosen such that each $p \in I$  is visible from $r$. 
The identity of the root node is either the start location or the most
recent corner point on the path used to reach $I$.
\end{defi}
To evaluate a search node $n = (I, r$) we will use a single representative 
point $p \in I$ which has a minimum $f$-value with respect to a target point
$t$:
\begin{equation}
f(n) = \arg\min \lbrace f(p) = g(r) + d(r, p) + h(p, t) \rbrace 
\end{equation}
Although each interval can contain a large number of points it is easy
to identify the one having a minimum $f$-value: 
we simply project a straight line $r \rightarrow t$ from the root point $r$ to the 
target $t$ and choose the point $p \in I$ which is intersected by the line.
In the case that $r \rightarrow t$ does not intersect any point in $I$ we simply
choose the endpoint of $I$ closest to $t$. 
In the following we give a formal argument that this strategy always identifies
$p \in I$ having a minimum $f$-value.

\begin{lemm}
A true statement will go here.
\end{lemm}
\begin{proof}
A very clever and convincing proof will go here.
\end{proof}
%
%Our search nodes will comprise of tuples of the form $(I, r)$ where
%$I$ is an interval and $r$ is a \emph{root} point which can be
%either the start location or the most recent corner point on the path used
%to reach to $I$.
%Given any such tuple we require that each $p \in I$ is visible from $r$.
%\\
During search we will project intervals from one row of the grid to another.
In particular we will say that the successors of an interval $I$
are just those intervals $I'$ which are immediately adjacent
to $I$.
\begin{defi}
\label{defi::intadj}
A grid interval $I$ is \emph{adjacent} to $I'$ iff for every pair 
of points $p \in I$ and $p' \in I'$ there exists a straight line
path that does not pass through any interval 
$J \not \in \lbrace I, I' \rbrace$
\end{defi}
Definition~\ref{defi::intadj} bounds the branching factor of each 
search node to $O(W)$ and unlike other exact algortihms (e.g. visibility graphs)
allows us to defer the generation of any neighbours that are not 
immediately promising.
\\

\section{Stuff that is not finished}

Essentially, the reason why the existing algorithms are not optimal
is that the search is based on the grid intersections.  
However, because the final any-angle path 
may never cross these intersections, 
the $A^*$ $f$ value associated with the intersections
is not relevant.  
In this article, we propose to run the search not on the grid 
but instead on the edges of the graph.  

bla bla

From $n_1$, all the nodes that can be reached with a one step North 
(and potentially any fraction of West/East steps) will be considered.  
We call ``interval'' this set of nodes that are considered together.  

How should we then compute the successors of an interval?  
Look at Figure~\ref{fig::succ1} (left) 
where a North step is taken from the interval $[(1,3)-(4,3)]$
reached from node $(1,1)$.  
The (single) successor of the interval 
is simply computed by projecting the visibility cone 
(represented with dashed lines)
from $(1,1)$ to the next row.  
This leads to the interval $[(1,4)-(5+\frac{1}{2},4)]$.  
Obverse that, as a consequence, the parent of the interval 
(here $(1,1)$) must be kept together with the interval.  

\begin{figure}[ht]
  \begin{minipage}{0.5\linewidth}
  \begin{center}
    \begin{tikzpicture}

\creategrid{6}{6}
\drawobstacle{0}{2}
\drawobstacle{4}{2}

\draw[dashed, line width=2] (1,1) -- (1,4);
\path[name path=p1] (1,1) -- +(6,4);
\path[name path=p2] (0,4) -- (6,4);
\path[name path=p3] (0,3) -- (6,3);

\draw[name intersections={of=p1 and p3,by=x2}];
\draw[dashed, line width=2] (1,1) -- (x2);
\draw[red,line width=2] (1,3) -- (x2);

\draw[name intersections={of=p1 and p2,by=x1}];
\draw[dashed, line width=2] (1,1) -- (x1);
\draw[red,line width=2] (1,4) -- (x1);

\end{tikzpicture}

  \end{center}
  \end{minipage}
  \begin{minipage}{0.5\linewidth}
  \begin{center}
    \begin{tikzpicture}

\creategrid{6}{6}
\drawobstacle{0}{2}
\drawobstacle{2}{3}

\draw[dashed, line width=2] (1,1) -- (1,4);
\path[name path=p1] (1,1) -- +(6,4);
\path[name path=p2] (0,4) -- (6,4);
\path[name path=p3] (0,3) -- (6,3);
\path[name path=p4] (1,1) -- +(4,4);

\draw[name intersections={of=p1 and p3,by=x2}];
\draw[dashed, line width=2] (1,1) -- (x2);
\draw[red,line width=2] (1,3) -- (x2);

\draw[name intersections={of=p1 and p2,by=x1}];
\draw[dashed, line width=2] (1,1) -- (x1);
\draw[red,line width=2] (1,4) -- (2,4);
\draw[dashed, line width=2] (1,1) -- (2,4);
\draw[name intersections={of=p4 and p2,by=x3}];
\draw[dashed, line width=2] (1,1) -- (x3);
\draw[red,line width=2] (x3) -- (x1);

\end{tikzpicture}

  \end{center}
  \end{minipage}
  \caption{Computing the successors of an interval}
  \label{fig::succ1}
\end{figure}

There are more complicated situations, 
as presented in Figure~\ref{fig::succ1} (right).  
Here, we can see that an obstacle splits the interval 
in two separate intervals.  
We could define disjunct intervals, 
but it seems more natural to consider them independently.  

Furthermore, the path might turn at node $(3,3)$
(e.g., to reach node $(3,4)$
which is not directly accessible from $(1,1)$).  
The corner $(3,3)$ is therefore a successor of node $(1,1)$ 
and intervals should be built from this node.  

%Let us present more precisely how the successors are splitted.  
%From the interval $I = [(1,3)-(4,3)]$, 
%we assume that an epsilon step 
%(i.e., an infinitely small step North) is taken.  
%This step reveals how the interval will be splitted.  
%Therefore, we generate interval 
%$[(1,3+\varepsilon)-(4+\frac{3}{2}\varepsilon,3+\varepsilon)]$; 
%we then remove the parts that belong to the obstacle, 
%which produces the two intervals 
%$[(1,3+\varepsilon)-(2,3+\varepsilon)]$ 
%$[(3,3+\varepsilon)-(4+\frac{3}{2}\varepsilon,3+\varepsilon)]$; 
%finally, we split the part that is not visible from $(1,1)$, 
%which leads to three intervals: 
%$[(1,3+\varepsilon)-(2,3+\varepsilon)]$, 
%$[(3,3+\varepsilon)-(3+\varepsilon,3+\varepsilon)]$, 
%and
%$](3+\varepsilon,3+\varepsilon)-
%(4+\frac{3}{2}\varepsilon,3+\varepsilon)]$. 
%If we project back on row 3, 
%we obtain three intervals: 
%$[(1,3)-(2,3)]$, 
%$[(3,3)-(3,3)]$, 
%and
%$](3,3)-(4,3)]$. 
%We say that these intervals are ``consistent''%
%\footnote{We need a better word here.} 
%with the obstacles.  
%The first and last intervals can be delt with similarly 
%to the example in Figure~\ref{fig::succ1} (left).  
%The second interval, being a single point, 
%requires to generate new intervals.  

The remaining question is what $f$ value 
should be associated with these intervals.  
For $A^*$ to return the optimal solution, 
we need $f$ to be an underestimate of the actual value 
(except for the final node).  
The interval has the following semantics: 
the path that is currently being constructed 
should start from $s$, cross the parent $P$, 
then cross the interval, and finally reached the goal.  
The shortest path that belongs to this set 
has length $g^*(P) + min_{x \in I}(d(P,x) + h^*(x))$.  
The value $g^*(P)$ should be known at this stage 
(this is one advantage of $A^*$).  
However, the minimum factor is unknown, 
but it can be lower-bounded by $min_{x \in I}(d(P,x) + h(x))$.  
This formula is actually pretty simple to solve, as we now demonstrate.  

Let $z_1$ and $z_2$ be the two extreme points of the interval.  
Draw a direct line between the goal and the parent $P$ of the interval, 
as shown in Figure~\ref{fig::fvalue}.  
If the line crosses the interval in \textit{\u z}, then the value is h(P).  If the line passes on the left, 
then the value is $d(P,z_1)+h(z_1)$; 
otherwise, it is $d(P,z_2)+h(z_2)$.  
If the goal is between the parent and the interval, 
as is the case with $g_4$, 
then one needs to consider the mirrored version of $g_4$ 
(here $g'_4$).  

\begin{figure}[ht]
  \begin{center}
%    \includegraphics[scale=0.3]{images/fvalue}
    \begin{tikzpicture}

\creategrid{6}{6}
%\drawobstacle{2}{3}

\draw[red,line width=2] (2,4) -- (4,4);

\coordinate (root) at (2,1);
\coordinate (g1)   at (3,5);
\coordinate (g2)   at (1,5);
\coordinate (g3)   at (5,5);

\foreach \g in {g1, g2, g3}
  \draw (root) -- (\g);

\path[name path=direct] (root) -- (g1);
\path[name path=interval] (2,4) -- (4,4);
\draw[name intersections={of=direct and interval,by=zsmile}];
\draw (zsmile) -- ++ (-0.2,0.2) + (-0.1,0.1) node {\textit{\u z}};

\draw[dashed,line width=2pt] (root) -- (g1);
\draw[dashed,line width=2pt] (root) -- (2,4) -- (g2);
\draw[dashed,line width=2pt] (root) -- (4,4) -- (g3);

\draw (g1)   + (-0.2,0.2) node {$g_1$};
\draw (g2)   + (-0.2,0.2) node {$g_2$};
\draw (g3)   + (-0.2,0.2) node {$g_3$};
\draw (root) + (-0.2,-0.2) node {$p$};

\coordinate (g4)   at (4,2);
\draw (g4)   + (+0.2,0.2) node {$g_4$};
\coordinate (g4mirror) at (4,6);
\draw (g4mirror) + (+0.2,-0.2) node {$g'_4$};
\path[name path=directtwo] (root) -- (g4mirror);
\draw[name intersections={of=directtwo and interval,by=zsmiletwo}];
\draw[dashed,line width=2pt] (root) -- (zsmiletwo) -- (g4);
\draw[dashed,line width=2pt] (root) -- (g4mirror);

\end{tikzpicture}

  \end{center}
  \caption{Computing the minimum distance 
    to go from $p$ through the interval to the goal; 
    if the goal is $g_1$, the direct distance between $p$ and $g_1$; 
    if the goal is $g_2$ or $g_3$, 
    the shortest distance is the distance represented by the dashed lines.}
  \label{fig::fvalue}
\end{figure}

% EOF
This section now presents our search algorithm, dubbed \anya, short for
any-angle pathfinding.  \anya{} is a variant of A* that does not search on the
space of intervals of the grid, but instead on the space of intervals and
corners.

\begin{algorithm}[ht!]
  \begin{algorithmic}
\STATE {\bf input}: Graph $G$, source $s$, target $t$
\STATE $open := \{s\}$
\LOOP
  \STATE $i := pop(open)$
  \IF{$i = t$}
    \STATE {\bf return} $i$
  \ENDIF
  \IF{$i$ is a node}
    \FORALL{$i' \in$ successors\_of\_corner($i$)}
      \STATE add\_edge($i$,$i'$)
    \ENDFOR
  \ELSE %\COMMENT{$i$ is an interval}
    \STATE {\bf let} $i = \langle [x_{\min},x_{\max}],y,p\rangle$.
    \STATE $i_1 :=$ move($i$)
    \STATE $cs :=$ corners\_of($i_1$)
    \COMMENT{Including target.}
    \STATE $is :=$ split($i_1,cs$)
    \FORALL{$c' \in cs$}
      \IF{$c'$ is visible from $p$}
        \STATE add\_edge($i$,$c'$)
      \ENDIF
    \ENDFOR
    \FORALL{$i' \in is$}
      \STATE {\bf let} $i' = \langle [x'_{\min},x'_{\max}],y',p\rangle$.
      \IF{$\langle \frac{x'_{\min}+x'_{\max}}{2},y'\rangle$ is visible from $p$}
        \STATE add\_edge($i$,$i'$)
      \ENDIF
    \ENDFOR
  \ENDIF
\ENDLOOP
\end{algorithmic}


  \caption{Procedure \anya, an any-angle pathfinding algorithm}
  \label{algo::anya}
\end{algorithm}

The general layout of \anya{} is given in Algorithm~\ref{algo::anya}.  
Like A*, \anya{} keeps a priority queue called $open$ 
and based on the $f$ value.  
The algorithm stops when the target is poped from the queue.  
Now, whether the current node is a corner or an interval 
is dealt with differently by \anya: 
\begin{itemize}
\item 
  For a corner, the successors of the corner are computed 
  (method successors\_of\_corner presented later) 
  and are added to the priority queue --- 
  the method add\_edge deals with computing the $f$ value 
  (presented later), updating if necessary the parent of a node, 
  and sorting the priority queue.  
\item 
  If the node is an interval, \anya{} moves it 
  and computes the corners of the new interval 
  (for this purpose, the target is seen as a corner).  
  Then, the interval is split according to the corners, 
  i.e., producing all the intervals between two consecutive corners.  
  All these corners and intervals 
  are potential successors of the current interval; 
  however, we need to check that they are visible from parent $p$.  
  To check whether an interval is visible, 
  we check whether its middle point is.  
\end{itemize}

We illustrate the computation of the successors 
of interval $i = \langle [1,2],3,p\langle$ 
where $p = \langle1,1\rangle$ 
(cf. Figure~\ref{fig::succ1}, right).  
The move of $i$ produces 
the interval $i' = \langle [1,2.5],4,p\rangle$.  
The set of corners of $i'$ is $\{\langle 2,4\rangle\}$.  
The split of $i'$ hence produces two intervals: 
$is = \{
\langle [1,2],4,p\rangle,  
\langle [2,2.5],4,p\rangle
\}$.  
The single corner is visible from $p$.  
The point $c' = \langle 1.5,4\rangle$ is also visible from $p$, 
which means that the first interval is visible from $p$.  
On the other hand, the point $\langle 2.25,4\rangle$ 
is not visible from $p$, 
which means that the second interval is not visible from $p$.  
The successors of $i$ are therefore $c'$ 
and $\langle [1,2],4,p\rangle$.  

\paragraph*{}

A corner $\langle x,y\rangle$ has potentially four successors.  
The first two obvious successors 
are the intervals at rows $y-1$ and $y+1$.  
However, the optimal path may follow a horizontal section, 
and the corner could have a successor on the left or on the right.  

\begin{algorithm}
  \begin{algorithmic}
\STATE {\bf input}: Graph $G$, corner $c = \langle x,y\rangle$, target $t$
\STATE $result := \{\}$
\STATE $left := x$
\WHILE{$\neg isObstacle([x,left-1])$}
  \STATE $left := left - 1$
\ENDWHILE
\STATE $right := x$
\WHILE{$\neg isObstacle([x,right])$}
  \STATE $right := right + 1$
\ENDWHILE
\STATE $i := \langle [(left,y+1),(right,y+1)], c\rangle$
\STATE {\bf return} $result$
\end{algorithmic}

  \caption{Computing the successors of a corner.}
  \label{algo::successorsofacorner}
\end{algorithm}

