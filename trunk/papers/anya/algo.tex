This section now presents our search algorithm, 
dubbed {\bf anya}, 
for any-angle pathfinding.  

\begin{algorithm}[ht!]
  \chapter{Any-angle Pathfinding}
\label{cha:anya}
In Chapters~\ref{cha:rsr} and \ref{cha:jps} we examined approaches for
improving the efficiency of pathfinding search on grid maps -- a common
setting for computer games and a domain which commonly appears in the AI
literature. 
In addition to minimising search time, a related problem in such settings
is minimising travel distance. For example: characters in a computer game 
must appear intelligent when navigating and should therefore prefer short
realistic-looking paths. However, paths which are computed on a grid map, 
even optimal paths, necessarily restrict movement to the fixed set of 
locations defined by the grid. 
\par
In this chapter we describe Anya: a new algorithm which addresses this
problem by computing \emph{any-angle} paths that do not have such constraints.
Anya operates on an input grid map but searches using intervals rather than 
the fixed points of the grid. 
From each such interval we select a representative point from which an
$f$-cost is computed. 
We prove that our approach maintains $A*$ expansion order and, unlike other
similar approaches, always returns the shortest possible path.
In the process we resolve an open question in the pathfinding community which
has been standing since at least 2007 and which has been the subject of
studies in the game development community for much longer.


  \caption{Procedure {\bf anya}, an any-angle pathfinding algorithm}
  \label{algo::anya}
\end{algorithm}

\begin{algorithm}[ht!]
  \begin{algorithmic}
\STATE {\bf input}: node $c$, target $t$
\IF{$c = t$}
\RETURN $c$
\ELSE
\RETURN  $\{
           \langle [0,W], y_c, c, \uparrow\rangle,
           \langle [0,W], y_c, c, \downarrow\rangle
         \}$  
\ENDIF
\end{algorithmic}

  \caption{Procedure $expand\_node(c)$}
  \label{algo::expandnode}
\end{algorithm}

The main procedure is presented in Algorithm~\ref{algo::anya}.  
The algorithm is a variant of $A^*$ 
with a specific successors generation.  
An interval can be generated from a corner $c$ 
(or the starting node $s$) 
through the procedure $expand\_node(c)$, 
or as the move of an existing interval.  
When an interval $i$ is expanded, 
it is splited as presented in the previous section 
via method $split\_epsilon(i)$.  
Finally, notice that the procedure $open.insert(I)$, 
where $I$ is a set of intervals, 
computes the $f$ value of each interval of $I$, 
and deals with ordering the intervals 
(including ignoring those that have already been expanded 
or those with existing smaller $f$ value).  

The $expand\_node(c)$ procedure is presented 
in Algorithm~\ref{algo::expandnode}.  
From the current node, 
the two intervals are created 
that stretch horizontally as far possible, 
at ordinate $y_c$, with origin $c$, and heading 
in vertical (North or South) direction.  

\begin{algorithm}[ht!]
  \begin{algorithmic}
\STATE {\bf input}: interval $\langle [x_{\min},x_{\max}], y, p, dir\rangle$
\STATE $I := \emptyset$, $C := \emptyset$
\STATE $x_1 := -1; x_2 := -1$
\STATE $obs := obstacle?(x_{\min}-1,y+dir)$
\IF{$\neg obs$}
\STATE $x_1 := x_{\min}$
\ENDIF
\FOR{$current\_x := x_{\min} ; current\_x \le x_{\max} ; current\_x++$}
  \STATE 
\ENDFOR
%
\WHILE{$\langle x_1, x_2\rangle := next\_interval([x_{\min},x_{\max}],y)$}
  \STATE $I := I \cup \{\langle [x_1,x_2], y, p, dir \rangle\}$
  \FORALL{$k \in \{1,2\}$}
    \IF{square from $\langle x_i,y\rangle$ 
      with horizontal direction $x_i - x$ and vertical direction $dir$ 
      is not an obstacle}
      \STATE $C := C \cup \{\langle x_i,y\rangle\}$
    \ENDIF
  \ENDFOR
\ENDWHILE
\RETURN $\langle I, C \rangle$
\end{algorithmic}

  \caption{Procedure $split\_epsilon(c)$}
  \label{algo::splitepsilon}
\end{algorithm}

$next\_interval([x_{\min},x_{\max}],y)$

$move(i)$

%EOF
