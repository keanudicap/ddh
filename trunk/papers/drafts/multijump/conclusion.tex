\section{Conclusion}
The conclusion goes here

We see several directions for improvement of this work.  
A first aspect is the construction of the table 
which is currently performed off-line but could be done on-line.  
In such a setting, PFS would first search for the jump points 
in the table, and if the table contains no entry, 
PFS would perform an on-line search and fill the table.  
This would considerably reduce the initial cost 
of computing the table.  
Observe furthermore that after the first jump, 
PFS stays in jump point locations, 
which means that most of the table 
is used only for the first move.  
An on-line generation of the table, 
or a table containing information only for jump point location, 
would contain much fewer entries 
which would reduce the memory overhead.  

Another problem is the update of the table 
in dynamic environment, 
for instance if an obstacle is removed 
or a bridge is destroyed.  
It is very likely that only a limited part of the table 
would be affected by the modification.  

% EOF
