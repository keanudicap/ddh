\section{Conclusion}
\label{sec::conclusion}

We study several techniques for improving Jump Point Search (JPS).  
The first improvement is motivated by the observation that JPS spends
a majority of its time scanning the grid for successors rather than
manipulating nodes from the open and closed list (i.e., searching). 
We give a new procedure to detect jump points more efficiently
by considering sets of nodes from the grid at one time (cf. one at a time).
The second improvement is motivated by the observation that most jump
points are goal-independent. We give a new pre-processing strategy
which computes and stores such jump points for every node on the map.  
Our third improvement is motivated by the observation that some
jump points are simply intermediary locations on the grid. We give a new 
pruning strategy that avoids expanding such nodes.

There are several interesting directions for further work. One possibility is
stronger pruning rules that will allow us to jump over some of the remaining
nodes in the graph. For example: we might consider %combining JPS with Fast
%Expansion~\cite{sun09}; i.e. we 
pruning a node $n$ if all of its successors have
an $f$-value that is not larger than $f(n)$. A stronger variant of this idea
is to keep jumping as long as we are heading in the same direction as when we
reached $n$ --- or in a new direction which is a component of the one used to
reach $n$. It is likely that this procedure will increase the branching factor
at $n$ but we posit that fewer node expansions will be required overall because
we do not need to stop each time the path turns due to an obstacle.

Combinations of JPS with existing grid-based speedup 
techniques appears to be another fruitful direction for further research. 
A number of well-known speedup techniques, both optimal and sub-optimal,
work by limiting the scope of grid search to a corridor of nodes relevant
for the instance at hand; e.g. Swamps~\cite{pochter10}, HPA{*}~\cite{botea04}
or any number of pruning-based heuristics e.g.~\cite{bjornsson06,goldenberg10}.
JPS can be trivially combined with such approaches to further speed up search.

One strength of the JPS family is that it performs very well even online.  We
have shown however that pre-computed jump points accelerate significantly the
search.  It is possible to combine both approaches, by populating a database
with jump points as they are discovered online.  We want to apply this feature
in dynamic environments where obstacles appear or disappear, immediately
rendering any pre-processed information obsolete.

% EOF
