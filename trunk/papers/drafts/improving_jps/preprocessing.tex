\section{Preprocessing}
\label{sec::preprocessing}
In the preceding section we have suggested a strategy for enhancing the
online performance of Jump Point Search.  In this section we give an offline
technique which can improve the algorithm further still. First, recall that 
Jump Point Search distinguishes between three different kinds of
nodes:
\begin{itemize}
\item Straight jump points. Reached by travelling in a
cardinal direction these nodes have at least one forced neighbour.
\item Diagonal jump points. Reached by travelling in a diagonal direction, 
these nodes have (i) one or more forced neighbours, or (ii) are 
intermediate turning points from which a straight jump point or the 
target can be reached.
\item The target node. This is a special node which JPS treats as a jump point.
\end{itemize}

We will precompute for every traversable node on the map the first
straight or diagonal jump point that can be reached by travelling 
away from the node in each of the eight possible cardinal or diagonal 
directions. During this step we do not identify any jump points that 
depend on a specific target node.  However, as we will show, these 
can be easily identified at run-time. 

The precomputation is illustrated on Figure~\ref{fig:preproc}(a).
The left side of the figure shows the precomputed jump points 
for node $N = \langle 4,5\rangle$.  
Nodes $1$--$3$ are typical examples of straight or diagonal 
jump points.  
The others, nodes $4$--$8$, would normally be discarded 
by JPS because they lead to dead-ends; we will remember them 
anyway but we distinguish them as \emph{sterile jump points} and never
generate them (unless they lead to the target).

Consider now Figure\ref{fig:preproc}(b), where $T$ is the target node.  
Travelling South-West away from $N$, JPS would normally identify 
$J$ as a diagonal jump point because it is an intermediate turning
point on the way to $T$.
However $J$ was not identified as a jump point during preprocessing 
because $T$ was unknown.  
Instead, the sterile jump point $S$ is recorded in the precomputed 
database. We use the location of $S$ to determine whether the jump from 
$N$ to $S$ crosses the row or column of $T$ (and where) and then test 
whether $T$ is reachable from that location. This procedure 
leads us to identify and generate $J$ as a diagonal jump point successor
of $N$. We apply the same intersection test more broadly -- to all 
successors of $N$. This is sufficient to guarantee both completeness 
and optimality. We call this revised pre-processing based algorithm 
JPS+.
%\\ \noindent
\subsection*{Properties}
%\textbf{Properties: }
JPS+ requires an offline pre-processing step that has worst-case quadratic 
time complexity and linear space requirements w.r.t the number of nodes in 
the grid. The time bound is very loose: it arises only in the case 
where the map is obstacle free and one diagonal jump can result in
every node on the map being scanned. In most cases only a small portion
of the total map needs to be scanned. Moreover, if we pre-compute using block-based 
symmetry breaking the entire procedure completes very quickly.
We will show that on our test machine (a midrange desktop
circa 2010) pre-processing never takes more than several hundred milliseconds,
even when the procedure is applied to large grid maps containing millions
of nodes.
%\\ \noindent
\subsection*{Advantages and Disadvantages}
%\textbf{Advantages and Disadvantages:} 
The main advantage of JPS+ is speed: instead of scanning the grid for jump
points we can simply look up the set of jump point successors of any given
location on the grid in constant time.  On the other hand, preprocessing has
two disadvantages (i) jump points need to be recomputed if the map changes
(some local repair seems enough) and (ii) it introduces a substantive memory
overhead: we need to keep for each node 8 distinct labels (one for each
successor).  In our implementation we use two bytes per label. The first 15
bits indicate the number of steps to reach the successor and the final bit
distinguishes the successor as sterile or not.
\par
We can use less memory if we store labels for intermediate locations instead
of actual jump points: for example using one byte per label we could jump up
to 127 steps at one time. The disadvantage of this approach is that more nodes
may be expanded during search than strictly necessary. A hybrid algorithm that
combines a single-byte database with a recursive jumping procedure is another
memory-efficient possibility.
