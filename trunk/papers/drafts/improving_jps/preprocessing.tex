\begin{figure*}[tb]
       \label{fig:preproc}
       \begin{center}
%		   \includegraphics[width=0.95\columnwidth]
%		{diagrams/preproc.png}
         \scalebox{.8}{\section{JPS+}
\label{sec:pre}

NEED TO SPELL OUT FORCED NEIS ONLY FOR STRAIGHT MOVES
%JPS+ is a previously unpublished search strategy
%which we introduce for the 2012 Grid-based Pathfinding Competition.
As we have discussed, jumping from one point to another in the grid
avoids many unnecessary A* queue operations. However, the identification
of these jump points then becomes the bottleneck of the algorithm. 
We therefore propose a preprocessing step during which we reformulate
the input map by replacing each adjacent neighbour of a node with
a jump point that lies in the same relative direction.
These off-line jump points are illustrated on Figure~\ref{fig::ojp}.  

\textbf{JPS vs JPS+:} 
Where JPS performs symmetry breaking online, JPS+ differs by breaking
symmetries during an offline preprocessing step. 
The precomputation runs in time linear to the size of the input graph
and produces a reformulated symmetry-reduced graph which is much faster
to search. JPS+ requires very little memory: the reformulated search space
is stored as an adjacency list that replaces the original input graph.
In cases where the input graph is also stored as an adjacency list, JPS+
can retain a zero memory overhead.


\begin{figure}[tb]
       \begin{center}
		   \includegraphics[width=0.95\columnwidth]
			{diagrams/preproc.png}
       \end{center}
	\vspace{-3pt}
       \caption{(a) A jump point is computed in place of each grid neighbour of node $x$.
		the set of neighbours from the grid. (b) In order to reach a jump point $y$ from $x$ it may be
necessary to cross the row or column of the target $t$ (here, both). To make sure we do not jump over 
the target we insert an intermediate successor $y'$ on the row or column of the target (whichever is closest to $x$).}
       \label{fig:preproc}
\end{figure}

The off-line jump points are computed similarly to the on-line jump points 
with the difference that a jump point is defined before each obstacle is hit.  
This addition is necessary 
because the goal state is unknown at preprocessing time.  

At runtime, the algorithm does not need to search for the next jump points 
and can simply find them in the precomputed table.  
Additional tests are however necessary to account for the goal.  

If the current move is a straight move 
and passes over the goal, 
then the jump is stopped at the goal.  

If the current move is a diagonal move 
and crosses the row or the column that contain the goal, 
then a test is performed to check 
whether the goal is visible from the point 
where the jump crosses the line.  
If the goal is indeed visible, 
then the jump is stopped at the crossing point; 
otherwise, the jump continues to the precomputed point.  

On the other hand, the off-line jump point are more numerous 
that the on-line jump point, 
since a jump point is defined even when an obstacle is hit.  
However, it is possible to identify these spurious jump points 
and to avoid inserting them in the queue list.  
DANIEL: YOU PROBABLY WANT TO WRITE SOMETHING ABOUT THIS PARAGRAPH.  
(HOW) DO YOU IDENTIFY THESE SPURIOUS JUMP POINT?  
ARE YOU SURE YOUR IDENTIFICATION PROCEDURE IS CORRECT?  

The benefit of this approach is to remove the computation of the jump point 
and replace it with a simple table lookup.  
The cost of the operation is the additional checks 
presented in the previous paragraphs, 
but it is very limited (much smaller than the gain).  
The preprocessed table induces a limited (linear) memory overhead 
(eight distances for each node in the graph).  
Finally, the preprocessing itself is very low, 
typically in the order of a few seconds.  

I DON'T KNOW WHETHER WE SHOULD MENTION POTENTIAL IMPROVEMENTS HERE.  
FOR INSTANCE, STORING THE TABLE ONLY FROM A JUMP POINT 
(I.E., SEARCH REQUIRED ONLY FOR THE START NODE); 
FILLING THE TABLE DURING THE SEARCH, 
UPDATING THE TABLE WHEN OBSTACLE ARE ADDED/REMOVED.  
}
       \end{center}
%	\vspace{-3pt}
       \caption{(a) A jump point is computed in place of each grid neighbour of node $x$.
		(b) When jumping from $x$ to $y$ we may cross the row or column of the target $t$ (here, both). 
To avoid jumping over $t$ we insert an intermediate successor $y'$ on the row or column of $t$ (whichever is closest to $x$).}
\end{figure*}

\section{Preprocessing}
\label{sec::preprocessing}

In the preceeding section we have suggested a strategy for enhancing the
online performance of Jump Point Search.  In this section we give an offline
pre-processing technique which can improve the algorithm further still.

Recall that Jump Point Search distinguishes between two different kinds of
nodes:
\begin{itemize}
\item Straight jump points. Reached by travelling in a
cardinal direction these nodes have at least one forced neighbour.
\item Diagonal jump points. Reached by travelling in a diagonal direction, 
these nodes have (i) one or more forced neighbours, or (ii) are 
intermediate turning points from which a straight jump point or the 
target can be reached.
\item The target node. This is a special node which JPS treats as a jump point.
\end{itemize}

We will precompute for every traversable node on the map the first
straight or diagonal jump point that can be reached by travelling 
away from the node in each of the eight possible cardinal or diagonal 
directions. During this step we do not identity any jump points that 
depend on a specific target node.  However, as we will show, these 
can be easily identified at run-time. 

%For any node, 
%the jump points of the first type and some of the last type 
%can be identified a priori (before the target is known); 
%they are stored in an adjacency list data structure 
%and retrieved in constant time at runtime.  
%The jump points of the second type and the rest of those of the last type 
%cannot be found at pre-process, 
%but it is easy to identify them at runtime: 
%one just needs to check whether the current jump 
%crosses the row or column of the target.  
%Notice however that the jump must be attempted 
%even if it leads to an obstacle.  

The precomputation is illustrated on Figure~\ref{fig:preproc}(a).
The left side of the figure shows the precomputed jump points 
for node $N = \langle 4,5\rangle$.  
Nodes $3$---$5$ are typical examples of straight or diagonal 
jump points.  
The others, nodes $1, 2$ and $6$ --- $8$, would normally be discarded 
by JPS because they lead to dead-ends; we will remember them 
anyway but we distinguish them as \emph{sterile jump points}.  

Consider now Figure\ref{fig:preproc}(b), where $T$ is the target node.  
Travelling South-East away from $N$, JPS would normally identify 
$J$ as a diagonal jump point because it is an intermediate turning
point on the way to $T$.
However $J$ was not identified at as a jump point during preprocessing 
because $T$ was unknown.  
Instead, the sterile jump point $S$ is recorded in the precomputed 
database. We use the location of $S$ to determine whether the jump from 
$N$ to $S$ crosses the row or column of $T$ (and where) and then test 
whether $T$ is reachanble from that location. This procedure 
leads us to identify and generate $J$ as a diagonal jump point successor
of $N$. We apply the same intersection test more broadly -- to all 
successors of $N$. This is sufficient to guarantee both completeness 
and optimality. We call this revised pre-processing based algorithm 
JPS+.

\subsection*{Properties}
JPS+ requires an offline pre-processing step that has worst-case quadratic 
time complexity and linear space requirements w.r.t the number of nodes in 
the grid. The time bound is very loose: it arises only in the case 
where the map is obstacle free and one diagonal jump can result in
every node on the map being scanned. In most cases only a small portion
of the total map needs to be scanned. Moreover, if pre-compute using block-based 
symmetry breaking the entire procedure completes much faster still. 
Infact we will show that on our test machine (a midrange desktop
circa 2010) pre-processing never takes more than several hundred milliseconds,
even when the procedure is applied to large grid maps containing millions
of nodes.

\subsection*{Advantages and Disadvantages}
The main advantage of JPS+ is speed: instead of scanning the grid for
jump points we can simply look up the set of jump point successors
of any given location on the grid in constant time.
On the other hand, preprocessing has two disadvantages (i) jump points need to be recomputed
if the map changes (some local repair seems enough) and (ii) it introduces 
a substantive memory overhead: we need to keep for each node 8 distinct 
identifiers (one for each successor). 
The overhead can be as large as $4\times8$ bytes per node. In our
implementation we have found it sufficient to store only three bytes 
per successor: the first 23 bits store the id and the final bit distinguishes
each successor as sterile or not. 
\par
We can use less memory if we
store intermediate locations instead of actual jump points: for example
using one byte per successor we could jump up to 127 steps 
at one time. The disadvantage of this approach is that more nodes may be 
expanded during search than strictly necessary. A hybrid 
algorithm that combines a pre-computed database with a recursive
jumping procedure is another memory-efficient possiblity.
