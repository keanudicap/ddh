\section{Optimisations}
\label{sec::optimisations}

\subsection{Lazy Node Generation}
When generating non-observable successors Anya constructs a maximal-size
interval of non-observable points and slices it up into smaller intervals
by dividing them at the corner points. This does not seem necessary.
We could just extend an interval to the first corner point and wait
for it to be expanded before generating other intervals further along
the current row. 


\subsection{Jumping}
The algorith we have described until now proceeds by expanding nodes
from one row of the grid and generating successors from an immediately
adjacent row. This row-by-row procedure can be improved by observing
that there are only two kinds of interesting intervals:
those which contain corner points and those which contain the goal.
We propose to only expand interesting search nodes while jumping
over the rest. The idea is simple:
if an interval does not contain the goal or any corner point we
do not generate it. Instead, we recurse and generate its immediate
neighbours instead. The idea here is to be able to jump several
rows at one time. For example in Figure~\ref{fig::anya_example} (Left)
we would like to be able to expand $I_1$ and generate as its successors
the intervals $I_3$ and $I_6$. The other intermediate intervals, such
as $I_2$, $I_3$, $I_4$ and $I_5$ are not interesting.
This idea has some similarities with Jump Point Search~\cite{harabor11b}
and other related algorithms~\cite{haraborG14}.

\subsection{Other redundant nodes}
Anya seems to allow projecting an interval to the next row when the interval and the root are
on the same row. This seems redundant. There should already be an interval generated previously.
We should clean this up and avoid generating such nodes.

