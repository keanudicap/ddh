\section{Choosing Representative Points}
\label{sec::representatives}

Once we have generated all successors we need to evaluate
each one and assignt to it a single $f$-value. This operation
involves finding a single point point to represent all points
in some interval $I$. In particular we want to always choose
as the representative the point with minimum $f$-value.
Choosing such a point is easy:
we simply draw a straight line between the associated root 
point $r$ and the target point $t$. The position of the line 
relative to $I$ allows us to identify in constant time
the correct representative.
This procedure is described in the original paper. Refer to 
that for an overview of what is required~\cite{haraborG13}.
