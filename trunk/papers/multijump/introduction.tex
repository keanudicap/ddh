\section{Introduction}

Path planning in grid is a critical problem for many applications 
ranging from robotics to video-games, 
where the optimal path 
between a unit location and its target must be found quickly.  
% TODO: Explain that in many games, path search accounts to 70% of AI 
% and that there can be 100s of search requests.  

The optimal algorithms are based on $A^*$, 
which develops several paths in parallel 
and uses a heuristic to determine 
which path is the most promising 
and should be developed next.  
Harabor and Grastien \cite{harabor11b} recently proposed 
a variant of $A^*$ called \emph{Jump Point Search} (JPS).  
The grid yields many symmetries that can be exploited 
by avoiding the expansion of certain nodes 
and ``jumping'' from the current node to so-called jump points.  
JPS is optimal and requires no memory overhead, 
whilst it has proved more efficient 
than the state-of-the-art methods.  

JPS's main advantage is that much fewer nodes are expanded, 
but each expansion is more expensive.  
In our first contribution, we show 
how the jump points can be pre-processed 
so as to reduce the expansion time.  
This operation is not trivial, 
as the ``on-line'' computation of the jump points 
automatically eliminates a number of potential points 
that cannot be disregarded off-line (when the target is unknown).  
The pre-computation requires only limited time 
and the size of the pre-computed table is linear in the number of nodes 
which makes the approach quite attractive.  

The second extension is based on the observation, 
that the JPS works very well in open environments, 
whilst it requires to expand many nodes around ``corners''.  
To overcome this limitation, 
we propose a multi-step version of JPS, 
in which we look-ahead to determine 
whether it is necessary to stop at a given jump point, 
or whether it is possible to push the search further.  

These extensions are tested with several benchmarks 
both synthetic and from real video games, 
exhibiting different characteristics.  

The paper is organised as follows.  

% EOF
