\section{Related Work}

Problem definition

Most optimal methods for path search in uniform-cost grids are based on $A^*$.  
Starting with the source, 
the most promising node is expanded 
(i.e., all its neighbours are added to the list of nodes to expand).  
The most promising node is defined as the node 
that minimzes the sum 
between the shortest distance from the source to the node 
(known when the node is expanded) 
and an (under-)estimate of the distance between the node and the target.  

Many extensions have been proposed for $A^*$.  
To reduce the number of nodes expanded, 
Botea et al. \cite{botea04} proposed a hierarchical approach
where the search is first performed at an abstract level 
and then refined.  
The resulting path is not optimal in general, 
albeit often optimal.  

The precision of the estimate of the distance 
between any node and the target 
affects greatly the number of nodes that need to be expanded.  


\subsection{Jump Point Search}

Our work is based on the Jump Point Search (JPS) approach.  
JPS uses the following observations about uniform grid path planning: 
\begin{itemize}
\item 
  Since we are interested only in optimal paths, 
  given a node and its predecessor, 
  the potential successors of the node are much fewer 
  that eight (generally three and, in some instances, four).  
\item 
  Using additional symmetry (for instance, the fact that going North-West 
  and then North is identical to going North and then North-West), 
  the number of successors can be further reduced (often to one, sometimes to zero).  
\item 
  When there is only one successor, 
  this successor can be immediately processed (this is the ``jump'').  
\end{itemize}

In JPS, the search works as follows: 
from a node $n$, there are generally three successors

% EOF 
