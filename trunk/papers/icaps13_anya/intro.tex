\section{Introduction}
Any-angle pathfinding is a navigation problem which appears in robotics
and computer video games. It involves finding a shortest path between an 
arbitrary pair of points on a two-dimensional grid map but asks that 
movement along the path is not artificially constrained to the points of 
the grid.  Within the game development community a simple and popular 
solution exists known as \emph{string pulling}~\cite{pinter01,botea04}.
The idea is to compute a grid-optimal path in the first
instance and smooth the result as part of a post-processing step that improves
both its length and aesthetic appeal. String pulling has two disadvantages: 
(i) it requires more computation than just finding a path (ii)
it only yields approximately shortest paths.

A number of algorithms improve on string-pulling by integrating post-processing
into node-expansion during search. Field D*~\cite{ferguson05} uses linear 
interpolation to smooth paths one grid cell at a time. 
Theta*~\cite{nash07} 
%and its variants %\cite{%nash09, nash10,munoz12}
introduces a shortcut each time a successful line-of-sight check
is made from the parent of the current node to any of its successors.
Meanwhile, Block A*~\cite{yap11} employs during search a pre-computed database
of optimal Euclidean distances between pairs of points in a localised area.
Each of these approaches improves on string pulling in terms of solution 
quality and, in many cases, running time. Unfortunately none are optimal.
Accelerated A*~\cite{sislak09b} is an any-angle algorithm that is conjectured 
to be optimal but for which no strong theoretical argument is made. Similar to Theta*, 
it differs primarily in that line-of-sight checks are performed from a set
of expanded nodes rather than a single ancestor. The size of the set is only
loosely bounded and, for challenging problems, can include a large proportion
of nodes on the Closed List.

%
% Its main disadvantage is that,
%for challenging problems, it can perform line-of-sight checks against a large 
%proportion of nodes on the Closed List.
%
%Theta*~\cite{nash07} and its variant~\cite{%nash09,
%nash10} improve things 
%by integrating post-processing into node expansion during search. Their 
%idea involves updating a node's parent label following a successful line-of-sight
%check to a previously expanded ancestor node.
%Another approach, Block A*~\cite{yap11}, avoids line-of-sight checks entirely 
%by precomputing a database of exact costs between pairs of points in a localised area.
%Both Theta* and Block A* improve on the running time and 
%solution quality of string pulling but neither guarantees optimality.
%Despite this, a number of exact solutions for the problem do exist.
%\\
%In computational geometry a problem similar to any-angle pathfinding exists 
%which has been very well studied: finding Euclidean shortest paths among 
%polygonal obstacles in the plane.
Tangent Graphs~\cite{liu92} and Visibility Graphs~\cite{lozanoperez79} are 
optimal techniques that can solve a generalised form of the any-angle pathfinding 
problem. Their primary disadvantage is that each such graph requires quadratic 
space in the worst case and must be computed offline.
Other exact approaches are based on the 
Continuous Dijkstra~\cite{mitchell87} paradigm.
The most efficient of these algorithms~\cite{hershberger99} pre-computes a 
planar subdivision of the map that can be used to extract a path in just
logarithmic time. Unfortunately the precomputation assumes the starting location
does not change.

In this work we introduce a new approach to any-angle pathfinding 
which addresses many of the shortcomings associated with existing research.
Our method, Anya, bears some similarity with Continuous Dijkstra: 
instead of searching over the individual nodes of the grid we 
search over contiguous sets of states that form intervals.
Each interval has a representative point used to derive an $f$-value
and each is projected from one row of the grid onto another until the 
goal is reached.
%instead of searching over individual states from the grid we consider
%contiguous sets of states together as an interval. From each interval we select a 
%representative point that is used to derive an $f$-value for the set.
%Intervals are associated with corner points and projected from one 
%row of the grid onto another until the goal is reached.
Anya does not rely on any precomputation, does not introduce any
memory overheads (beyond what is required by e.g. A*) and always finds 
an optimal any-angle path.
%compares favourably with existing research: (i)
%it always finds an optimal any-angle path (ii) 
%it does not rely on any precomputation (iii) 
%it does not introduce any memory overheads 
%beyond those required by a pathfinding algorithm such as A*.

%http://www.valvesoftware.com/publications/2009/ai_systems_of_l4d_mike_booth.pdf
%http://digestingduck.blogspot.com.au/2010/03/simple-stupid-funnel-algorithm.html
