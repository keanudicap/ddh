\section{Related Work}
\label{sec:relatedwork}
In the presence of symmetry, search algorithms often waste time evaluating many
equivalent states and make little real progress toward the goal.  The problem of
how to deal with symmetry has received significant attention in other parts of
the literature (e.g.~\cite{rossi06}) but few studies focus on symmetry in
pathfinding domains, such as grid maps.  We are only aware of Empty Rectangular
Rooms \cite{harabor10}: an offline symmetry-breaking technique limited to
4-connected uniform-cost grid maps; We discuss the connection with RSR in the
next section.
\par
The \emph{dead-end heuristic} \cite{bjornsson06} and \emph{Swamps} \cite{pochter10} 
are two closely-related pruning techniques
that identify areas in the search space not relevant for reaching the goal. 
This is a similar yet complementary goal
to RSR, which tries to reduce the search effort involved in exploring any given
area. 
\par
The \emph{gateway heuristic} \cite{bjornsson06} and the \emph{portal heuristic}
\cite{goldenberg10} are two typical memory-based techniques for optimal 
pathfinding on grids. 
%Like RSR, both decompose the map into a series of adjacent areas and both
%pre-compute a database of exact distances between all pairs of nodes that
%transition from one area to another.  
The main idea is to reduce the number of
A* node expansions by improving the accuracy of cost-to-go estimates during
search.  The portal heuristic also identifies, online, areas
not relevant to the pathfinding instance at hand.
\par
\emph{Contraction Hierarchies} \cite{geisberger08} is a method for
very fast optimal pathfinding on road networks. 
Based on a combination of Dijkstra's algorithm and memory-intensive
abstractions, this approach relies on the existence of ``highway edges'' that appear
on most shortest paths between nodes.
Largely orthogonal to RSR, there is little work applying these ideas to searching on grid maps.
One recent result \cite{sturtevant10} suggests they are less effective when the 
underlying graph contains a high degree of path
symmetry.
