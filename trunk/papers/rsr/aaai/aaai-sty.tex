\documentclass{article}
\def\BibTeX{Bib\TeX}
\parindent=0pt
\parskip=\baselineskip
\begin{document}
\title{A Quick Guide to the AAAI Style for \LaTeX2e}
\author{George Ferguson \\ \texttt{ferguson@cs.rochester.edu}} 
\date{April 8, 1999 (Revised in  2009 by the AAAI Staff)}
\maketitle
\begin{abstract}
\noindent This document provides a very brief overview of the AAAI style file
for \LaTeX2e. It is assumed that you already understand \LaTeX\ in
general, and that you are familiar with the AAAI ``Instructions to
Authors.'' This document just tells you how to use the
\verb+aaai.sty+ style file to satsify the instructions.
\end{abstract}
\section{Obtaining the Style File}
The latest version of the style file is available via the World Wide
Web from
\begin{verbatim}
http:/www.aaai.org/Publications/Templates/aaai.sty
\end{verbatim}
You should download this file and place it in a file named
\verb+aaai.sty+ in the \TeX\ search path. Placing it in the same
directory as the paper should also work (unless you have done
something silly with your search paths).

You should also download the AAAI Press author kit. The author kit, containing all the documents and files necessary for format your paper for an AAAI Press proceedings or technical report,
is available from
\begin{verbatim}
http:/www.aaai.org/Publications/Templates/AuthorKit.zip
\end{verbatim}

\section{Document Preamble}
At the top of your \LaTeX\ source for your paper, copy all the the following lines before \verb+\begin{document}+:
\begin{verbatim}
\documentclass[letterpaper]{article}
\usepackage{aaai}
\usepackage{times}
\usepackage{helvet}
\usepackage{courier}
% %%%%%%%%%%%%%%%%%%%%%%%%%%%%%%%%%%%%%%%%%%%%%%%%%%%%%%
% PDFMARK for TeX and GhostScript
% Uncomment and complete the following for metadata if
% your paper is typeset using TeX and GhostScript (e.g
% if you use .ps or .eps files in your paper):
% \special{! /pdfmark where
% {pop} {userdict /pdfmark /cleartomark load put} ifelse
% [ /Author (John Doe, Jane Doe)
% /Title (Paper Title)
% /Keywords (AAAI, artificial intelligence)
% /DOCINFO pdfmark}
% %%%%%%%%%%%%%%%%%%%%%%%%%%%%%%%%%%%%%%%%%%%%%%%%%%%%%%
% PDFINFO for PDFTeX
% Uncomment and complete the following for metadata if
% your paper is typeset using PDFTeX
% \pdfinfo{
% /Title (Input Your Title Here)
% /Subject (Input The Proceedings Title Here)
% /Author (First Name, Last Name;
% First Name, Last Name;
% First Name, Last Name;)
% }
% %%%%%%%%%%%%%%%%%%%%%%%%%%%%%%%%%%%%%%%%%%%%%%%%%%%%%%
% Uncomment only if you need to use section numbers
% and change the 0 to a 1 or 2
% \setcounter{secnumdepth}{0}
% %%%%%%%%%%%%%%%%%%%%%%%%%%%%%%%%%%%%%%%%%%%%%%%%%%%%%%
\title{Title}
\author{Author 1 \and Author 2 \\ Address line \\ Address line \And
Author 3 \\ Address line \\ Address line}
\end{verbatim}
\section{Preparing your Paper}
After the preamble above, you should prepare your paper as follows:

\begin{verbatim}
\begin{document}
\maketitle
...
\bibliography{Bibliography-File}
\bibliographystyle{aaai}
\end{document}
\end{verbatim}

Various alternatives and additional commands are described below.
\subsection{Formatting Author Information}
Author information can be set in various styles:
\begin{itemize}
\item For several authors from the same institution, use \verb+\and+:
\begin{verbatim}
\author{Author 1 \and ... \and Author n \\
Address line \\ ... \\ Address line}
\end{verbatim}
If the names do not fit well on one line use:
\begin{verbatim}
\author{Author 1 \\ {\bf Author 2} \\ ... \\ {\bf Author n} \\
Address line \\ ... \\ Address line}
\end{verbatim}
\item For authors from different institutions, use \verb+\And+:
\begin{verbatim}
\author{Author 1 \\ Address line \\ ... \\ Address line
\And ... \And
Author n \\ Address line \\ ... \\ Address line}
\end{verbatim}
\item To start a separate ``row'' of authors, use \verb+\AND+:
\begin{verbatim}
\author{Author 1 \\ Address line \\ ... \\ Address line
\AND
Author 2 \\ Address line \\ ... \\ Address line \And
Author 3 \\ Address line \\ ... \\ Address line}
\end{verbatim}
\end{itemize}
If the title and author information does not fit in the area
allocated, place
\begin{quote}
\verb+\setlength\titlebox{+\emph{height}\verb+}+
\end{quote}
after the \verb+\documentclass+ line where \emph{height} is
something like \verb+2.5in+.
\subsection{Adding Metadata to Your PDF}
AAAI requires that you add author and title metadata to your PDF. There are two ways to do this and both have been included in the preamble example listed previously. The option you choose will depend on whether you use GhostScript or PDFTeX to create your paper.

If you have no illustrations, or all of your illustrations are compatible with PDFTeX (e.g. they are PDF or another compatible format), you should use PDFTeX. To add metadata to your
paper, you should put uncomment following in your document preamble.
\begin{verbatim}
\pdfinfo{
/Title (Input Your Title Here)
/Subject (Input The Proceedings Title Here)
/Author (First Name, Last Name;
First Name, Last Name;
First Name, Last Name;)
}
\end{verbatim}

You should include your title in mixed case, and include all your authors exactly as they appear in your paper. Separate authors with semicolons.

If your paper includes illustrations that are not compatible with PDFTex (such as.eps or .ps documents), you will be using GhostScript to create your PDF. To add metadata to your paper, you should uncomment the following in your document preamble
\begin{verbatim}
\special{! /pdfmark where
{pop} {userdict /pdfmark /cleartomark load put} ifelse
[ /Author (John Doe, Jane Doe)
/Title (Paper Title)
/Keywords (AAAI, artificial intelligence)
/DOCINFO pdfmark}
\end{verbatim}

You should include your title in mixed case, and include all your authors exactly as they appear in your paper. Separate authors with commas.

\subsection{Adding Section Numbers}
By default, aaai.sty will not add section numbers to your paper. If you require
section numbers, uncomment the following line in your document preamble 
and change the 0 to a 1 or 2. Do not enter a number higher than 2 as the style
file will not work properly above 2.
\begin{verbatim}
\setcounter{secnumdepth}{0}  
\end{verbatim}
\subsection{Adding Acknowledgements}
To acknowledge other contributors, grant support, or whatever, use
\verb+\thanks+ in either the \verb+\author+ or \verb+\title+ commands.
For example:
\begin{verbatim}
\title{Very Important Results in AI\thanks{This work is
 supported by everybody.}}
\end{verbatim}
Multiple \verb+\thanks+ commands can be given. Each will result in a
separate footnote indication in the author or title with the
corresponding text at the bottom of the first column of the document.
For example:
\begin{verbatim}
\author{A. Researcher\thanks{Now at Microsoft.} \and
B. Researcher\thanks{Not at Microsoft.}}
\end{verbatim}
One common error with \verb+\thanks+ is forgetting to use
\verb+\protect+ on what \LaTeX\ calls ``fragile'' commands.
\subsection{Adding a Publication Note}
To add a comment to the header of document, use \verb+\pubnote+, as in:
\begin{verbatim}
\pubnote{\emph{To appear, AAAI-09}}
\end{verbatim}
This should be placed after the title and author information but
before \verb+\maketitle+. Note that \verb+\pubnote+ is for printing
the paper yourself, and should not be used in submitted versions!
\subsection{Copyright Information}
By default, the AAAI copyright slug will be printed at the bottom of
the first column of your document. To not print any copyright slug,
use \verb+\nocopyright+ somewhere before \verb+\maketitle+. To change
the year in the copyright slug from the current year, use:
\begin{quote}
\verb+\copyrightyear{+\emph{year}\verb+}+
\end{quote}
To change the entire text of the copyright slug, use:
\begin{quote}
\verb+\copyrighttext{+\emph{text}\verb+}+
\end{quote}
Either of these must appear before \verb+\maketitle+.
\section{Bibliography Style and References}
The \verb+aaai.sty+ file includes a set of definitions for use in
formatting references with \BibTeX. These definitions make the
bibliography style closer to the one specified in the ``Instructions
to Authors'' for AAAI papers.
To use these definitions, you also need the \BibTeX\ style file
\verb+aaai.bst+, available from wherever you obtained \verb+aaai.sty+.
Then, at the end of your paper but before \verb+\end{document}+, you
need to put the following lines:
\begin{verbatim}
\bibliographystyle{aaai}
\bibliography{bibfile1,bibfile2,...}
\end{verbatim}
The list of files in the \verb+\bibliography+ command should be the
names of your \BibTeX\ source files (that is, the \verb+.bib+ files
referenced in your paper).
The following commands are available for your use in citing
references:
\begin{description}
\item \verb+\cite+: Cites the given reference(s) with a full citation.
This appears as ``(Author Year)'' for one reference, or ``(Author Year;
Author Year)'' for multiple references.
\item \verb+\shortcite+: Cites the given reference(s) with just the year.
This appears as ``(Year)'' for one reference, or ``(Year; Year)''
for multiple references.
\item \verb+\citeauthor+: Cites the given reference(s) with just the
author name(s) and no parentheses.
\item \verb+\citeyear+: Cites the given reference(s) with just the
fate(s) and no parentheses.
\end{description}
%\section{A Note on Printing}
%Some laser printers have a serious problem printing \TeX\ output. These
%printing devices, commonly known as ``write-white'' laser printers,
%tend to make characters too light. To get around this problem, a
%darker set of fonts must be created for these devices.
\section*{Acknowledgements}
Originally prepared by Peter F. Patel-Schneider, liberally using the
ideas of other style hackers, including Barbara Beeton. This style is
NOT guaranteed to work. It is provided in the hope that it will make
the preparation of papers easier. There are undoubtably bugs in this
style. If you make bug fixes, improvements, etc. please let me know.
My e-mail address is: \verb+pfps@research.bell-labs.com+. (Please also
notify AAAI.)
The preparation of this file was supported by Schlumberger Palo Alto
Research, AT\&T Bell Laboratories, AAAI, and Morgan Kaufmann Publishers.
Bibliography style changes added by Sunil Issar, \verb+si@cs.cmu.edu+.
\verb+\pubnote+ added by J. Scott Penberthy.
George Ferguson, \verb+ferguson@cs.rochester.edu+, added support for
printing the AAAI copyright slug and pieced together this document
from the comments in \verb+aaai.sty+. Additional changes to aaai.sty and aaai.bst
were made by the AAAI staff in 2007 and 2009.
\section*{Restrictions on Further Use}
These instructions can be modified and used in other conferences as
long as credit to the authors and supporting agencies is retained,
this notice is not changed, and further modification or reuse is not
restricted.
\end{document}
