\section{Conclusion}
We introduce RSR, a new search space reduction algorithm applicable to
pathfinding on uniform cost grid maps. RSR is fast, memory efficient,
optimality preserving and can, in some cases, eliminate entirely the need
to search.  
When running on a grid pruned by RSR, A* is up to 38 times faster than
otherwise.
\par
We compare RSR with a range of search space reduction algorithms from the
literature. Compared to 4ERR~\cite{harabor10}, on which it is based, RSR is
shown significantly faster on the set of instances where both methods can be
applied (i.e. 4-connected maps).  Next, we compare RSR to Swamps-based
pruning~\cite{pochter10} and show that the two algorithms have complementary
strengths.  We find that Swamps are more useful on maps with small open areas
while RSR becomes more effective as larger open areas are available on a map. We
also identify a broad range of instances where RSR dominates convincingly and is
clearly the better choice.  Finally, we compare RSR to the enhanced Portal
Heuristic~\cite{goldenberg10}.  We show that our method exhibits similar or
improved performance but requires up to 7 times less memory.  As with Swamps, we
find that the two ideas are complementary and could be easily combined.

%Other pathfinding techniques,
%such as hierarchical abstraction and memory-based heuristics,
%While also effective, both of these methods have their shortcomings.
%For example, hierarchical algorithms such as HPA*~\cite{botea04} and 
%PRA*~\cite{sturtevant05} are fast and memory-effective but produce 
%sub-optimal paths.
%Memory-based admissible heuristics preserve the optimality of 
%solutions but carry a significant memory overhead~
%\cite{sturtevant09,goldberg05,Cazenave:06}. %,bjornsson06}.
%By comparison, RSR is not only fast and optimal but also has very little
%memory overhead.

%Recent work~\cite{pochter10,harabor10} has introduced
%techniques that are fast, produce optimal solutions and
%require little additional memory.
%In this paper we have introduced RSR,
%an algorithm that converts an initial map
%into a smaller search graph without sacrificing the optimality of solutions.
%Similarly to \citeauthor{harabor10}'s approach~\cite{harabor10},
%our algorithm uses a decomposition of a map into disjoint rectangular
%rooms with the property that no room contains obstacle tiles.
%All interior tiles and some of the perimeter tiles 
%(except for start and target locations)
%are pruned from search.


%OPTIONAL: 
%Have here a brief summary of speed-up numbers:
%Overall, the best speed-ups we have observed for RSR are blah.
%On the same data, swamps achieve a maximal speed-up of blah.
%Harabor and Botea's method has a speed-up of blah.

\par
Future work includes reducing the branching factor in RSR further through the 
development of better map decompositions and stronger online node pruning
strategies.
%We have observed cases where the number of expanded nodes is reasonably small
%but a relatively large branching factor increases the number of visited nodes significantly.
Another interesting topic is combining RSR with Swamps or the Portal Heuristic.
%We are also interested in applying our ideas to more general types of graphs that exhibit
%(local) symmetry, such as the search graphs of planning instances.

%Perimeter Search is a symmetric path elimination technique which extends the the
%work of \citeauthor{harabor10}~\cite{harabor10} from 4-connected grid maps
%to the much more common 8-connected case.
%We speed up optimal pathfinding by applying an offline decomposition algorithm to
%divide a grid map into a series of empty rectangular rooms.
%We then show that it is possible to traverse optimally from one side of a room
%to another without exploring any tiles from the interior of any room.
%Our experiments show that in the presence of large rooms or wide open areas we can 
%compute optimal paths very quickly: up to 16 times faster, on average for 4-connected maps and up to
%6.5 times faster, on average, for 8-connected maps.
%On less favourable map topographies we achieve more modest improvements.
%Our method is simple to understand, very effective and can be combined with
%existing speedup techniques such as memory heuristics or hierarchical methods;
%for example as described in \cite{botea04,bjornsson05,bjornsson06}. 
%\par
%One direction for future work is to investigate alternative decomposition
%algorithms which produce bigger rooms and improve the performance of the current
%method.

