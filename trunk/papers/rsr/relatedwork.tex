\section{Related Work}
\label{sec:relatedwork}
In the presence of symmetry, search algorithms often
evaluate many equivalent states and make little real progress toward the goal.
The topic of how best to deal with symmetry has received significant attention
in other parts of the literature~\cite{rossi06} but there are very few works
that explicitly identify and deal with symmetry in pathfinding domains 
such as grid maps. The only work of which we are aware is Empty Rectangular Rooms
\cite{harabor10}: a symmetry breaking technique specific to 4-connected
uniform-cost grid maps which we refer to as 4ERR.  We discuss the main
differences between 4ERR and RSR in Sections \ref{sec:introduction} and
\ref{sec:rsr}.
\par
The \emph{dead-end heuristic} \cite{bjornsson06} and \emph{Swamps-based
pathfinding} \cite{pochter10} are two closely-related pruning techniques
that identify areas in the search space not relevant for reaching the goal. 
This is a similar yet complementary goal
to RSR, which tries to reduce the search effort involved in exploring any given
area. Both methods decompose the map into a
series of obstacle-free areas. A preliminary
online search in the decomposed graph is then used to identify areas that can 
be ignored during a subsequent search in the original grid.
\par
The \emph{gateway heuristic} \cite{bjornsson06} and the \emph{portal heuristic}
\cite{goldenberg10} are two similar memory-based techniques which also attempt
to speed up optimal pathfinding on grid maps.  
Both decompose the map into a series of adjacent areas and both
pre-compute a database of exact distances between all pairs of nodes that transition 
from one area to another (called variously ``'portals'' or ``gates'').  
The main idea is to use this information to improve the accuracy of cost-to-go estimates
during search as a way of reducing the number states expanded by A*. 
Where the portal heuristic differs from other similar works
\cite{bjornsson06,sturtevant09} is in its use of the decomposed
graph to further prune the state space. A preliminary search 
identifies portals relevant to the problem instance at hand and, during 
a subsequent search in the original grid, 
any nodes from an area that contains no relevant portals are ignored.
\par
In the algorithm engineering community the problem of quickly computing optimal
shortest paths has received significant attention.  State of the art methods
such as Contraction Hierarchies \cite{geisberger08}
are based on a combination of Dijkstra's algorithm together with
memory-intensive abstractions. 
Such algorithms are very fast but they are also highly optimised for road
networks in which certain topological properties hold true: for example, the
existence of ``highway'' edges that appear on most shortest paths between
arbitrary pairs of nodes.  Though mostly orthogonal to RSR, there has been very
little work applying these ideas to searching on grid maps. One recent result
however \cite{sturtevant10} suggests they are not as effective when the
underlying graph contains a high degree of path symmetry.
