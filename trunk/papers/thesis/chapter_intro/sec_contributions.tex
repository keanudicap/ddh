\section{Contributions}
\label{cha::intro::contributions}
We address our first research challenge, related to the difficulty of
pathfinding on grid maps, by developing Rectangular Symmetry Reduction and
Jump Point Search: two very fast and in some cases very economical algorithms
that can dramatically improve the performance of pathfinding search.  Our
central insight in this case concerns the novel application of symmetry
elimination to pathfinding.

We address our second research challenge, related to the optimality of paths
on a grid, by giving a new theoretical results which show that the Any-Angle
Pathfinding problem can be solved exactly and online. We also describe in
detail the implementation and correct operation of a corresponding search
algorithm, Anya, which appears especially well suited to pathfinding in
computer games.

\subsection{Symmetry Breaking in Pathfinding Search}
In this thesis we speed up grid-optimal pathfinding in an entirely novel way:
through the identification and elimination of path symmetries.  Unlike the
majority of works from the literature, in which a path is concretely defined
in terms of a specific set of nodes and edges, we consider a path as an
ordered sequence of (often) interchangeable steps or actions.
By taking an alternative view of the problem we show that many of the paths
between two points on a grid map are simply permutations of each other; i.e.
they all involve the same set of actions and they all have identical cost.

We show that in the presence of path symmetry classical and usually very
performant algorithms such as A{*}~\citep{hart68} will waste much time looking at
permutations of all shortest paths: from the start node to each expanded node.
To avoid this behaviour we develop a range of pathfinding techniques that can
discard from consideration many equivalent paths and in the process
dramatically speed up grid-based pathfinding.  Improvements typically range from 
several factors to as much as two orders of magnitude.  In addition
to being very fast we show that symmetry breaking in pathfinding can be very
economical: our approaches require little-to-no upfront pre-processing and
little-to-no additional memory; making them especially well suited to
pathfinding in computer games.

\subsubsection{Rectangular Symmetry Reduction}
\label{cha::intro::contributions::rsr}
Rectangular Symmetry Reduction (RSR) is an offline symmetry breaking algorithm
for grid maps. It is fast, memory efficient, optimality preserving and can, in
some cases, eliminate entirely the need to search.  RSR decomposes an
arbitrary uniform-cost grid map into a set of empty rectangles and removes
from each such rectangle all interior nodes.  Macro edges are then added
between selected pairs of perimeter nodes to facilitate provably optimal
traversal through each rectangle.  We develop two variants of RSR: one for
4-connected grid maps and the other for 8-connected grid maps; both common
domains that often appear in the AI literature and in real-life computer
games.

In addition to the basic algorithm we also develop two pruning strategies
which can significantly reduce the number of nodes that need to be explored
during search.  The first enhancement is applied offline and allows us to
discard in many cases nodes from the perimeter of an empty rectangle. The
second enhancement is applied online and allows us to speed up node expansion
by selectively evaluating either all neighbours associated with a node or only
a small subset.

We evaluate RSR on a range of uniform-cost grid maps from the academic
literature and find that it can improve the running time of A{*} search by
several factors. In certain cases the speedup can be as much as one order of
magnitude. When we compare RSR to Swamps~\citep{pochter10}, a contemporary and
state-of-the-art graph pruning technique, we find that the two approaches have
complementary strengths and that they could be easily combined. We also
identify a range of benchmark problems on which RSR dominates convincingly.

\subsubsection{Jump Point Search}
\label{cha::intro::contributions::jps}
Jump Point Search (JPS) an online pruning strategy that deals with path
symmetry by selectively expanding only certain nodes on a grid map which we
call \emph{jump points}.  Moving from one jump point to the next involves
travelling in a fixed direction while repeatedly applying a set of simple
neighbour pruning rules until either an obstacle or a jump point is reached.
Because we do not expand any intermediate nodes can improve the search time
performance of standard A{*} by an order of magnitude and more.  JPS is unique
in the pathfinding literature in that it has very few disadvantages: it is
simple, yet highly effective; it preserves optimality, yet requires no extra
memory;  it is extremely fast, yet requires no pre-processing.  Further, our
method is completely orthogonal to and easily combined with competing speedup
techniques from the literature.  We are unaware of any other algorithm which
has all these features.

In addition to the basic algorithm we make three further contributions: (i) we
give a new and more efficient procedure for online symmetry breaking by
manipulating ``blocks'' of nodes at a single time rather than individual
nodes; (ii) we give new offline pre-processing technique that can identify
jump points apriori in order to further speed up pathfinding search.  (iii) we
enhance the pruning rules of JPS, allowing it to ignore many intermediate
nodes that must otherwise be expanded; On benchmark domains taken from real
computer games we show that our enhancements can improve JPS performance by
anywhere from several factors to over one order of magnitude. We also compare
our approaches with SUB~\citep{urasKH13}: a very recent state-of-the art
pathfinding algorithm. We find that the two techniques have complementary
strengths. In addition we show that there are large sets of benchmark
instances where our improved JPS variants are faster.

\subsection{Optimal Any-angle Pathfinding}
\label{cha::intro::contributions::anya}
We develop a new optimal algorithm for solving the Any-Angle Pathfinding
Problem.  Our approach involves searching over sets of states represented as
intervals. From each interval we select a representative point to derive a
corresponding $f$-value for the set.  We give theoretical results that show
our algorithm, Anya, always returns an optimal path. Moreover it does so
entirely online and without any pre-processing. To the best of our knowledge
Anya is the first known algorithm to provide such features and guarantees.
Anya answers an open question from the areas of Artificial Intelligence and
Game Development: is there an any-angle pathfinding algorithm which is online
and optimal?  The answer is yes.

%
%
%often in a purely online fashion and without the introduction of any
%additional memory overheads. We also 
%
%
%In this thesis we study a a range of pathfinding algorithms applicable
%to pathfinding in computer games. 
%
%In this thesis we will aim to compute shortest paths in a static environment
%represented as a two-dimensional grid of traversable and non-traversable cells.
%This common formulation appears often in contexts such as robotics and especially 
%computer games. Despite their apparent simplicity problems of this type can often 
%be surprisingly challenging to solve. This is because practitioners 
%(game developers in this case) must often satisfy two very different and competing 
%requirements. On the one hand characters in games should appear intelligent and 
%when they move they should find short realistic-looking paths. On the other hand, 
%paths should be computed quickly and with as few resources as possible. 
%This situation has led game developers to con
%
%In modern computer games 
%
%%for example it is not unusual to dedicate only 1-10\% of 
%%CPU resources for all AI-related activities -- including pathfinding~\cite{tozour08}.
%%Likewise, there are often very restrictive limitations placed on the amount
%%of available memory. This situation has led researchers and game developers
%%to consider 
%
%%A more general variant is known as the Euclidean
%%Shortest Path Problem. It differs from the Any-Angle case in that it does
%%not require a grid map as input: any planar map will sufficie.
%%Game developers have historically approached this challenge using a
%%post-processing technique known as ``string pulling''. Applied to a grid-optimal
%%path between the start and target location, such techniques simply ``pull''
%%the endpoints of the path until it is taut.
%%More recent approaches improve things by incorporating the post-processing step into 
%%online search. Both methods typically yield only approximately shortest paths.
%%
%
%
%
%
%We will describe novel
%algorithms that break symmetries both entirely online and also using a small pre-computation step. 
%Our approaches are not only fast but also memory efficient; in some cases requiring zero additional
%space. Moreover, these ideas are almost entirely orthogonal to existing speedup techniques and can
%therefore be combined to achieve even better performance.
%
%When compared to other works
%from the literature we find that our symmetry-breaking approaches have several compelling advantages:
%
%
%The second major challenge facing 
%
%
%Symmetry is manifest in pathfinding when there exist many variants of the same path. 
%Symmetry slows 
%
%
%
%
%
%\begin{itemize}
%\item Spatial abstraction techniques such as HPA{*} and PRA{*} create  a small approximation of the 
%map that is faster to search; for example by dividing the cells of the grid into connected neighbourhoods, 
%each of which can be treated as a single macro-location. Such techniques are usually very fast but
%they trade optimality for speed. Computing the abstraction requires a pre-computation step and introduces 
%a small memory overhead.
%\item Graph pruning techniques such as Swamps~\citep{pochter08} reduce pathfinding effort by 
%identifying areas of the map that can be ignored because crossing them would not improve the
%length of any optimal path. These approaches require a pre
%
%
%\end{itemize}
%is often
%
%including spatial abstraction, graph pruning and memory heuristics. Each
%of these approaches involves some kind of trade-off. k
%
%
%
%
%
%shortest path
%
%shortest path 
%
%they are often coarse approximations of the underlying environment. This means that
%(i) computed paths, even those which are optimal with respect to the grid, are only approximations
%
% Higher resolution grids
%can help to mitigate such problems but the size of map grows by a factor of four each time
%the dimensions of the map (i.e. the length and width) are doubled. Moreover, pathfinding on 
%grid maps is surprisingly challenging; depending on the topography of the map 
%(the total number of cells and the density and placement of obstacles) even a performant
%algorithm such as A{*}, often regarded as the gold standard in computer games, 
%can explore a large portion of all traversable cells before finding an optimal path
%between the start and tar
%
%
%\begin{itemize}
%\item Local search algorithms, some sometimes which mimic the behaviour of real-life insects. 
%These approaches are often very fast and very memory efficient but they make no guarantees that 
%they can actually find a path if one exists.
%\item Blind search algorithms. Developed for solving mazes,  these ``classical'' approaches can
%often (not always) guarantee that a path will be found and sometimes even that the path
%will be optimal -- but they are typically not very efficient.
%\item Informed search algorithms. This family of approaches will always find the optimal path if 
%one exists. The A{*} algorithm is a notable example; it comes with strong guarantees about
%efficiency and optimality and is regarded by many as the gold standard approach for pathfinding. 
%\item Near-optimal pathfinding techniques. Such approaches are usually very fast, very memory
%efficient and will find a path if one exists. Often such methods also come with guarantees
%about the quality of the paths they return.
%\end{itemize}
%
%
%Just as there are many different types of maps there are equally many different types of approaches that have 
%been developed to actually find a path between arbitary pairs of start and target locations.
%%Road maps are another specialised representation technique widely employed in logistics and personal 
%%navigation. They describe the features of physical transportation networks: 
%%usually roads but also e.g. public transportation routes or rail lines. Road maps are simple to understand
%%and information rich but they can be time consuming to accurately construct and they 
%%are not very useful for certain types of pathfinding; e.g. computer games or robotics.
%
%%Visibility graphs are special kinds of maps developed specifically for robotics applications.
%%%Their construction requires only a description of the position and size of 
%%%obstacles in an environment.
%%%The resulting map simply 
%%A visibility graph describes which pairs of points in an environment are mutually observable 
%%and can thus can be reached from one another by travelling in a straight line. Like road maps, 
%%visibility graphs are simple to understand and they can be quickly constructed. 
%%The main drawback is that they can be quite dense and thus computationally inefficient 
%%to search when trying to find a path. Another disadvantage is that visibility graphs often
%%do not record information about the underlying terrain. Such details can be important; e.g.
%%certain types of terrain are easier to traverse than others.
%
%\section{Search Techniques}
%\label{cha::intro::search}
%
%
%\begin{itemize}
%\item there are many application areas where the problem is interesting
%\item there are many variations of the problem
%\item solving the problem  involves (i) choosing a domain representation
%and (ii) choosing a search strategy. both choices are important and they
%can have a dramatic effect on how challenging (or not) pathfinding can be.
%\item many domain representations exist; discuss a few of the most popular
%including grid maps.
%\item many solution strategies exist; discuss a few possibilities and
%spell out that e.g. A* is regarded as the gold standard by industry
%practitioners: cite gamasutra and mention that many practitioners use A*
%-- if not directly then as a basic building block.
%\item
%Approaches attempting to do better than A* have typically focused on 
%exploiting the domain 
%\end{itemize}
%
%
%particular constraints on what the path 
%
%Such activities are central
%to a great many application areas. For example: in robotics 
%
%Such activities are
%central to a great many applic
%
%
%of central importance in so much of daily life the problem is studied
%
%
%
%are of central importance in a great many application areas.
%
%Since navigation
%activities are central to a great many applicaton 
%
%cartographic
%representa
%
