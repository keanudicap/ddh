\chapter{Introduction}
\label{cha:intro}

\section{The Pathfinding Problem}
Pathfinding is the problem of navigating from A to B. It is a very old
problem in Computer Science and one which commonly appears in a variety
of settings including robotics, logistics, personal navigation and 
digital entertainment.

\section{Practical Challenges and Tradeoffs}
Put here an example showing why symmetry is a problem.

\section{Contributions and Target Applications}
\subsection{Rectangular Symmetry Reduction}
\subsection{Jump Point Search}
\subsection{Any-angle Pathfinding}

\section{Publications and Thesis Overview}
\label{sec:outline}
This thesis is structured as follows: Chapter~\ref{cha:related}
surveys related work from academic and industry literature and provides 
a background for remaining chapters.
Chapter~\ref{cha:rsr}
presents Rectangular Symmetry Reduction. This work was
previously reported in~\cite{harabor10,harabor11a}.
Chapter~\ref{cha:jps} presents Jump Point Search. This work was
previously reported in~\cite{harabor11b,harabor12}.
Chapter~\ref{cha:anya} presents ANYA, our approach for solving
the any-angle pathfinding problem.
Chapter~\ref{cha:conc} presents the conclusion of the thesis
and a summary of future work.

