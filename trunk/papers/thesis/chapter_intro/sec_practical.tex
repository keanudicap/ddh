\section{Practical Considerations}
\label{cha::intro::practical}

There are two common questions which every pathfinding practitioner must address
at the outset:
%Regardless of the particular variant at hand there exist two common questions that 
%each formulation must address:
 (i) how to construct a map (or search graph) that represents the operating 
environment in which we want to navigate and (ii) how to actually search the map
for a path. We discuss each in turn.
%The first question which every pathfinding practitioner must address at the 
%outset is how to best construct a map that represents the operating
%environment in which we want to navigate.

Maps used for pathfinding are exactly analogous to their real-life cartographic 
counterparts: they document the salient features of an environment such as the
%\section{Types of Maps}
%\label{cha::intro::map}
%Before we can start navigating from one location to another -- whether in 
%in a physical or virtual environment -- we need to first construct a map to
%represent the salient features of the environment; features such as the
locations of roads, the lengths of road segments, terrain type and elevation
and the placement of both natural and artificial obstructions such as buildings 
or waterways. 
Depending on the application and setting some of these 
features may be more important or less important; still others we might choose
 to simply ignore. For example:

\begin{itemize}
\item In computer games the virtual environment is often discretised into a grid of 
cells, each of which is either traversable or non-traversable. 
%An alternative approach
%discretises the environment into a mesh of simple polygons that can be traversed.
%popular approach,  known as a navigation mesh, discretises the environment into 
%a set of simple polygons. Grids are often favoured for their simplicity while 
%meshes are favoured for their typically small size. 
%The main
%advantage of grids is simplicity; their main disadvantage is that at high resolutions
%grids become very large and ineffcient to search. Navigation meshes meanwhile
%require less space but they are complicated to construct and need to be repaired
%if the environment changes.
%In other domains specialised types of maps used for pathfinding have also arisen: 
\item In logistics and personal navigation road maps accurately describe the
features of transportation networks but details about the rest of the world
are omitted.
\item In robotics,  visibility graphs capture information about which points
in the environment can be reached from one another by travelling in a straight line.
Other information is often secondary from the perspective of a mobile robot.
\end{itemize}
%
%are discarded.
%In 
%In logistics and personal navigation road maps are ubiquitous while 
%in robotics visibility graphs are very popular. Both capture details of the physical 
%environment that are most salient for the application at hand. In the case of road maps, it is critial
%to accurately describe the features of physical transportation networks but details about the rest
%of the world can be discarded. Similarly, visibility graphs only capture information
%about which points in the environment can be reached from one another by travelling in a straight
%line; everything else is secondary from the perspective of a mobile robot.
%\end{itemize}
%Grid maps are simple to construct and very popular
%with researchers and practitiners 
%build information rich topographic models and they are popular with researchers
%and practitioners alike. 
%Their main disadvantage is that at high resolutions, grids become
%very large and they can be time-consuming to search -- a problem further compounded when
%trying to find a path in two or three dimensions.
%They are easy to understand, trivial to apply and 
%A grid map for example is a decomposition of a physical or virtual environment 
%into square cells that are marked as either traversable or non-traversable. 
%Grids are very popular in robotics and computer games because they are easy to 
%understand, trivial to construct and because they can be used to build information 
%rich topographic models. The main disadvantage is that at high-resolution grids
%become very large and can be time-consuming to search -- a problem further compounded
%when trying to find a path in three or more dimensions. 
%
%Navigation meshes are a specialised alternative to grid maps that have arisen
%from the game development community. A navigation mesh can be described as a collection
%of simple polygons that together describe the walkable surfaces of a virtual environment.
%Navigation meshes are popular in many games -- especially three-dimensional games -- because 
%they can be much smaller in size than grid maps are usually faster to search.
%On the other hand navigation meshes are complicated to construct (often requiring hand-tuning) 
%and they are non-trivial to repair of the environment changes.
%%A navigation mesh is the name given to a set of polygons that together
%describe the walkable surfaces in a virtual environment such as a computer game. This type
%of map is a popular alternative to grids because in many cases few polygons can be used 
%to effectively represent large portions of the environment. On the other hand meshes
%are complicated to construct, are non-trivial to repair if the environment changes and
% may require hand tuning.
%In other domains other specialised types of maps have also arisen: in the area of logistics
%and personal navigation road maps are ubiquitous while in robotics visibility graphs 
%are very popular. Both types of maps capture details of the environment that are
%most salient for the application at hand. In the case of road maps, it is critial
%to accurately describe the features of physical transportation networks but details about the rest
%of the environment can be discarded. Visibility graphs meanwhile only capture information
%about points in the environment that can be reached from one another by travelling in a straight
%line; everything else is secondary from the perspective of a mobile robot.
%
Many more types of maps exist but for the purposes of pathfinding there is no single ``best'' choice.
Each type of map emphasises different features of the physical or virtual environment and each 
has distinct strengths and weaknesses -- characteristics which sometimes only become apparent in 
a particular application or setting. 

Having decided on a particular type of map, pathfinding practitioners must also decide, often
at the same time, how to actually search the map for a path.
The academic literature is rich with works that describe different 
techniques for solving pathfinding problems. Examples include:
\begin{itemize}
\item Search algorithms inspired by the behaviour of real-life insects.
\item Blind-search algorithms developed to systematically solve mazes.
\item Informed-search algorithms that employ heuristic lower-bounds in order
to search the most promising areas of the map first.
\end{itemize}

\noindent Some approaches always find the shortest path while others do not. Some come with performance and and 
efficiency guarantees while others do not. Depending on the target application (and even map type)
some techniques may dominate others but there is no single ``best'' technique that is preferable in all cases.
%In this thesis we will build directly on the A{*} algorithm: a milestone technique from the area 
%of Artificial Intelligence, A{*} promises to always return the shortest path if one exists. Moreover, 
%it promises to do so using the least amount of possible effort. Our contribution is an
%enhancement of A{*}

