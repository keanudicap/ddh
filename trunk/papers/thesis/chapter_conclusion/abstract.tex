\chapter{Conclusion}
\label{cha::conclusion}
My work in this area can be broadly classified as a search space pruning
technique. Where it differs from existing efforts is that, instead of trying to
identify areas that do not have to be crossed during search, I aim to identify
and prune symmetric nodes that prevent the fast exploration of an area. This
idea nicely complements existing search-space reduction techniques and, as it
turns out, also complements most grid-based abstraction methods and memory
heuristic approaches.

Our work in this area represents a significant advancement of the state-of-the-art.
We have shown that finding distance-optimal paths on many of the most challenging 
grid-based pathfinding benchmarks from the literature, maps drawn from real computer games,
can be reduced to almost trivial problems that require very little effort to solve.

\section{Future Work}
We discuss future work in terms of our two main contributions: Symmetry breaking
in pathfinding search and Any-angle pathfinding.

\subsection{Symmetry Breaking}
One interesting direction for further work is Jump Point Search is well suited
for applications such as robotics and computer games; both of which often
involve pathfinding across uniform-cost grid maps.  We have given some simple
suggestion by which JPS can be extended to weighted grid maps: simply consider
as forced any neighbouring tile which has a different terrain type than the
current node. The effectiveness of this approach depends on the topgraphy of
the map: grid maps of the type typically found in computer games can often be
divided into different regions, each having a distinct type of terrain. In
these cases JPS breaks symmetries when pathfinding across a particular region
and stops when it encounters a node that functions as a transition point from
one region to another. A more general approach to weighted grids, and an
interesting direction for future work, is to adapt the neighbour pruning rules
of JPS to explicltly calculate the cost of local paths through and around the
current node.  Using such an approach the search is will not necessarily stop
to expand a node each time a path crosses from weighted region to another.

Another interesting direction is to extend jump points to other types of
grids, such as hexagons or texes~\citep{yap02}. We propose to achieve this by
developing  a series of pruning rules analogous to those given for square
grids.  As the branching factor on these domains is lower than square grids,
we posit that jump points could be even more effective than observed in the
current paper.  

Another interesting direction is combining jump points with
other speedup techniques: e.g. Swamps or HPA*. 

Stronger pruning operators are another interesting possibility. SUB~\cite{urasKH13}
is a recent offline pruning algorithm that can prune an intermediate node from
a path as long as the distance between the start and target location is the same
as the lower-bound distance given by the local heuristic (usually Octile Distance).
It seems reasonable to (i) combine this style of pruning with online JPS and
(ii) combine JPS' directional pruning approach to reducing the branching
factor of SUB in order to further enhance the performance of that algorithm.


\subsection{Any-angle Pathfinding}
An obvious direction for further work is a concrete implementation of Anya
together with an empirical evaluation. This is a topic of current research.

Another interesting direction is to generalise the theoretical results from
Anya to the problem of finding Euclidean-optimal paths in continuous planar
environments with polygonal obstacles -- rather than the square obstacles of
the type found in grid maps. Such a generalisation could be achieved by (i) 
using a grid to tesselate the environment and (ii) testing if any obstacles
intersect the current interval as we move it from one row of the grid to anoter.
Each time we detect an intersection we generate a successor interval whose y-axis
is the same as the obstacle rather than the y-axis of the next row.
