\chapter{Conclusion}
\label{cha::conclusion}
My work in this area can be broadly classified as a search space pruning
technique. Where it differs from existing efforts is that, instead of trying to
identify areas that do not have to be crossed during search, I aim to identify
and prune symmetric nodes that prevent the fast exploration of an area. This
idea nicely complements existing search-space reduction techniques and, as it
turns out, also complements most grid-based abstraction methods and memory
heuristic approaches.

\subsection{Comments re JPS}
One interesting direction for further work is Jump Point Search is well suited
for applications such as robotics and computer games; both of which often
involve pathfinding across uniform-cost grid maps.  We have given some simple
suggestion by which JPS can be extended to weighted grid maps: simply consider
as forced any neighbouring tile which has a different terrain type than the
current node. The effectiveness of this approach depends on the topgraphy of
the map: grid maps of the type typically found in computer games can often be
divided into different regions, each having a distinct type of terrain. In
these cases JPS breaks symmetries when pathfinding across a particular region
and stops when it encounters a node that functions as a transition point from
one region to another. A more general approach to weighted grids, and an
interesting direction for future work, is to adapt the neighbour pruning rules
of JPS to explicltly calculate the cost of local paths through and around the
current node.  Using such an approach the search is will not necessarily stop
to expand a node each time a path crosses from weighted region to another.

Another interesting direction is to extend jump points to other types of
grids, such as hexagons or texes~\cite{yap02}. We propose to achieve this by
developing  a series of pruning rules analogous to those given for square
grids.  As the branching factor on these domains is lower than square grids,
we posit that jump points could be even more effective than observed in the
current paper.  Another interesting direction is combining jump points with
other speedup techniques: e.g. Swamps or HPA*.
