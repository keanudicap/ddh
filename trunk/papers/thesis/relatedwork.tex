\chapter{Related Work}
\label{cha:related}

A large number of techniques have been proposed to speed up pathfinding. Most
can be classified as variations on three themes:
\begin{enumerate}
\item{Reducing the size of the search space through abstraction.}
\item{Improving the accurary of heuristic functions that guide search.}
\item{Dead-end detection and other state-space pruning methods. }
\end{enumerate}

\section{Abstraction}
\label{cha:related:abs}
Algorithms of this type are fast and use little memory but compute paths which are usually not
optimal and must be refined via further search. Typical examples: HPA*~\cite{botea04} and
MM-Abstraction~\cite{sturtevant07}.

\section{Improved Heuristics}
\label{cha:related:heuristics}
Algorithms of this type usually pre-compute and store distances between key pairs of locations
in the environment. Though fast and optimal, such methods can incur signficant
memory overheads which is often undesirable. Typical examples: Landmarks~\cite{goldberg05} and
True Distance Heuristics~\cite{sturtevant09}.

\section{Search Space Pruning}
\label{char:related:pruning}
Algorithms of this type usually aim to identify areas on the map that do not need to be explored in
order to reach the goal optimally. Though not as fast as abstraction or memory
heuristics, such methods usually have low memory requirements and can improve
pathfinding performance by several factors. Typical examples: Dead-end Heuristic~\cite{bjornsson06}
Swamps~\cite{pochter09} and the Portal Heuristic~\cite{goldenberg10}.

My work in this area can be broadly classified as a search space pruning
technique. Where it differs from existing efforts is that, instead of trying to
identify areas that do not have to be crossed during search, I aim to identify
and prune symmetric nodes that prevent the fast exploration of an area. This
idea nicely complements existing search-space reduction techniques and, as it
turns out, also complements most grid-based abstraction methods and memory
heuristic approaches.
