\chapter{Jump Point Search}
\label{cha:jps}
In Chapter~\ref{cha:rsr} we showed that in regular domains such as grid maps
symmetry can signficiantly impact the performance of pathfinding search.
 To address the problem we developed Rectangular Symmetry Reduction:
 an offline preprocessing strategy which decomposes the map in order to 
identify and break symmetries. 

In this chapter we develop Jump Point Search (JPS): an online pruning strategy that
deals with symmetry by selectively expanding only certain nodes on a grid map
which we call \emph{jump points}.  Moving from one jump point to the next
involves travelling in a fixed direction while repeatedly applying a set of
simple neighbour pruning rules until either an obstacle or a jump point is
reached.  Because we do not expand any intermediate nodes between jump points
our strategy can have a dramatic positive effect on search performance.
Furthermore, computed solutions are guaranteed to be optimal.  
JPS is unique in the pathfinding literature in that it has very few
disadvantages: it is simple, yet highly effective; it preserves optimality, yet
requires no extra memory;  it is extremely fast, yet requires no preprocessing.
Further, our method is completely orthogonal to and easily combined with 
competing speedup techniques from the literature.
We are unaware of any other algorithm which has all these features.
\\ \newline 
The contributions in this chapter have appeared previously in~\cite{harabor11b,harabor12}.
%As an additional contribution we also study JPS+: an enhancement of the basic 
%algorithm which identifies jump points offline and improves the efficiency of pathfinding
%search by several factors compared to the purely online variant.
