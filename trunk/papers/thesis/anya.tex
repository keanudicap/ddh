\chapter{Any-angle Pathfinding}
\label{cha:anya}
In Chapters~\ref{cha:rsr} and \ref{cha:jps} we examined approaches for
improving the efficiency of pathfinding search on grid maps -- a common
setting for computer games and a domain which commonly appears in the AI
literature. 
In addition to minimising search time, a related problem in such settings
is minimising travel distance. For example: characters in a computer game 
must appear intelligent when navigating and should therefore prefer short
realistic-looking paths. However, paths which are computed on a grid map, 
even optimal paths, necessarily restrict movement to the fixed set of 
locations defined by the grid. 
\par
In this chapter we describe Anya: a new algorithm which addresses this
problem by computing \emph{any-angle} paths that do not have such constraints.
Anya operates on an input grid map but searches using intervals rather than 
the fixed points of the grid. 
From each such interval we select a representative point for which an
$f$-cost is computed. 
We prove that our approach maintains $A*$ expansion order and, unlike other
similar approaches, always returns the shortest possible path.
In the process we resolve an open question in the pathfinding community which
has been standing since at least 2007 and which has been the subject of
studies in the game development community for much longer.

