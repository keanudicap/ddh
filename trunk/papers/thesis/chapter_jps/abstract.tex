\chapter{Jump Point Search}
\label{cha::jps}
In Chapter~\ref{cha:rsr} we showed that in regular domains such as grid maps
symmetry can signficiantly impact the performance of pathfinding search.
 To address the problem we developed Rectangular Symmetry Reduction:
 an offline preprocessing strategy which decomposes the map in order to 
identify and break symmetries.
 In this chapter we develop Jump Point Search 
(JPS): an online pruning strategy that
deals with symmetry by selectively expanding only certain nodes on a grid map
which we call \emph{jump points}.  
%\par
Moving from one jump point to the next
involves travelling in a fixed direction while repeatedly applying a set of
simple neighbour pruning rules until either an obstacle or a jump point is
reached.  Because we do not expand any intermediate nodes %between jump points
our strategy can have a dramatic positive effect on search performance.
%Furthermore, computed solutions are guaranteed to be optimal.  
\par
We make the following contributions: (i) a detailed description of the jump
points algorithm; (ii) a theoretical result which shows that searching with jump
points preserves optimality;  (iii) an extensive empirical analysis using
a range of synthetic and real-world benchmarks from the pathfinding literature.
We find that jump points can improve the search time performance of standard A* by
an order of magnitude and more.  We also report significant improvement over
Swamps~\citep{pochter10}~(a recent optimality preserving pruning technique) and
performance that is competitive with HPA*~\citep{botea04}~(a well known 
sub-optimal pathfinding algorithm).
\par
JPS is unique in the pathfinding literature in that it has very few
disadvantages: it is simple, yet highly effective; it preserves optimality, yet
requires no extra memory;  it is extremely fast, yet requires no preprocessing.
Further, our method is completely orthogonal to and easily combined with 
competing speedup techniques from the literature.
We are unaware of any other algorithm which has all these features.
\\ \newline 
The contributions in this chapter have appeared previously in~\citep{harabor11b,harabor12}.
\newpage
%As an additional contribution we also study JPS+: an enhancement of the basic 
%algorithm which identifies jump points offline and improves the efficiency of pathfinding
%search by several factors compared to the purely online variant.
%\begin{abstract}
%Pathfinding in uniform-cost grid environments is a problem commonly found in
%application areas such as robotics and video games.  The
%state-of-the-art is dominated by hierarchical pathfinding
%algorithms which are fast and have small memory overheads but usually return
%suboptimal paths.  
%In this paper we present a novel search strategy, specific to grids, which is
%fast, optimal and requires no memory overhead. Our algorithm can be described as
%a macro operator which identifies and selectively expands only certain
%nodes in a grid map which we call \emph{jump points}.  Intermediate nodes on a
%path connecting two jump points are never expanded.  We prove  
%that this approach always computes optimal solutions and then undertake
%a thorough empirical analysis, comparing our method with related works 
%from the literature. 
%We find that searching with jump points can speed
%up A* by an order of magnitude and more and report significant improvement over
%the current state of the art.  
%\end{abstract}
