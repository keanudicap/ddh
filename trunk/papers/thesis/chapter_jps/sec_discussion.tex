\section{Discussion}
\label{cha::jps::discussion}
In this chapter we introduced Jump Point Search: a new online symmetry breaking
strategy for speeding up pathfinding on grid maps.  Our algorithm identifies
and selectively expands only certain nodes from a grid map which we call
\emph{jump points}.  Moving between jump points involves only travelling in a
fixed direction, either straight or diagonal.  We prove that intermediate nodes
on a path between two jump points never need to be expanded and ``jumping'' over
them does not affect the optimality of search. 
\par
Our method is unique in the pathfinding literature in that it has very few
disadvantages: it is simple, yet highly effective; it preserves optimality, yet
requires no extra memory;  it is fast, yet requires no preprocessing.
Further, it is largely orthogonal to and easily combined with 
other speedup techniques from the literature.
We are unaware of any other algorithm which has all these features.
\par
The new algorithm is highly competitive with recent and related works from the
literature.  When compared to Swamps~\citep{pochter10}, a recent
state-of-the-art optimality preserving pruning technique, we find that jump
points are up to an order of magnitude faster.  We also show that jump point
pruning is competitive with, and in many instances clearly faster than,
HPA*~\citep{botea04}; a popular though sub-optimal pathfinding technique often
employed in performance sensitive applications such as video games.
