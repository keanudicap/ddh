\section{Time and Space Requirements}
Jump Point Search performs on-the-fly symmetry breaking by recursively travelling
along the map in a given direction and checking each node it encounters for the
presence of forced neighbours. 
Given an initial node $x$, Algorithm~\ref{alg::jps::jump} steps recursively in
a fixed direction direction $\vec{d}$ until one of three stopping conditions 
are satisfied: 
(i) the procedure encounters the goal node (line 6); 
(ii) the procedure encounters a node with a forced neighbour (line 9); 
(iii) the procedure encounters an obstacle or the edge of the map (line 3).  

In the case of a straight jump
(i.e. one where $\vec{d}$ is one of the four cardinal directions)
Algorithm~\ref{alg::jps::jump} will recurse at most until it reaches the edge of
the map. In the process it will check up to $(h-1)$ or $(w-1)$ nodes ($h$
being the height of the map and $w$ its width) 
depending on whether the direction of travel is horizontal or vertical.
In the case of a diagonal jump Algorithm~\ref{alg::jps::jump} can once again
recurse at most until it reaches the edge of the map. However it must perform two 
straight jumps before recursing diagonally. Assuming there are no obstacles that
will result in early termination, it is easy to see that the procedure can visit
every node on the map, checking each one for forced neighbours. 
Each forced neighbour check requires a constant amount of memory and takes a constant
amount of time. Moreover, since JPS does not cache the results of these operations, 
the same node can be checked multiple times during the course of solving single 
pathfinding problem. 

We will show in Section~\ref{cha::jps::results}
that although JPS performs a large number of forced neighbour checks, much more than
the number of nodes expanded, each of these operations can be performed much more 
quickly than expanding a node. The reason for this efficiency is due to the fact that
our search space (i.e. the grid at hand) is defined explicitly and can be stored
in memory. These characteristics are common to pathfinding benchmarks but uncommon 
to other areas of the Search literature e.g. AI Planning. 
