\section{Time and Space Requirements}
\label{cha::jps::complexity}
Jump Point Search performs on-the-fly symmetry breaking by recursively travelling
along the map in a given direction and checking each node it encounters for the
presence of forced neighbours. 
Given an initial node $x$, Algorithm~\ref{alg::jps::jump} steps recursively in
a fixed direction direction $\vec{d}$ until one of three stopping conditions 
are satisfied: 
(i) the procedure encounters the goal node (line 6); 
(ii) the procedure encounters a node with a forced neighbour (line 9); 
(iii) the procedure encounters an obstacle or the edge of the map (line 3).  

In the case of a straight jump
(i.e. one where $\vec{d}$ is one of the four cardinal directions)
Algorithm~\ref{alg::jps::jump} will recurse at most until it reaches the edge of
the map. In the process it will check up to $(h-1)$ or $(w-1)$ nodes ($h$
being the height of the map and $w$ its width) 
depending on whether the direction of travel is horizontal or vertical.
In the case of a diagonal jump Algorithm~\ref{alg::jps::jump} can once again
recurse at most until it reaches the edge of the map. However it must perform two 
straight jumps before recursing diagonally. Assuming there are no obstacles that
will result in early termination, it is easy to see that the procedure can visit
every node on the map, checking each one for forced neighbours. 
Each forced neighbour check requires a constant amount of memory and takes a constant
amount of time. Moreover, since JPS does not cache the results of these operations, 
the same node can be checked multiple times during the course of solving single 
pathfinding problem. 

We will see in Section~\ref{cha::jps::results} JPS is efficient despite needing to
perform many forced neighbour checks. The reason for this efficiency is due to the 
fact that our search space (i.e. the grid at hand) is defined explicitly and can be 
stored in memory. Because the grid is readily available each forced neighbour check 
is equivalent to a small handful of memory operations. 
By comparison, expanding a node is much more expensive. In addition to reading
values to and from memory we also need to invoke the heuristic function $h$ and push 
and pop values to and from the open list (using a binary heap open list operation 
takes $O(log_2{n})$ time).
We explore these issues in more depth in Chapter~\ref{cha::jps2} wherein we suggest 
a number of enhancements to JPS that make forced-neighbour checks cheaper still.
