\section{Neighbour Pruning Rules} 
\label{cha::jps::pruning}

In this section we develop two simple rules for recursively pruning the set of
nodes immediately adjacent to some node $x$ from the grid.  The objective is
to identify from each set of such neighbours, i.e. $neighbours(x)$, any nodes
$n$ that do not need to be evaluated in order to reach the goal optimally. We
achieve this by comparing the length of two paths: $\pi$, which begins with
node $p(x)$ visits $x$ and ends with $n$ and another path $\pi'$ which also
begins at node $p(x)$ and ends with $n$ but does not mention $x$.
Additionally, each node mentioned by either $\pi$ or $\pi'$ must belong to
$neighbours(x)$. There are two cases to consider, depending on
whether the transition to $x$ from its parent $p(x)$ involves a straight move or
a diagonal move. Note that if $x$ is the start node $p(x)$ is null; neither case is
applicable and nothing is pruned.

\begin{enumerate}
%\par \noindent \newline
%\subsection{Straight Moves} 
\item {
\textbf{Straight Moves: } 
When $x$ is reached from $p(x)$ by a straight move we prune any neighbouring node 
$n \in neighbours(x)$ which satisfies the following dominance constraint:
\begin{equation}
len(~\begin{pth}p(x), \ldots, n\end{pth} \setminus x~)
\leq len(~\begin{pth}p(x), x, n \end{pth}~)
\end{equation}
Figure \ref{fig::jps::pruning}(a) shows an example of this pruning rule applied. Here $p(x) = 4$ and we prune
 all neighbours except $n = 5$.
}

\item{
%\par \noindent \newline
%\subsection{Diagonal Moves}
\textbf{Diagonal Moves: }
When $x$ is reached from $p(x)$ by a diagonal move we prune any neighbouring node 
$n \in neighbours(x)$ which satisfies the following dominance constraint:
\begin{equation}
len(~\begin{pth}p(x), \ldots, n\end{pth} \setminus x~) < len(~\begin{pth}p(x), x, n\end{pth}~)
\end{equation}
This case is similar to the pruning rule for straight moves; the only difference is that 
the path which excludes $x$ must 
be strictly dominant. Figure \ref{fig::jps::pruning}(c) shows an example. 
Here $p(x) = 6$ and we prune all
neighbours except $n = 2$, $n = 3$ and $n = 5$.  
%As before when $x$ is adjacent
%to an obstacle we may be forced to evaluate additional neighbours. For example
%in Figure \ref{fig::jps::pruning} the evaluation of node $n = 1$ is forced.
}
\end{enumerate}

\par \noindent
\subsection{Forced and Natural Neighbours}
Assuming $neighbours(x)$ contains no obstacles, we will refer to the nodes
that remain after the application of straight or diagonal pruning (as
appropriate) as the \emph{natural neighbours} of $x$. These correspond to the non-gray
nodes in Figures \ref{fig::jps::pruning}(a) and \ref{fig::jps::pruning}(c).
When $neighbours(x)$ contains an obstacle, we may not be able to prune
all non-natural neighbours. If this occurs we say that the evaluation of each
such neighbour is \emph{forced}.

\begin{definition}
\label{def::jps::forced}
A node $n \in neighbours(x)$ is \emph{forced} if: \\
\indent
1. $n$ is not a natural neighbour of $x$\\
\indent
2. $ len(~\begin{pth} p(x), x, n \end{pth}~) < len(~ \begin{pth} p(x), \ldots, n \end{pth} \setminus x~)$
\end{definition}
\par \noindent
In Figure \ref{fig::jps::pruning}(b) we show an example of a straight move where 
the evaluation of $n = 3$ is forced. Figure \ref{fig::jps::pruning}(d)  
shows an similar example involving a diagonal move; here the evaluation of
$n = 1$ is forced.

\begin{figure}[tb]
       \begin{center}
		   \includegraphics[width=0.95\columnwidth, trim = 10mm 10mm 10mm 0mm]
			{chapter_jps/diagrams/pruningrules.pdf}
       \end{center}
	\vspace{-3pt}
       \caption{We show several cases where a node $x$ is reached from its
parent $p(x)$ by either a straight or diagonal move. When $x$ is expanded we can
prune from consideration all nodes marked grey.}
       \label{fig::jps::pruning}
\end{figure}


