\section{Weighted Grid Maps}
The jump point identification procedure given in
Section~\ref{cha::jps::algorithm} breaks path symmetries found in uniform-cost
grid maps. It assumes that straight steps always have a cost of $1$ and
diagonal steps always have a cost of $\sqrt{2}$.  Though such settings are
common in a variety of different application areas, including robotics and
computer games, it is sometimes desirable to represent the map as a weighted
grid. Such a change allows us model more realistic environments and represent
the fact that some areas on a map can be easier (or more difficult) to
traverse than others.  In this section we will discuss a simple modification
to  Jump Point Search which allows it to identify symmetries in such settings.

Since we are unware of any standard approach for representing weighted grids
we propose the following simple cost model:
We assume the input map is an undirected grid where each tile is either
traversable or not.  Each traversable tile is associated with a particular
terrain type and each terrain type has a corresponding numeric value 
$\ge 1$ called the terrain cost.
The cost of moving one step in the grid is equal to the uniform-cost
distance between the two tiles at hand (i.e. $1$ or $\sqrt{2}$) multiplied by
the average of the terrain costs of all tiles intersected during the
transition.  In the case of a straight step, the multiplier is the average of
the terrain costs of the source and destination tiles. In the case of diagonal
steps, the multiplier is averaged over the terrain costs of the three tiles
the agent must intersect while moving.

We now propose a very simple modification Jump Point Search; 
simply replace Definition~\ref{def::jps::forced} with the following
stronger alternative:

\begin{definition}
\label{def::jps::wforced}
A node $n \in neighbours(x)$ is \emph{forced} if: \\
\indent
1. $n$ is not a natural neighbour of $x$\\
\indent
2. $ len(~\begin{pth} p(x), x, n \end{pth}~) < len(~ \begin{pth} p(x), \ldots, n \end{pth} \setminus x~)$ \\
\indent
3. $terrain(x) \neq terrain(n)$
\end{definition}

Clauses 1 and 2 are identical to Definition~\ref{def::jps::forced}; they gurantee that JPS will continue 
to break uniform-cost symmetries when Algorithm~\ref{alg::jps::jump} is recursing in a region where all tiles
have the same terrain type. Clause 3 is newly introduced; its purpose is to terminate the recursion
whenever Algorithm~\ref{alg::jps::jump} encounters a node $n$ that has a different terrain type to the current
node $x$. This approach is equivalent to dividing the map into a set of uniform-cost 
regions and applying JPS to each region; stopping the recursion each time we find a node that is forced
according to Definition~\ref{def::jps::forced} or whenever we encounter a node that allows that search to 
transition from one weighted region to another. Since we only ever jump over nodes that have the same terrain 
type as the current node we are guaranteed to break only uniform-cost symmetries. It is thus easy to see that 
this strategy is optimality preserving. 
