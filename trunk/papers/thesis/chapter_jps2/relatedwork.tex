\section{Related Work}
\label{sec::relatedwork}
Efficiently computing optimal paths is a common topic
in the literature of AI, Game Development and 
Robotics. We discuss a few recent results.

A variety of highly successful and performant ``Highway-based'' 
techniques can be found in the literature of Algorithmics. 
Two such techniques are TRANSIT~\cite{bast06} and Contraction
Hierarchies~\cite{geisberger08}. Though different in detail
both techniques operate on similar principles: they employ
pre-processing to identify nodes that
are common to a great many shortest paths. Pre-processed data is
exploited during online search to dramatically improve pathfinding 
performance. Though very fast on road networks these algorithms have
been shown to be less performant when applied to grids, especially
those drawn from real computer games;
e.g.~\cite{sturtevant10,antsfeld12,storandt13}.

Swamps~\cite{pochter10} is an optimal pathfinding technique
that uses pre-processing to identify areas of the map that do 
not need to be searched. A node is added to a set called a swamp
if visiting that node does not yield any shorter path than could
be found otherwise. Swamps can improve the performance of classical
A{*} search by several factors. This technique is orthogonal to much
of the work we develop in this paper.

SUB~\cite{urasKH13} is another recent and very fast technique 
for computing optimal paths in grid-map domains. 
This algorithm works by pre-computing a grid analogue of a visibility graph,
called a subgoal graph, which it stores and searches instead of the original
grid.
A further improvement involves directly connecting pairs of nodes
for which the local heuristic is perfect (this operation is similar 
to graph contraction~\cite{geisberger08} in that it has the effect of 
pruning any intermediate nodes along the way).
To avoid a large growth in its branching-factor SUB prunes other 
additional edges from the graph but this latter step makes the 
algorithm sub-optimal in certain cases. We compare our work against
two variants of SUB in the empirical evaluation section of this paper.

%
%\begin{enumerate}
%\item{Swamps}
%\item{Landmarks}
%\item{SUB}
%\item{Highway methods; i.e. TRANSIT, Contraction Hierarchies etc.}
%\end{enumerate}
