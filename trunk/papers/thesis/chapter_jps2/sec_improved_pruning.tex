\input chapter_jps2/diagrams/pruning.tex

\section{Improved Pruning Rules}
\label{sec::pruning}
We have seen that JPS distinguishes between jump points that have at least 
one forced neighbour and those that have none.
The former can be regarded as
the grid equivalent of a visibility point: they are adjacent to at least one
obstacle, and they are on at least one optimal path between two neighbours
that are not mutually observable. If JPS prunes any of these nodes it is
entirely possible that it will not return an optimal path.  The second type of
jump points are not adjacent to any obstacle. They are simply intermediate
locations where the optimal path can change direction in order to reach the
first type of jump point or the goal.
\par
We argue that intermediate jump points can be safely pruned without
affecting the optimality of Jump Point Search.  
Each intermediate jump point has at most three successors: the first is a jump 
point that is reachable horizontally, the second is a jump point that is reachable 
vertically and the third is the next intermediate jump point that can be reached 
without changing direction.
When we prune an intermediate jump point we store its successors in a list and
generate them in its stead. We apply this procedure recursively to any
successors which are also intermediate jump points and terminate only when a
dead-end is reached. Figure~\ref{fig::pruning} shows an example.

To see that our strategy is optimality preserving we need only observe that
for each intermediate jump point that is pruned the $g$-value of any of its
successors remains unchanged. We simply generate these nodes earlier without
first expanding any intermediary location.  Once we have pruned such a node 
the parent of each newly orphaned successor becomes the starting location from
where we initiated the jump. To extract a concrete path we simply walk from
one jump point on the final path to the next in a diagonal-first way: i.e. we
follow the octile heuristic but take all diagonal steps as early as possible.
Such a path is guaranteed to be valid and thus obstacle-free.

Our pruning strategy is applied entirely online; it does not require any
special data structures and does not store nor compute any additional
information.  It can be combined with other suggestions discussed in the 
current work and it allows us to jump further than we otherwise might. The 
cost is up to two additional open list operations for each node that we
prune. Nevertheless, we will show empirically that pruning intermediate nodes
from the search tree improves the performance of Jump Point Search.

