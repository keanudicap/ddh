\begin{figure}[tb]
\center
  %\begin{center}
    \scalebox{0.74}{%
      \begin{tikzpicture}
        \creategridjps{10}{6}
        \drawobstaclejps{3}{5}
        \drawobstaclejps{4}{1}
        \drawobstaclejps{5}{1}
        \drawobstaclejps{7}{2}
%        \drawobstaclejps{7}{4}
%        \drawobstaclejps{6}{4}
%        \drawobstaclejps{3}{7}
%        \drawobstaclejps{2}{2}
%        \drawobstaclejps{3}{2}
        \draw[->] (0.7,0.7) -- (1.3,1.3);
        \drawgridnodejps{1}{1}{$S$}
        \drawgridnodejps{2}{2}{{\color{red} 1}}
        \drawgridnodejps{5}{2}{$2$}
        \drawgridnodejps{2}{5}{$3$}
        \drawgridnodejps{3}{3}{{\color{red} 4}}
        \drawgridnodejps{7}{3}{$5$}
        \drawgridnodejps{4}{4}{{\color{red} 6}}
        \drawgridnodejps{4}{5}{$7$}
        \draw[->] (0.7,0.7) -- (1.3,1.3);
        \draw[->] (1.5,1.8) -- (1.5,4.3);
        \draw[->] (1.8,1.5) -- (4.3,1.5);
        \draw[->] (1.8,1.8) -- (2.3,2.3);
        \draw[->] (2.8,2.5) -- (6.3,2.5);
        \draw[->] (2.5,2.8) -- (2.5,3.5);
        \draw[->] (2.8,2.8) -- (3.3,3.3);
        \draw[->] (3.5,3.8) -- (3.5,4.3);
        \draw[->] (3.8,3.5) -- (9.5,3.5);
        \draw[->] (3.8,3.8) -- (5.5,5.5);
        \draw[->] (4.5,4.8) -- (4.5,5.5);
        \draw[->] (4.8,4.5) -- (9.5,4.5);
        \draw[->] (5.8,5.5) -- (9.5,5.5);
      \end{tikzpicture}%
    }
  %\end{center}
\caption{\small
We prune all intermediate jump points (here nodes 1, 4 and 6) and
instead generate their immediate successors (nodes 2, 3, 5 and 7) as
children of the node from where initiated the jump (i.e., $S$). This 
allows us to jump from location $\langle 1, 1\rangle$ to $\langle 6, 6\rangle$ (and beyond)
in a single operation.
}
  \label{fig::pruning}
\end{figure}
