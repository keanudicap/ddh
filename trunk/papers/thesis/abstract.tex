\chapter*{Abstract}
\addcontentsline{toc}{chapter}{Abstract}
\vspace{-1em}
Pathfinding (or navigating) from A to B is a common problem in Computer
Science with broad practical applications in areas as diverse as digital
entertainment, logistics and robotics.  Pathfinding is made difficult when
many variations, or symmetries, of the same path exist.  Symmetry slows down
search by forcing otherwise performant algorithms to waste time considering
many equivalent states.  We speed things up by developing new online and
offline symmetry-breaking techniques that eliminate a large number of
symmetric states. Our contributions are optimality preserving, memory
efficient and can have a dramatic positive effect on the performance of
pathfinding search. Moreover, our work is mostly orthogonal with a wide range
of efficiency-improving techniques which have been previously described in the
literature.  \par Our first contribution, Rectangular Symmetry Reduction
(RSR), breaks symmetries during an offline preprocessing step. This approach
is optimal, requires very little overhead (usually a few seconds of
preprocessing time and a linear amount of memory) and is often better than
recent search space reduction algorithms.  Our second contribution, Jump Point
Search (JPS), improves on the performance of RSR by several factors and
represents the current state of the art for pathfinding on grid-map domains.
In its online form JPS requires zero preprocessing, zero additional memory and
always finds the shortest path. Our experiments show that JPS can consistently
improve the performance of A* search by over one order of magnitude and more.
In its offline form JPS reformulates the search space to achieve even better
performance but requires an up-front investment of time. The algorithm has
zero memory overheads when applied to graphs that stored as an adjacency list.
When applied to graphs stored as an adjacency matrix, the algorithm introduces
a linear-sized memory overhead.
\par
In addition to RSR and JPS we study the related any-angle pathfinding problem.
Commonly appearing in robotics and digital entertainment domains, this problem
involves finding a shortest path in a grid-map domain but asks that the path
is not constrained to the points of the grid.  Though a range of existing
approaches have been developed to solve this problem all require trading away
at least one of: speed, space efficiency or optimality.  We give new results
showing that this problem can be solved both optimally and online.

