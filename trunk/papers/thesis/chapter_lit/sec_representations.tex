\section{Search Space Representations}
The process of navigating from one location to another involves searching a
\emph{state space}.  Typically the search space is represented as a graph $G =
(V, E)$ where $V$ is a set of locations known as \emph{nodes} or \emph{vertices}
and $E$ is a set of \emph{edges} that represent transitions from one node or
vertex to another.  Each edge in the graph has a \emph{weight} which represents
the cost of moving from one node in the graph to an adjacent neighbouring node.
We will assume that weights represent distance but they could stand for other
types of metrics; for example travel time or fuel consumption.
Search graphs can be given explicitly, in which case all vertices and edges
are enumerated apriori, or implicitl, in which case the graph is constructed
during search by applying a set of available actions.
Implicit graphs are common in multi-dimensional pathfinding problems and in
areas such as AI Planning. In two and three dimensional pathfinding, the 
search graph is given explicitly.
Many approaches exist for constructing a search graph from an input map.

\begin{enumerate}
\item{Grid Maps}
\item{Road Maps}
\item{Visibility Graphs}
\item{Voronoi Diagrams}
\item{Navigation Meshes}
\item{Shortest Path Maps}
\end{enumerate}

