\section{Types of Search Graphs}
\label{cha::lit::graphs}

%% some context? why do we need to discretise the operating environment?

There are many ways to construct a search graph and in this section we discuss 
a broad range of popular approaches: grid maps, navigation meshes, road
maps, shortest path maps and visibility graphs.  All are examples of
\emph{explicit} search graphs.  Explicit means that all vertices and all edges
are enumerated apriori, before any pathfinding can begin.  Such graphs arise in
many settings including computer games~\citep{davis00,tozour02,champandard09},
routing~\citep{sanders05,goldberg06} and robot motion
planning~\citep{latombe91,choset05}.  Not all search graphs are explicit.  Some
are~\emph{implicit}, which means that the vertices and edges of the graph are
identified on-the-fly during search.  Implicit graphs appear in higher
dimensional pathfinding settings~\citep{lavalle98,bohlin00} and related
application areas such as AI Planning~\citep{russel03}.

When comparing different types of search graphs we will make reference to
whether or not they preserve two important properties from the operating
environment: \emph{solution existence} and \emph{solution optimality}.  A search
graph which preserves solution existence guarantees that (i) every non-obstacle
point in the environment can be mapped to a vertex in the graph and (ii) if two
non-obstacle points can be connected by a path in the operating environment then
they can also be connected by a path in the search graph (though not necessarily
the same path).  A search graph which preserves solution optimality makes a similar 
but stronger guarantee: if a path between two points exists in the graph then there
also exists in the graph a path which is cost optimal with respect to the
operating environment.

%% some discussion about optimality implying completeness?
%% note that there can be cases where the mapping from the
%% environment is not complete but for the points that are mapped
%% the graph can be optimality preserving.

As we will see not every search graph preserves existence or optimality however
each representation has its own distinct advantages and disadvantages. Choosing the
right one depends on the particular requirements of the pathfinding problem at
hand. We discuss in turn a variety of different search graphs and then in
Section~\ref{cha::lit::graphs::analysis} we compare them directly.

\subsection{Grid Maps}
\label{cha::lit::graphs::grid}
A grid map is a tessellation of geometric shapes used to discretise an n-dimensional
operating environment. The canonical case involves unit squares, called tiles, 
which are applied over a plane.\footnote{Other types of two dimensional shapes,
such as hexes and texes, are also used in this setting but less commonly; cf.~\citep{yap02}.}.
Each square tile has up to eight adjacent neighbours and is marked 
as traversable or non-traversable depending on whether or not it
intersects any obstacles.  Agents are modeled as unit-size entities that
occupy a single traversable tile and move from the centre of one such tile 
to the next.  
An alternative model involves point-size agents which travel in straight 
steps along the explicit edges of grid (and sometimes diagonally through 
tile interiors)~\citep{nash07}.
In both cases straight steps incur a cost of 1. Diagonal steps, if permitted, 
incur a cost of $\sqrt{2}$.  When diagonal moves are not permitted the grid map 
is said to be \emph{4-connected}; otherwise it is \emph{8-connected}.  

Grid maps are popular for several reasons: (i) they are simple to understand 
and simple to apply (ii) they can be represented as a matrix of bits and stored
efficiently (iii) individual nodes can be updated or queried in constant time.
One significant disadvantage of grid maps is their fixed resolution. In many cases 
grids are too coarse to accurately model the underlying environment. Increasing the
number of tiles is not always possible: there is always a corresponding increase in
memory requirements and searching in a larger grid often makes pathfinding more 
challenging.
Another disadvantage of grid maps is that they produce paths which are
constrained to the points of the grid. Such paths may be not only aesthetically
displeasing but they can also be longer than strictly necessary and may
require post-processing to ``smooth'' them.

\subsection{Navigation Meshes}
\label{cha::lit::graphs::nav}
A navigation mesh can be described as a low-fidelity model of an operating
environment.  Typically comprising a set of convex adjacent polygons,
navigation meshes are often used in computer games to represent walkable and
non-walkable surfaces in two and three dimensions~\citep{snook00,tozour02}.
There are many approaches for constructing navigation meshes.  One recent
technique triangulates the environment~\citep{demyen07,kallmann10}.  Another
decomposes the map into voxels (three-dimensonal pixels) which are then
manipulated in order to construct a mesh of walkable polygons.  Recursive
spatial indexing algorithms can also be applied to produce navigation meshes.
Quad Trees~\citep{finkel74,samet85}, for example, can be used to decompose an
input map into a set of adjacent obstacle-free rectangular regions.

Meshes are popular because they are representationally complete and often more
memory efficient than other search graphs e.g. grid maps. Another advantage is
flexibility. For example, meshes can be hand-edited by game developers needing
fine grained control over agent navigation. Meshes are also well understood by 
those in the game development industry and tools exist to aid in their creation; 
e.g. \textsc{NavPower} and \textsc{Recast Navigation}. Disadvantages include: 
\begin{itemize}
\item Meshes need to be recomputed/repaired if the environment changes.
\item Computed paths often need to be smoothed or otherwise post-processed.
\item Locating particular polygons involves search. In the case 
of Quad Trees each such operation takes logarithmic time. In other cases,
such as when a polygon ordering is not defined, linear search may be required.
\end{itemize}

\subsection{Roadmaps}
\label{cha::lit::graphs::road}
A roadmap is a set of connected points that are drawn from a given environment.
Conceptually similar to \emph{road networks}, which model automotive
transportation systems, roadmaps are used to solve high dimensional pathfinding
problems in the area of robotics.  Many variants exist. The Probabilistic
Roadmap (PRM)~\citep{kavraki94}, for example,
 is created by randomly sampling a configuration space (i.e. the operating environment) in
order to build a connected graph. Another approach, known as the Reachability
Roadmap (RRM)~\citep{geraerts05} applies a grid tessellation in the first
instance and then chooses points from the grid.  Voronoi Diagrams~\citep{aurenhammer91}
can also be regarded as a type of roadmap. In this case a network of edges is constructed
which are all equi-distant from the two closest obstacles; the nodes of the network are 
found at the intersections of these edges.

Roadmaps are advantageous because they are an effective means of discretising
large multi-dimensional search spaces into small graphs that can be searched
efficiently.\footnote{An example of a search space with more than three dimensions
is the operating environment of an industrial robot arm with $k > 3$ degrees of
freedom. Such robots can be found in e.g. automotive manufacturing.}
For example: using a roadmap any pathfinding problem can be
factored into three smaller sub-problems: find a path from each of the start and
target configuration to a point on the roadmap then search for a path on the
roadmap that connects these two points. Disadvantages include:
\begin{itemize}
\item{Some types of roadmaps are not complete. For example, when using a PRM the probability of 
a path being found, if one exists, increases with the number  of sampled points~\citep{barraquand97}. 
By comparison RRMs have stronger guarantees but their applicability
is limited to pathfinding in two and three dimensions.}
\item{Roadmaps are applicable only in static environments. The introduction of
new obstacles or changes to the placement of existing obstacles typically require 
the roadmap to be re-computed or repaired. A one-shot variation of the roadmap idea,
known as Rapidly-exploring Random Trees, has been developed to address this 
limitation~\citep{lavalle98} but it is not as reliable as static roadmaps.}
\item{Roadmaps compute only approximately shortest paths and they assume a single 
robot type (i.e. they are normally not reusable across different types of robots).}
\end{itemize}

\subsection{Shortest Path Maps}
\label{cha::lit::graphs::spm}
A Shortest Path Map (SPM)~\citep{mitchell87,mitchell97}, is a tree whose nodes
form a shortest path cover of an environment.  The nodes of the tree are regions
that are computed and organised with respect to a single source point. Each
point interior to a region can be reached from the source by following an optimal 
path in the SPM that is fixed.
% Associated with each region is a root
%which is the last common vertex on the shortest path from the source to any
%point inside the region.  To extract a shortest path, from the source to any
%target point, one simply walks from the target point, which is a leaf, to the
%top of the tree, which is the source, by moving from r

Developed by practitioners working in the area of Computational Geometry, SPMs
are highly efficient data structures: they can facilitate the extraction
of a shortest path from the source to any point on the map in time logarithmic to
the depth of the tree.
Their primary disadvantage is that they are limited to queries originating at
a single fixed source. Each time the source changes the SPM must be recomputed and
each such operation requires $O(n\log_2{n})$ space and time. 
Another disadvantage is that techniques developed for computing SPMs are complex 
and difficult to implement in practice~\citep{surazhsky05}. 

\subsection{Visibility Graphs}
\label{cha::lit::graphs::vis}
A visibility graph~\citep{lozanoperez79} is a model of an operating environment based on
line-of-sight.  In the canonical case each vertex in a planar environment
becomes a node in a graph. Edges are then added between any pair of nodes that
are visible from one another.  Often employed in robot
motion planning and computer games, visibility graphs are popular due to their 
simplicity and ability to provide strong guarantees including completeness and 
(depending on the search strategy at hand) Euclidean optimality. Disadvantages 
include:
\begin{itemize}
\item Visibility graphs can be very large; in the worst case every
vertex is visible from every other vertex and a quadratic number
of edges is required. Tangent Graphs~\citep{liu92} and Silhouette Zones~\citep{young01b}
are two variant approaches that address this shortcoming in many practical
cases but their memory requirements remain worst-case quadratic.
\item Visibility graphs are static models and must be recomputed or 
repaired if the environment changes.
\item Visibility graphs do not preserve solution optimality when applied
to environments in three-dimensional space or higher.\footnote{
In two dimensions an optimal path can only circumvent a polygonal 
obstacle by passing through one or more of its vertices.
The problem in e.g. three dimensions is that an optimal path can cross 
a polyhedral obstacle at any point along its surface -- not just at the vertices
that define its faces. The same observation 
is true for obstacles in higher-dimensional spaces.}
\end{itemize}

\subsection{Comparative Analysis}
\label{cha::lit::graphs::analysis}
Selecting a technique that models an operating environment as a search graph
is not a straightforward task. We have discussed a number of popular
approaches but none are ideal in every situation; each has their own set of
advantages and disadvantages.

Shortest-path maps, for example, preserve solution existence and solution
optimality and they can solve any shortest path query w.r.t a single fixed
source  in just logarithmic time.  Unfortunately their applicability is
limited: in many pathfinding settings, particularly computer games and
robotics, it is common for the source location to change frequently and each
time it does a new Shortest-path map must be computed requiring time
$O(n\log{n})$.  Grid maps, by comparison, are much better suited in such
contexts: though they only preserve solution existence they can be constructed
in amortized linear time and operations on individual tiles can be performed
in constant time. Unfortunately grid maps can require high resolutions to
accurately represent an environment and this slows pathfinding search.

Roadmaps and Navigation Meshes are techniques that produce sparse graphs
because they do not rely on cells of uniform size.  They can be very fast to
search but must be constructed as part of an offline preprocessing step. They
must both be repaired (or recomputed) if the environment changes and they both
have linear-time worst-case requirements for operations on individual nodes.
A similar description is true of Visibility Graphs however in this case an
additional and often prohibitive space overhead may be required.

In this thesis we will employ grid-based map representations. We do not claim
this representation to be \emph{the best} however, as we have discussed, it
has certain advantages that simplify pathfinding search. Moreover, grids are
very popular with academic and industry practitioners; often appearing in
real-world applications and often employed as benchmark problems for search
algorithms.  Despite widespread use very few works from the pathfinding
literature are explicitly concerned with grids as objects for study and
analysis. In later chapters we will see that grids present a unique set of
challenges when it comes to finding shortest paths and we will present a range
of grid-based techniques that significantly improve the state-of-the-art for
pathfinding search on grid-based domains.  We focus not only on improving the
efficiency of search but also the quality of computed solutions.
