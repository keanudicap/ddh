\section{Discrete Search Graphs}
\label{cha::lit::graphs}
Pathfinding is the problem of navigating between an arbitrary pair of points on
a given \emph{input map}.  In the most common setting input maps are
two-dimensional Euclidean (i.e. flat) or three-dimensional Geodesic (i.e. curved) 
spaces. They can take the form of a \emph{spatial network} (i.e. a set of
connected points) or they can be given as a collection of traversable and
non-traversable polytopes; the latter often being called obstacles.

In order to find an obstacle-free path from one point to another practitioners
begin by constructing from the input map a data structure $G = (V, E)$
known as a \emph{search graph}.  Here $V$ is a set of permissible locations that
an agent can occupy; these are often referred to as the nodes or \emph{vertices}
of the graph.  Meanwhile $E$ is the set of \emph{edges} that connect adjacent
vertices.  Edges can be thought of as roads or corridors that an agent can
travel along or actions that can be executed in order to transition the agent
from one location to another.
The cost associated with each such move is called the \emph{edge weight}.
Weights often represent distance travelled but they could stand for other types
of metrics as well e.g. travel time or fuel consumption.  When the cost of
moving between two vertices $a$ and $b$ can differ to the cost of moving from $b$ to
$a$ the graph is said to be \emph{directed}.  When this is not the case, the
graph is said to be \emph{undirected}.  

Search graphs can be explicit, in which case all vertices and edges are
enumerated apriori, or they can be implicit, in which case the graph is
gradually built during search. Explicit graphs arise in many settings including
computer games~\citep{davis00,tozour02,champandard09}, 
routing~\citep{sanders05,goldberg06} and robot motion
planning~\citep{latombe91,choset05}.  Implicit graphs appear in higher
dimensional pathfinding settings~\citep{lavalle98,bohlin00} and related
application areas such as AI Planning~\citep{russel03}.  

Many approaches exist for creating a search graph from a given input map; we
discuss a broad range of popular methods in the remainder of this section.

\subsection{Grid Maps}
\label{cha::lit::graphs::grid}
A grid map is a tesselation of unit squares, often called tiles, which is
applied over a planar input map. Each tile has up to eight adjacent neighbours
and is marked as traversable or non-traversable depending on whether or not it
intersects any obstacles.  Agents are often modeled as unit-size entities that
occupy a single tile and move from the centre of one traversable tile to the
next.  
An alternative approach is to model agents as point-size entities that occupy 
the intersections of the grid and which travel in straight steps along the
explicit edges of grid and diagonally through tile interiors.
In both cases straight steps incur a cost of 1. Diagonal steps, if permitted, 
incur a cost of $\sqrt{2}$.  When diagonal moves are not permitted the grid map 
is said to be \emph{4-connected}; otherwise it is \emph{8-connected}.  

Grid maps are popular for several reasons: (i) they are simple to understand 
and simple to apply (ii) they can be represented as a matrix of bits and stored
efficiently (iii) individual nodes can be updated or queried in constant time.
One significant disadvantage of grid maps is their fixed resolution. In many cases 
grids are too coarse to accurately model the underlying input map. Increasing the
number of tiles is not always possible: there is always a corresponding increase in
memory requirements and searching in a larger grid often makes pathfinding more 
challenging.
Another disadvantage of grid maps is that they produce paths which are
constrained to the points of the grid. Such paths are not only aesthetically
displeasing but they can also be longer than stricictly necessary and may
require post-processing to ``smooth'' them.

\subsection{Roadmaps}
\label{cha::lit::graphs::road}
A roadmap is a small set of connected points that are drawn from a given
input map. Roadmaps are analogus to \emph{road networks} which model automotive 
transporation systems and have many similarities to highways in particular. 
The central idea is to construct a network from a small set of traversable
points. At least one of these points should be reachable from any valid location
on the input map with little to no search.

Canonical roadmap construction methods involve randomly sampling the input map.
This approach yields the Proabilistic Roadmap (PRM)~\citep{kavraki94} which is
common in settings involving complex robot motion planning (e.g. moving a
robotic arm which has many degrees-of-freedom).  Another approach, known as the
Reachability Roadmap (RRM)~\citep{geraerts05} is specialised for two and three
dimensional motion planning problems.  Constructing an RRM involves applying a
grid tesselation over the input map in the first instance and selecting a set of
``guard points'' such that every tile of the grid is visible by at least one
guard point.

Roadmaps are advantageous because they are an effective means of discretising
large multi-dimensional search spaces into small graphs that can be searched
efficiently. For example: using a roadmap any pathfinding problem can be
factored into three smaller subproblems: find a path from each of the start and
target position to a point on the road network then search for a path on the
roadmap that connects these two points. Disadvantages include:
\begin{itemize}
\item{Limited completeness guarantees. For example, PRMs are only 
probabilistically complete~\citep{barraquand97} i.e. the probability 
of a path being found increases with the number of sampled points included in the 
roadmap.
RRMs can guarantee completeness but they are limited to problems in two and three 
dimensions.}
\item{Roadmaps are applicable only in static environments. Any introduction of
new obstacles or changes in the placement of existing obstacles require the PRM
to be re-computed. One attempt at overcoming this limitation is the
Rapidly-exploring Random Tree (RRT)~\citep{lavalle98}: a one-shot sampling-based
approach which constructs a roadmap-like data structure, from scratch, for each
problem at hand.  RRTs are applicable in dynamic environments but they are less
reliable than roadmaps and may not find a solution.}
\item{Roadmaps compute only approximately shortest paths.}
\item{Roadmap construction is multi-step process which can be more complicated than 
other discretisation techniques (e.g. navigation meshes or grid maps)}
\end{itemize}

\subsection{Navigation Meshes}
\label{cha::lit::graphs::nav}
A navigation mesh can be described as a low-fidelity wire-frame model of an 
input map: it is a decomposition of traversable and non-traversable space into 
a set of adjacent heterogeneous polygons. Commonly used in computer games, 
navigation meshes are a popular technique for representing walkable and
non-walkable surfaces for arbitrary agents in two but more often 
three-dimensional space.

Many approaches exist for constructing navigation meshes. For example~\citep{snook00}.
A different approach involves the use of constrained Delaunay triangulations~\citep{demyen07}.

Navigation meshes are popular for several reasons:
\begin{itemize}
\item{They are complete.}
\item{They are often more memory efficient than e.g. grid maps.}
\item{The same mesh can be re-used for different size agents.}
\item{Meshes can be easily visualised and hand-edited by game developers needing
fine-grained control of agent navigation.}
\item{Meshes are well understood by game developers and industry standard tools exist; e.g. NavPower and Recast.}
\end{itemize}


Disadvantages: 
\begin{itemize}
\item Need to be recomputed/repaired if the environment changes.
\item Locating particular polygons can involve linear search.
\item Paths are not optimal.
\end{itemize}



\subsection{Visibility Graphs}
\label{cha::lit::graphs::vis}

\subsection{Voronoi Diagrams}
\label{cha::lit::graphs::voronoi}

