\section{Improved Heuristics}
\label{cha::lit::heuristics}
%Heuristic functions are employed in search to evaluate states in terms
%of merit. In the case of local search, heuristics are used to choose 
%a successor state from a large set of possible successors. In the case
%of informed search, heuristics provide estimates on the remaining distance 
%to the goal. We discuss a range of representative works for computing such 
%lower-bounds. This is in keeping with the scope of this thesis which is
%concerned with optimality-preserving pathfinding techniques.

%A heuristic, in the context of optimal pathfinding search, is a function that 
%estimates the distance from any given node to the goal. 
%
Heuristic functions are employed in optimal search to evaluate states in terms of merit. In the case
of pathfinding, a heuristic is a function that estimates the distance from any given location on the
map to the goal. Heuristics are useful if they can be computed efficiently (i.e.  polytime or
better) and if they are accurate.

A popular default when pathfinding in the plane is $h_{SLD}$ or Straight Line Distance.
$h_{SLD}$ is consistent, admissible and runs in constant time. In the absence of obstacles
$h_{SLD}$ is perfect but in more complex environments it can dramatically underestimate 
optimal distances between two arbitrary points; e.g.~\citep{goldberg05}.
Many works focus on improving the accuracy of $h_{SLD}$ without negatively impacting its running 
time. This approach usually translates into faster search algorithms. However, there are limits.

A number of theoretical models~\citep{pohl77,helmert08} have shown that optimal search strategies
employing even almost perfect heuristics, i.e. those having only a small additive constant for
error, must expand, in the worst case, an exponential number of nodes before reaching the goal.
Under certain simplifying assumptions a similar result can also be derived for the average
case~\citep{pearl84}.  This simplified model often aligns well with models used in pathfinding
search: constant branching factor, uniform-cost edges and a singleton goal state.  Despite this
seemingly discouraging result researchers have demonstrated that, in many pathfinding domains of
practical interest (e.g.~\citep{sturtevant12}), better heuristic estimates can dramatically improve 
the performance of optimal search.

One idea for improving heuristic accuracy is to take into accout domain specific movement rules.
$h_{MD}$ (Manhattan Distance) and $h_D$ (Diagonal or Octile Distance) are two such heuristics; they
retain all the properties of $h_{SLD}$ but provide better lower-bound estimates when pathfinding in
square grids. Similar domain-specific heuristics have been developed for pathfinding in hexagonal
grids~\citep{yap02} and more broadly for optimally solving large combinatorial puzzles~\citep{korf96}.

A different approach improves heuristic accuracy by way of a pre-computed distance database.  
A large family of such approaches, sometimes called \emph{memory heuristics}, has been
described in the literature~\citep{goldberg05,bjornsson06,sturtevant07,felner09,goldenberg10}.
ALT~\citep{goldberg05} is a typical representative. During a preprocessing step ALT selects from the
map a set of \emph{landmark} nodes and then computes a database of optimal distances from each node
to every landmark. Given such a database it can be shown that subtracting the distances of two nodes
$a$ and $b$ from a fixed landmark $l$ is an admissible estimate for the true distance between 
$a$ and $b$. For any given pair of nodes $a$ and $b$ ALT takes as its heuristic estimate the maximum 
difference over all landmarks $l$.
To further enhance performance some memory heuristics combine distance databases and 
spatial abstraction into a holistic search framework. Examples include 
The Gateway Heuristic~\citep{bjornsson06} and The Portal Heuristic~\citep{goldenberg10}.

In a recent paper~\cite{rayner11} describes an approach for automatically constructing entire families of 
accurate memory heuristics. Their approach involves embeding a multi-dimensional graph into
a manifold. Computing the embeding requires solving a constrained optimisation 
problem that minimises heuristic error. The authors show that their approach produces very accurate
results in practice but solving the optimisation problem can require significant preprocessing 
time and space.

Compressed Path Databases (CPDs)~\citep{sanka05,botea11,botea13} are a powerful method 
for automatically constructing memory-based heuritics; this time from spatial abstractions. 
A CPD can be described as a highly compressed set of tables that store the first move on the optimal 
path between any pair of nodes in a graph. Through a simple process of recursive lookups 
CPDs can extract any optimal path in near-linear-time and without employing any state-space 
search. Though among the fastest pathfinding techniques today, CPDs nevertheless require
substantial amounts of memory and are thus not applicable in many popular pathfinding 

%Pattern Databases (PDBs)~\cite{culberson96} are a powerful method for automatically constructing
%memory-based heuristics from domain abstractions. Though often employed in solving large 
%combinatorial problems, PDBs cannot be feasibly applied in pathfinding as the goal state is 
%not fixed. Block A*~\cite{yap11} is one attempt at applying PDBs 
