\chapter{Literature Review}
\label{cha::lit}
%Pathfinding is the problem of navigating from one location to another on a 
%given two or three dimensional input map.
%The problem is among the oldest in Computer Science and appears in a range
%of common application areas including logistics, robotics and computer games.
Navigation and pathfinding are among the oldest topics in Computer Science.
They are studied in areas as diverse as Artificial Intelligence, 
Computational Geometry, Computer Graphics, Game Development and Robotics. 
Many solutions have been proposed with each such work solving the problem to 
some degree in a specific and often specialised context.
In this chapter we identify common themes from across the academic 
literature and review a range of both classical and more recent results.
We focus specifically on the single-agent pathfinding problem and its
two most common variations: finding a shortest path in a discrete search graph 
and finding a shortest path in a continuous plane.

We consider a broad range of approaches for constructing discrete search graphs
including grid maps, road maps and navigation meshes.  We then compare and
contrast a variety of algorithms that have been developed for computing shortest paths in 
discrete graphs. These include: heuristic methods, 
abstraction techniques and search space pruning strategies.  
We also examine known geometric techniques for pathfinding in continuous planar 
environments. This variation of the single-agent pathfinding problem is
particularly challenging as there are no known methods that can solve the
problem optimally and online.  We compare and contrast a range of solutions,
both exact and approximate, and discuss their various tradeoffs.


