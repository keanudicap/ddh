\chapter{Literature Review}
\label{cha::lit}
Pathfinding is the problem of navigating from one location to another on a
map. The topic is well studied in Computer Science and represents an active
area of research in sub-fields such as Artifical Intelligence, Computational
Geometry, Computer Graphics, Game Development and Robotics.  Many different
pathfinding techniques have been proposed with each such work solving the
problem to some degree in a specific and often specialised context.  In this
chapter we identify broad themes from across the academic literature and
review a range of both classical and more recent results.  We focus
specifically on two common variations of the single-agent pathfinding problem:
finding a shortest path in a discrete search graph and finding a shortest path
in a continuous plane.

We consider a wide range of approaches for constructing discrete search graphs
including grid maps, road maps and other popular and successful methods.  We
then compare and contrast a variety of algorithms that have been developed for
computing shortest paths in discrete graphs. These include: heuristic methods,
abstraction techniques and search space pruning strategies.  We also examine
known geometric techniques for pathfinding in continuous planar environments.
This variation of the single-agent pathfinding problem is particularly
challenging as there are no known methods that can solve the problem optimally
and entirely online (i.e. without any pre-processing).  We compare and
contrast a range of solutions, both exact and approximate, and discuss their
various trade-offs.


