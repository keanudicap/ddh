\chapter{Literature Review}
\label{cha::lit::abstract}
Pathfinding is the problem of navigating from one location to another on a 
given two or three dimensional input map.
The problem is among the oldest in Computer Science and appears in a range
of common application areas including logistics, robotics and computer games.
In this chapter we discuss two common formulations of the pathfinding problem: 
finding a shortest path in a graph and finding a shortest path in a plane.
We consider a range of approaches for constructing search graphs including
grid maps, road maps and navigation meshes.
We then compare and contrast a range of search-based approaches for efficiently
computing both exact and approximate solutions to the problem. These include:
heuristic methods, abstraction techniques and search space pruning algorithms.
Particular emphasis is given to optimality-preserving methods which speed up
search via symmetry and isomorphism detection.  Finally, we examine geometric
techniques for solving pathfinding problems in continuous rather than discrete
environments. 
%This variation of the problem 

