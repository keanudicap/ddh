\section{Symmetry Breaking}
\label{cha::lit::symmetry}
Symmetry is a naturally occurring phenomenon that arises whenever we are forced
to enumerate permutations of elements in a set. Typically
undesirable, symmetry forces search algorithms to waste time and prevents real 
progress toward the goal.
In the context of graph search, two kinds of symmetries can arise: \emph{state
symmetries}, which are equivalences between the individual nodes of a graph,
and \emph{path symmetries}, which are equivalences between ordered sequences of
nodes.
For an example of each consider the \textsc{Gripper} domain: a simple
but challenging series of problems from the area of AI Planning. A typical instance
involves a robot tasked with moving a series of identical balls between two adjacent rooms.
The robot has two arms (the eponymous grippers for which the problem is named) and can
move freely between the rooms but is restricted to manipulating only one
ball at a time.
To solve this problem a typical systematic search algorithm will enumerate every 
permutation of balls and grippers, despite the fact that many of these cofigurations
are equivalent and produce identical results. For instance: picking up a ball with 
the left gripper yields a state equivalent to picking up a ball with the right gripper.
 Similarly, the process of moving balls from one room to another involves a sequence of
independent and interchangeable actions; i.e. it doesn't matter in which order 
the balls are selected and moved as the result of any ordering is the same as any other.

\subsection{Approaches for Breaking State Symmetries}

A range of methods for reducing state symmetries have been described in the 
context of AI Planning, Constraint Programming and Model Checking. Though different in detail
the same high level approach is used in each case: the problem description is analysed
as part of a preprocessing step and symmetry-breaking predicates are introduced 
to avoid generating equivalent states.

In~\citep{emerson96} state symmetries in transition systems are broken by applying ideas from 
group theory: two states are identified as being symmetric if they can both be mapped to the same
symmetry group.
The authors propose a transformation of the problem graph into a quotient graph where each node is a
canonical represenative for a group of symmetric states. This graph is 
much smaller than the original, free of symmetries and much easier to search.
Unfortunately breaking symmetries in this way reduces repeatedly testing for graph isomorphism; a
problem which is $NP$-hard in general. 
A faster and more practical variation is to perform only approximate symmetry breaking~\cite{pochter11}.
To achieve this the authors propose two changes to the basic approach of~\citep{emerson96}. First, they 
only consider symmetric sub-groups where the identity of certain elements is fixed under every
permutation. Second, they compute only approximate representatives for each subgroup, rather than
exact represenatives. These changes allow the authors to break a large number of symmetries
across a wide variety of planning domains.

A less powerful but much more tractable approach to identifying state symmetries focuses on 
the identification of \emph{symmetric transitions}; i.e. actions that can be shown to have
equivalent results.
Fox and Long describe one such approach in the context of propositional
planning with \textsc{GraphPlan}~\citep{fox99,fox02}. They analyse the problem description to 
identify sets of interchangeable objects and from these sets they derive symmetric actions. 
By avoiding both they produce smaller planning graphs and increase the efficiency of seach. 
Stronger and more general variations of this idea also appear elsewhere in the literature; e.g. 
in~\citep{rintannen03} and \citep{pochter11}. 


\subsection{Approaches for Breaking Path Symmetries}
In the context of Constraint Programming symmetries arise when interchanging
the order in which values are assigned to variables or when the variables themselves
are interchangeable~\citep{walsh07}. Practitioners avoid generating equivalent states
Practitioners in the field deal with symmetries
by posting constraints that restrict the order in which assignments can occur.
In some cases these constraints are produced by a preprocessing step,
 e.g. as in~\citep{crawford96}, or computed dynamically during search,
 e.g. as in~\cite{roney-dougal04}. Each approach involves a compromise between 
effectiveness and efficiency.

\item Commutativity Pruning~\cite{haslum00} identifies sets of actions that are 
pairwise commutative. To avoid symmetries arising in cases where such actions
need to be applied sequentially, the authors propose a canonical ordering 
\item  


A different type of symmetry


\end{itemize}

Despite a wealth of literature on the topic of symmetry breaking, in areas
as diverse as AI Planning, Constaint Programming, Combinatorial Optimisation
and Model Checking, there have been no works that explicitly identify symmetry 
in pathfinding search.

\subsection{TODO}
Approaches for identifying and eliminating search-space symmetry have been
proposed in areas including planning \cite{fox99}, constraint programming
\cite{gent00}, and combinatorial optimization \cite{fukunaga08}. 
Very few works however explicitly identify and deal with symmetry in pathfinding
domains such as grid maps. 

\begin{enumerate}
\item{Symmetry Breaking in CP and CO}
\item{Isomorphism Detection}
\item{Partial Orderings}
\item{Stubborn Sets}
\item{Duplicate Pruning}
\end{enumerate}

