\section{Symmetry Breaking}
\label{cha::lit::symmetry}
Symmetry is a naturally occurring phenomenon that arises whenever we are forced
to enumerate permutations of elements in a set. Typically
undesirable, symmetry forces search algorithms to waste time and prevents real 
progress toward the goal.
In the context of graph search, two kinds of symmetries can arise: \emph{state
symmetries}, which are equivalences between the individual nodes of a graph,
and \emph{path symmetries}, which are equivalences between ordered sequences of
nodes.
For an example of each consider the \textsc{Gripper} domain: a simple
but challenging series of problems from the area of AI Planning. A typical instance
involves a robot tasked with moving a series of identical balls between two adjacent rooms.
The robot has two arms (the eponymous grippers for which the problem is named) and can
move freely between the rooms but is restricted to manipulating only one
ball at a time.
To solve this problem a typical systematic search algorithm will enumerate every 
permutation of balls and grippers, despite the fact that many of these cofigurations
are equivalent and produce identical results. For instance: picking up a ball with 
the left gripper yields a state equivalent to picking up a ball with the right gripper.
 Similarly, the process of moving balls from one room to another involves a sequence of
independent and interchangeable actions; i.e. it doesn't matter in which order 
the balls are selected and moved as the result of any ordering is the same as any other.

\subsection{Approaches for Breaking State Symmetries}
\label{cha::lit::symmetry::state}
A range of methods for reducing state symmetries have been described in the literature of 
search. Though different in detail each approach involves a preprocessing step that takes a given
problem description and converts it into an equivalent form that is symmetry free or symmetry 
reduced.

In the context of Model Checking state symmetries are handled by applying ideas from group theory:
two states are considered symmetric if they can both be mapped to the same symmetry
group~\citep{emerson96}.  The authors propose a transformation of the problem graph into a quotient
graph where each node is a canonical represenative for a group of symmetric states. Being symmetry
free, this graph is much smaller than the original and consequently much easier to search.  Similar
ideas have also appeared in the area of Constraint Programming~\citep{crawford96,roney-dougal04} and AI
Planning~\citep{pochter11}. In each case the authors report a significant improvement in search
performance for a variety of practical problems. Unfortunately this approach to symmetry breaking
reduces to repeatedly testing for graph isomorphism; a difficult problem for which no general
polytime algorithm is known.

A faster and more practical approach to handling state symmetries is \emph{partial symmetry breaking}.
This can involve considering only a subset of the permutations in each symmetry group~\citep{crawford96}
or computing only an approximately canonical representative for a given symmetry group~\citep{pochter11}.
Each of these approaches trades time spent identifying symmetry groups up-front against the risk
of considering multiple symmetric states during search.

In the area of AI Planning a number of studies are concerned with the identification of 
\emph{symmetric transitions}. These are  actions that are are applicable in the same state but which 
can be shown to have equivalent results. 
Fox and Long describe one version of this idea in the context of propositional
planning with \textsc{GraphPlan}~\citep{fox99,fox02}. They analyse a given problem description to 
identify sets of interchangeable objects and from these sets they derive symmetric actions. 
By avoiding such actions they produce smaller planning graphs and increase the efficiency of seach. 
Stronger and more general variations of this idea have since appeared elsewhere in the literature, e.g. 
in~\citep{rintannen03} and~\citep{pochter11}, but each one involves the same type of trade-offs
discussed for other partial symmetry breaking approaches.

\subsection{Approaches for Breaking Path Symmetries}
\label{cha::lit::symmetry::por}
Path Symmetries occur during search when considering sequences of independent and interchangeable
transitions. Applied separately none of these transitions lead to symmetric state however every 
possible interleaving of such a set of transitions always yields the same state. Path symmetries
are a common and recurring problem in the area of Model Checking and a number of works exist 
that deal with the problem of identifying and eliminating such interleavings. 
These methods are collectively known as \emph{partial-order reduction} techniques. 

\item Commutativity Pruning~\cite{haslum00} identifies sets of actions that are 
pairwise commutative. To avoid symmetries arising in cases where such actions
need to be applied sequentially, the authors propose a canonical ordering 
\item  


\subsection{TODO}
Approaches for identifying and eliminating search-space symmetry have been
proposed in areas including planning \cite{fox99}, constraint programming
\cite{gent00}, and combinatorial optimization \cite{fukunaga08}. 
Very few works however explicitly identify and deal with symmetry in pathfinding
domains such as grid maps. 

\begin{enumerate}
\item{Symmetry Breaking in CP and CO}
\item{Isomorphism Detection}
\item{Partial Orderings}
\item{Stubborn Sets}
\item{Duplicate Pruning}
\end{enumerate}

