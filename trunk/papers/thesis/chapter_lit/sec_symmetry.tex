\section{Symmetry Breaking}
\label{cha::lit::symmetry}
Symmetry is a naturally occurring phenomenon that arises whenever we are forced
to enumerate permutations of elements in a set. Typically
undesirable, symmetry forces search algorithms to waste time and prevents real 
progress toward the goal.
In the context of graph search, two kinds of symmetries can arise: \emph{state
symmetries}, which are equivalences between the individual nodes of a graph,
and \emph{path symmetries}, which are equivalences between ordered sequences of
nodes.
For an example of each consider the \textsc{Gripper} domain: a simple
but challenging series of problems from the area of AI Planning. A typical instance
involves a robot tasked with moving a series of identical balls between two adjacent rooms.
The robot has two arms (the eponymous grippers for which the problem is named) and can
move freely between the rooms but is restricted to manipulating only one
ball at a time.
To solve this problem a typical systematic search algorithm will enumerate every 
permutation of balls and grippers, despite the fact that such cofigurations are equivalent.
For instance: picking up a ball with the left gripper yields a state equivalent to 
picking up a ball with the right gripper.
 Similarly, the sequence of actions necessary to move a single ball from one room to another
is independent and interchangeable with the sequence of actions that moves any other ball;
thus it doesn't matter in which order the balls are selected as every interleaving of the set
of all such action sequences leads to the same state.

\subsection{Approaches for Breaking State Symmetries}
\label{cha::lit::symmetry::state}
A range of methods for reducing state symmetries have been described in the literature of 
search. Though different in detail each approach involves a preprocessing step that takes a given
problem description and converts it into an equivalent form that is symmetry free or symmetry 
reduced.

One recurring idea for breaking symmetries involves the application of concepts from
group theory: two nodes in a graph are considered symmetric if they can both be mapped
to the same symmetry group. \cite{emerson96} describe such an algorithm in the context
of Model Checking. Their method proceeds by mapping every node in a problem graph to a
a canonical representative from its associated symmetry group. The result of this process
is an equivalent \emph{quotient graph} that consists only of these canonical representatives. 
Being symmetry free this graph is much smaller than the original and consequently much 
easier to search. Other algorithms using similar principles have been also proposed; for example
in Constraint Programming~\citep{crawford96,roney-dougal04} and AI Planning~\citep{pochter11}. 
In each case the authors report a significant improvement in search performance for a variety 
of practical problems.
Unfortunately this approach to symmetry breaking reduces to repeatedly testing for graph
isomorphism; a difficult problem for which no general polytime algorithm is known.

A faster and more practical strategy for dealing with state symmetries involves considering 
only a subset of the permutations in each symmetry group. This is known as 
\emph{partial symmetry breaking}. The idea is to trade time spent identifying symmetry 
groups up-front against the risk of considering multiple symmetric states during search.
Examples of such techniques include: computing only a subset of the permutations in each 
symmetry group (e.g. as in ~\citep{crawford96}), computing only an approximately canonical 
representative for a given symmetry group (e.g. as in ~\citep{pochter11}) and looking for 
\emph{symmetric transitions} rather than symmetric states 
(e.g. as in~\citep{fox99,fox02,rintannen03}).
%This latter idea has appeared several times in the context of AI Planning; the objective 
%in this case is to identify actions that are applicable in the same state but which can 
%be shown to have equivalent results.
%Fox and Long describe one such implementation in the context of propositional
%planning with \textsc{GraphPlan}~\citep{fox99,fox02}. Other variations, which are stronger and
%more general, appear elsewhere in the literature~\citep{rintannen03,pochter11}.

\subsection{Approaches for Breaking Path Symmetries}
\label{cha::lit::symmetry::por}
Path symmetries occur during search when considering sequences of independent and interchangeable
transitions. Applied separately these transitions do not necessarily lead to a symmetric state but
given a set of such transitions every possible interleaving always yields the same state. 
Enumerating such interleavings is pointless: each one involves searching a subgraph that is 
isomorphic to all the rest and each one yields a state that is equivalent to all the rest.

%Path symmetries are a common and recurring problem in the area of Model Checking where practitioners
%must often reason about the operation of concurrent transition systems.
%In this context the state of a system (sometimes called its \emph{configuration}) is defined in terms 
%of its execution history; an ordered sequence of \emph{events} where each event
%represents a concurrent computation undertaken by one or more indepenent components of the system.
%The objective of model checking is to automatically determine whether specifications of such systems
%guarantee certain properties; e.g. that certain events will occur or that no deadlocks exist.
%Though verification is $\mathsf{NP}$-hard in general, practical algorithms are further
%hampered by the fact that many of the events which can occur at some point time are
%independent from and interchangeable with many other events. 

%i.e. as an ordered sequence of \emph{events}\footnote{Note that in the terminology of Model Checking 
%events are simply labeled instances of \emph{global transitions}~\cite{esparza08}. For the purposes of the 
%current discussion this distinction is not important and we treat the two concepts as though they are
%identical.}
%$\mathbf{h} = \langle e_1, e_2, \ldots, e_n \rangle$ where each $e_i$ represents a set of transitions;
% one from each component of the system that participates in the event.

A Marzurkiewicz Trace~\citep{mazurkiewicz86} (or simply a \emph{trace}) is a conceptual approach to
concurrent model checking that avoids isomorphic subgraphs.  Trace semantics define an equivalence
class $[\mathbf h]$ for execution histories of a concurrent system. Each history in a given
equivalence class is identical to all the orders save for the order in which interchangeable and
independent computations occur.  A number of different strategies for searching in the space of
Mazurckiewicz Traces have been proposed~\citep{esparza08}. All of them define a precedence relation
$\prec$ over histories. This is combined with a strategy such as depth-first search in order to
explore the space of all possible traces.  The idea is that only one interleaving from each
equivalence class is ever considered: the one that is $\prec$-minimal.  Provided that $\prec$ is
well-founded and preserved by extensions it can be shown that this strategy is both sound and
complete.  Moreover, if $\prec$ is a total order then the maximum length of any history is linearly
bounded.

A more general approach, which is similar to but which does not instantiate Mazurckiewicz Trace semantics, 
is Persistent Sets~\citep{godefroid96}. This is a family of related algorithms that all proceed in a 
similar manner: 
they construct in each state $s$ a set of enabled transitions $T$ such that every
non-empty sequence of transitions composed using only elements of $T$ is independent from (i.e. can be
interleaved with) every enabled transition outside $T$. During search only transitions
found in $T$ are followed; all other transitions enabled in the current state are ignored.
Like Mazurckiewicz Traces, Persistent Sets are both sound and complete. They differ
in two important ways: (i) Persistent Sets do not require the specification of any precedence relation; 
(ii) Persistent Sets do not guarantee that every state that is reached during search has a lex-minimal 
or even unique execution history. This latter property means that a search algorithm employing
Persistent Sets may explore several interleavings of the same set of transitions instead of just one.

A similar idea, which addresses some of the drawbacks associated with Persistent Sets, is 
Sleep Sets~\citep{godefroid96}. A Sleep Set is a set of transitions which are
enabled in a state $s$ but which will not be executed. Sleep Sets are passed from one state to 
another during search. The idea is to avoid executing any transitions which are independent and
interchangeable with the last transition taken in order to reach the current state.
For example: suppose a state $s'$ is reached by applying transition $t$ in a preceeding state $s$.
The sleep set of $s'$ is the sleep set of $s$ plus all enabled transitions in $s$ that 
are indepedent and interchangeable with $t$ and minus all transitions that are not.
Sleep Sets are sound and complete but less powerful than Persistent Sets. However, two approaches 
are orthogonal and can be combined. The result is a pruning strategy that is at least as effective
and often more effective than either approach applied in isolation.

Approaches for dealing with path symmetries also appear in the literature of AI Planning.  For
example, Commutativity Pruning~\citep{haslum00} deals with interleavings of independent and
interchangeable actions by only considering actions sequences that respect a given lexicographic
order.  Another approach involves the development of Persistent Set algorithms for AI
Planning~\citep{alkhazraji12,wehrle12} while a third formulates planning as Petri Net
unfolding~\citep{hickmott07}. In this latter case interleavings of independent and interchangeable
actions are handled by computing Mazurckiewicz Traces.

