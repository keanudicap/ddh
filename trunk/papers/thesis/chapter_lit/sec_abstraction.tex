\section{Abstraction}
\label{cha::lit::abstraction}
Spatial abstraction is a technique often employed in pathfinding to improve
the efficiency of search. Abstraction involves dividing an arbitrary map into
distinct areas, much in the same way cities are often divided into suburbs.
The idea is simple and natural: by representing each area as a single location 
(or as a small set of points) one can quickly identify an \emph{abstract path} 
that begins in the same area as the starting location and terminates in the
same area as the target location.  Once such a path has
been found it can be refined by solving a small set of sub-problems in the
original map; each one involving the traversal of only a single area. In most
cases this factored approach is much more efficient than searching for a path
using only the original map. Another advantage, important in many realistic 
settings, is that the agent can begin moving
as soon as the first sub-problem is refined. Finally, abstractions 
are, in general, entirely independent from any particular map representation and 
also from the the search strategy at hand.

The literature is rich with works that study the effectiveness of different
abstraction techniques.  Some of these are concerned with building memory 
efficient approximations of arbitrary maps.  Others works study specialised 
abstractions, developed for particular domains such as road networks. 
We discuss both of these approaches.

\subsection{Approximate Abstractions}
Often studied in the literature of Artificial Intelligence and Heuristic
Search, approximate abstraction methods are simple but very efficient
decompositions of arbitrary two dimensional maps.  A typical example is
HPA*~\citep{botea04}. This algorithm takes as input a given map\footnote{The
authors study grid maps but the technique is easily generalised to arbitrary
planar graphs} and returns a multi-level abstract graph.  Each level of the
abstraction is constructed by first dividing the map into square clusters, all
of equal size, and then adding to the abstract graph nodes and edges to
represent intra-cluster and inter-cluster transitions.  Other works proceed in
an analogous manner; e.g. PRA*~\citep{sturtevant05}, MMA~\citep{sturtevant07}
and HAA*~\citep{harabor08}.  These abstractions are attractive for several
reasons: they are small and efficient to search, they require little
additional memory, they preserve completeness and, because the abstract path 
can be refined piecemeal, they allow agents to start moving long before the 
entire concrete path has been identified -- an important advantage in many
real-world settings including computer games.
The price for all these benefits is that, in most cases, computed solutions
are not optimal.  Additionally the final path, once refined, may need further
post-processing to improve its length and aesthetic appeal~\citep{pinter01,botea04}.

\subsection{Road Network Hierarchies}
Another line of work, originating in the literature of Algorithmics, is
concerned with efficient and optimality-preserving abstractions; usually in
the context of pathfinding on road networks.  One popular example of such a
work is TRANSIT~\citep{bast06,bast07}: a pathfinding algorithm that combines
spatial decomposition with an all-pairs database of exact costs between a
distinguished set of nodes called \emph{access nodes}. Identified during a
preprocessing step, these nodes are interesting because they form the backbone
of a ``highway network''; i.e. they appear on a great many number of shortest paths.
The existence of such nodes is a well known and often exploited phenomenon;
see e.g. Highway Hierarchies~\citep{sanders05,sanders06}, Highway-Node
Routing~\citep{schultes07} and Contraction Hierarchies~\citep{geisberger08}.

Algorithms such as TRANSIT are attractive because they preserve solution 
optimality and because they are very
fast; query times do not typically depend the number of nodes in the graph but some 
other parameter such as the number of access nodes.
Disadvantages include an often expensive preprocessing
step to compute the set of access nodes (exponential time in the optimal case 
and poly-time otherwise~\cite{abrahamDFGW13}), the introduction of a 
significant memory overhead (e.g. TRANSIT requires an amount of space quadratic 
in the number of access nodes) and degraded performance when answering certain
types of queries (e.g. shortest path rather than shortest distance queries and
so-called ``local'' queries which are not handled by the pre-processed data and 
must be solved online using classical graph search).
Another disadvantage is that these algorithms appear to have limited applicability. 
Recent theoretical
work~\citep{abraham10,abrahamDFGW13} has shown that TRANSIT-like approaches are 
best suited to graphs having low degree and low \emph{highway dimension}.
These properties are typically true in road networks but they are not true for 
general planar graphs.  For example, when TRANSIT and other similar algorithms 
are applied to grid graphs they exhibit large increases in memory consumption 
and their performance can be strongly impacted in a negative way~\citep{antsfeld12}.
