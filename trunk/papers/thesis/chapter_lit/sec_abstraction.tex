\section{Spatial Abstraction}
\label{cha::lit::abstraction}
Spatial abstraction is a technique often employed in pathfinding to improve the efficiency of
search. It involves decomposing an arbitrary map into a small set of connected areas. The idea is
simple and natural: when a path is required between a pair of points on a map, begin by first
identifying which areas the path will need to cross. Once such an \emph{abstract path} has been
found it can be refined by solving a small set of subproblems in the original map; each involving
the traversal of only a single area. In most cases this factored approach is much more
efficient than searching for a path using only the original map.

The literature is rich with works that study the effectiveness of different abstraction techniques. 
Some of these are concerned with building small approximations of arbitrary maps.  Others works 
are specialised for particular domains, such as road networks. We discuss both.

\subsection{Approximate Abstractions}
Often studied in the literature of Artificial Intelligence and Heuristic Search, approximate
abstraction methods are simple but very efficient decompositions of arbitrary two dimensional maps.
A typical example is HPA*~\citep{botea04}. This algorithm divides a given map into a set of 
square cells, all of equal size. During a preprocessing step transitions between cells are identified
and an abstract graph is constructed with edges added to represent intra-cell and inter-cell traversals.
Other works proceed in an analogous manner; e.g. PRA*~\citep{sturtevant05}, MMA~\citep{sturtevant07}
and HAA*~\citep{harabor08}.  These abstractions are attractive for several reasons: they are small and 
efficient to search, they require little additional memory and they preserve solution existence. 
The price for all these benefits is that, in most cases, computed
solutions are not optimal. Additionally, the final path, once refined, may need further
post-processing to improve its length and aesthetic appeal~\citep{pinter01,botea04}.

\subsection{Road Network Hierarchies}
Another line of work, originating in the literature of Algorithmics, is concerned with efficient 
and optimality-preserving abstractions; usually in the context of pathfinding on road networks.
One popular example of such a work is TRANSIT~\citep{bast06,bast07}: a pathfinding algorithm that combines
spatial decomposition with an all-pairs database of exact costs between a distinguished set of nodes
called \emph{transit nodes}. Identified during a preprocessing step, these nodes are interesting 
because they form the backbone of a highway network and thus appear on a great many number of 
shortest paths. The existence of such nodes is a well known and often exploited
phenomenon; c.f. Highway Hierarchies~\citep{sanders05,sanders06}, Highway-Node Routing~\citep{schultes07}
and Contraction Hiearchies~\citep{geisberger08}. 

Algorithms such as TRANSIT are attractive because they are typically very fast, especially when
employed as distance oracles, and because they preserve solution optimality. Disadvantages often
include an expensive preprocessing step, the introduction of a significant memory overhead
(typically many times but possibly many orders larger than the space needed to store the original
map) and degraded performance when answering shortest path rather than shortest distance queries.
Another disadvatage is that these algorithms appear to have limited applicability. Recent
theoretical work~\citep{abraham10} has shown that TRANSIT-like approaches are best suited to graphs
having very lowe degree -- as is typically the case in road networks.  When this is not the case
TRANSIT and other similar algorithms exhibit large increases in memory consumption and their
performance can be strongly impacted in a negative way~\citep{antsfeld12}.
