\section{Euclidean Shortest Paths}
\label{cha::lit::euclidean}
%In Chaper~\ref{cha::anya} we give a first optimal and online algorithm for the any-angle pathfinding
%problem. In this section we review a number of related works. 
Many of the optimal pathfinding algorithms we have discussed to now are optimal only in 
a graph-theoretic sense. In this section we review the state of the art for computing
Euclidean shortest paths: i.e. shortest paths with respect to the underlying operating 
environment of an agent and not a graph-based discretisation. We discuss two popular
formulations for this problem: the Any-Angle Pathfinding Problem and the Euclidean Shortest
Path Problem. 

\subsection{Any-Angle Pathfinding Algorithms}
\label{cha::lit::euclidean:anyangle}
The Any-Angle Pathfinding Problem appears in robotics and computer video games. It involves finding
a shortest path between an arbitrary pair of points on a two-dimensional grid map but asks that
movement along the path is not artificially constrained to the points of the grid.

Within the game development community a simple and popular solution to the Any-Angle Pathfinding
Problem is \emph{string pulling}~\cite{pinter01,botea04}.  The idea is to compute a grid-optimal
path in the first instance and smooth the result as part of a post-processing step that improves
both its length and aesthetic appeal. String pulling has two disadvantages: (i) it requires an
additional computation beyond just finding a path (ii) it can only yield approximately shortest
paths.

A number of algorithms improve on string-pulling by integrating post-processing into node expansion
during search. Field D*~\citep{ferguson05} is one such algorithm. It uses a simple approach based on
linear interpolation that smooths paths one grid cell at a time.  Theta*~\citep{nash07} and its
variants \citep{nash09,nash10,munoz12} are similar ideas that can smooth much longer path segments in a
single operation. This algorithm relies on line-of-sight checks from each successor of a node to its
grandparent. A successful check introduces a new shortcut that bypasses the current node.  Another
approach, Block A*~\cite{yap11}, avoids line-of-sight checks entirely by precomputing a database of
exact costs between pairs of points in a localised area.  Each of Field D*, Theta* and Block A*
improve on string pulling in terms of solution quality and, in many cases, also in terms of running
time.  The main disadvantage of all these algorithms is that they provide no optimality guarantees.

Accelerated A*~\citep{sislak09a,sislak09b} is a recent technique for the Any-Angle Pathfinding
Problem which is conjectured to be optimal. Similar in spirit to Theta* this algorithm differs in
two ways: (i) it performs line-of-sight checks to a set of previously expanded nodes (not just a
single ancestor); (ii) it implements a block-based expansion strategy that allows it to consider
sets of nodes at a time.  
A primary advantage of Accelerated A* is that it uses less memory than competing algorithms (e.g.
Theta*).  Another advantage is that computed solutions are shorter than those produced by competing
algorithms. In a range of grid-based domains the authors show that Accelerated A* always finds the
optimal path.  Unfortunately no theoretical guarantees are provided.  The main disadvantage of
Accelerated A* is its running time. During each node expansion there can be a large number of
line-of-sight checks between the current node and a set of potential alternative parent nodes.  For
large problems with long optimal solutions this set can comprise most nodes appearing on the closed
list.

%. Similar in spirit to Theta*, this algorithm it differs
%in two ways: (i) it checks visibility to a number of alternative parents, not just one (ii) it 
%applies a dynamic expansion strategy to quickly traverse over empty regions. 
%It is conjectured~\cite{sislak09b}, on the basis of limited empirical results, that Accelerated 
%A* always finds an optimal path. However, no proof is given.

\subsection{Euclidean Shortest Path Algorithms}
\label{cha::lit::euclidean::euclidean}
The Euclidean Shortest Path Problem is a well known and well researched topic in the areas of
Computational Geometry and Computer Graphics. It can be seen as a generalisation of the Any-Angle 
Pathfinding Problem. It asks for a shortest path in a plane but does not impose any restrictions on
obstacle shape or obstacle placement (cf. grid aligned polygons made up of unit squares). 

Visibility Graphs~\cite{lozanoperez79} (discussed in Section~\ref{cha::lit::graphs})
are a a well-known and and popular technique for solving the Euclidean Shortest Path 
Problem. Searching in such graphs requires $O(n^2log_{2}{n})$ time.  There are two 
main disadvantages: (i) computing the graph requires an offline preprocessing step;
(ii) the graph is static and must be repaired if the environment changes;
(ii) storing the graph introduces an $O(n^2)$ memory overhead (each node in 
the graph can be adjacent to every other node). Tangent Graphs~\cite{liu92} are a 
particularly efficient variant but their space requirements remain worst-case 
quadratic. 

Other exact approaches are based on the Continuous Dijkstra~\cite{mitchell87,mitchell97} paradigm.
The most efficient of these algorithms~\cite{hershberger99} involves a
precomputation requiring $O(n \log_{2}{n})$ space and $O(n\log_{2}n)$ time. The result 
is a Shortest Path Map (Section~\ref{cha::lit::graphs}); a planar subdivision of the environment 
that can be used to find a Euclidean shortest path in just $O(\log_{2}n)$; but only for queries 
originating at a fixed source. Like Visibility Graphs, this approach also introduces additional
memory overheads (storing the subdivision) and the preprocessing step must be re-executed each 
time the environment or the start location changes.
