\section{Search Strategies}
\label{cha::lit::search}
Given a search graph $G$ and a pair of vertices $s$ and $t$, which correspond to the 
start and target positions of a single agent, our objective is to find a path in $G$
from $s$ to $t$.
A large number of strategies have been developed for this purpose. Most fall into
one of three distinct paradigms:
\begin{itemize}
\item{\emph{Blind-search algorithms} which explore the solution space in a systematic fashion.}
\item{\emph{Informed-search algorithms} which employ lower-bounds to improve the 
efficiency of systematic search.}
\item{\emph{Local-search algorithms} which employ heuristic decision-making or mimicry of
natural processes in order to explore only select portions of the solution space.}
\end{itemize}

In the next sections we will discuss each paradigm in turn. In doing so we will sometimes
find it useful to refer to two important properties of search algorithms: \emph{completeness} 
and \emph{optimality}.
We say that a search algorithm is \emph{complete} when it guarantees that it will return a
solution if there exists a path between nodes $s$ and $t$ in $G$. 
Optimality is a similar but stronger property which entails completeness. 
We will say that a search algorithm is \emph{optimal} if it guarantees that every solution it 
returns is also a shortest path from $s$ to $t$ in $G$. 

Both properties are defined only with respect to the current search graph $G$.
This has some important ramifications. For example: a path returned by an optimal search algorithm
is not guaranteed to be \emph{truly} optimal (that is, optimal with respect to the operating environment 
of the agent) unless $G$ is constructed in a way that preserves solution optimality.
(cf. Section~\ref{cha::lit::graphs}).
Similarly, in order for a complete algorithm to accurately report whether or not any path exists between 
$s$ and $t$ it will require a graph $G$ that preserves solution existence.

\subsection{The Search Process}
\label{cha::lit::search:terms}
Looking for a path between two locations $s$ and $t$ is a process of repeatedly
\emph{expanding} and \emph{generating} nodes from a search graph $G$.  
Expansion is equivalent to
performing an action that moves the agent from its previous location to the 
present location (the one associated with the node at hand).
%As part of this procedure the neighbours of the current node,
%which have become immediately reachable for the agent, are \emph{generated}.
Generating a node simply means to evaluate it (in terms of cost) and possibly
pushing it onto an expansion queue.

\subsection{Blind Search}
\label{cha::lit::search::blind}

\subsection{Informed Search}
\label{cha::lit::search::informed}

\subsection{Local Search}
\label{cha::lit::search::local}

