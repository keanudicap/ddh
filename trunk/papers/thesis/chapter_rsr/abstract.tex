\chapter{Rectangular Symmetry Reduction}
\label{cha::rsr} 
In this chapter we present Rectangular Symmetry Reduction (RSR): a graph pruning
algorithm for undirected uniform-cost grid maps which is fast, memory efficient,
optimality preserving and which can, in some cases, eliminate entirely the need
to search.  The central idea that we will explore involves the identification
and elimination of \emph{path symmetries} from the search space. 
Symmetry appears in many domains; for example in planning~\citep{fox99}, constraint 
programming \citep{walsh07} and combinatorial optimisation~\citep{fukunaga08}. 
Unless it is handled properly, symmetry almost always 
increases the size of the search space and forces search algorithms to waste time
considering variations of already known solutions.
\par
To deal with path symmetries RSR decomposes an arbitrary uniform-cost grid map
into a set of empty rectangles, removes from each such rectangle all interior
nodes and retains only nodes from along the perieter.  A series of macro edges
are then added between selected pairs of perimeter nodes to facilitate
provably optimal traversal through each rectangle.  We first develop RSR in
the context of 4-connected grid maps; a common setting in video games and one
which appears often in the AI literature.  We then generalise RSR to
8-connected grid maps where the increase in branching factor makes effective
symmetry elimination more challenging.  We also develop two new pruning
strategies which can signficantly reduce the number of nodes that need to be
explored during search.  The first enhancement is applied offline and allows
us to discard in many cases nodes from the perimeter of an empty rectangle.
The second enhancement is applied online and allows us to speed up node
expansion by selectively evaluating either all neighbours associated with a
node or only a small subset.

We evaluate RSR on a range of uniform-cost grid maps from the academic literature and find
that it can improve the running time of A* search by several factors. In certain cases the
speedup can be as much as one order of magnitude. We then compare RSR to 
Swamps~\citep{pochter10}, a recent state-of-the-art
pruning technique. We find that the two algorithms have complementary strengths and that they
could be easily combined. We also identify a range of benchmark problems on which RSR 
dominates convincingly.  
\par

The contributions described in this chapter have been presented previously in
~\citep{harabor10,harabor11a,harabor11c}.

