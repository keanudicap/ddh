\section{Discussion}
\label{cha::rsr::discussion}
We study the problem of pathfinding in 4 and 8-connected grid maps;
two domains which appear regularly in modern video games and academic literature.
We show that in such settings optimal search can be very difficult because between
any pair of nodes there exist many alternative paths which are symmetric: that is
the paths are identical save for the order in which individual moves occur.
We presented RSR: a novel offline method for breaking such symmetries which is simple
to understand and requires no significant extra memory. 
Our method involves decomposing a map into empty rectangular rooms, pruning all nodes
appearing in the interior and replacing them with a set of \emph{macro edges}
that facilitate optimal traversal from the perimeter of any room to the perimeter
of any other.
We also give an online node insertion technique that extends these guarantees
 to arbitrary pairs of locations appearing in the original unmodified map.
\par
We evaluate the performance of our algorithm by running A* on a wide
range of realistic game maps including one well known set from the game
\emph{Baldur's Gate II}. 
In many cases we are able to prune over 50\% of all nodes on a given map
and improve the average search time performance of A* by anywhere from 
several factors to over one order of magnitude. 
We compare the performance of RSR to Swamps-based
pruning~\citep{pochter10} and show that the two algorithms have complementary
strengths.  We find that Swamps are more useful on maps with small open areas
while RSR becomes more effective as larger open areas are available on a map. We
also identify a broad range of instances where RSR dominates convincingly and is
clearly the better choice.  Finally, we compare RSR to the enhanced Portal
Heuristic~\citep{goldenberg10}.  We show that our method exhibits similar or
improved performance but requires up to 7 times less memory.  As with Swamps, we
find that the two ideas are complementary and could be easily combined.
\par
The effectiveness of RSR is strongly dependent on the topography of individual maps: 
in the presence of large rooms or wide open areas (both commonly seen in video games\footnote{For 
example, Blizzard's popular multi-player game \emph{World of Warcraft}})
we can often compute optimal paths much faster than searching on the original map. 
On less favourable map topographies we achieve more modest improvements.
However, since our method is orthogonal to existing search techniques, it could be integrated
as part of a larger framework involving specialised heuristics or other speedup techniques; 
for example as described in \citep{botea04,bjornsson05,bjornsson06}. 
