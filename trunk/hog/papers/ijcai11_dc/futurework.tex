\section{Future Work}
\label{sec:futurework}
There are several directions available for future work; I plan to pursue each
of these during the remainder of my doctoral candidature:
\begin{itemize}
\item{\textbf{Alternative decomposition methods:} In the presence of large open 
areas, RSR can often compute optimal paths much faster than searching on the original
map. On less favourable map topographies we achieve more modest improvements.
I would therefore like to explore alternative decomposition techniques, based
for example on convex shapes, which would allow bigger empty regions to be
identified and lead to better performance.}
\item{\textbf{Stronger online pruning techniques:} RSR can produce map decompositions 
in which individual nodes have a high branching factor. This is undesirable as 
considering a large number of neighbours
usually slows down search. I intend to address  
this problem through the development of stronger pruning techniques 
that can be applied during the online phase of a pathfinding search.}
\item{\textbf{Synthesis with existing pruning methods:} RSR is orthogonal to almost
all existing techniques for speeding up pathfinding. It could therefore be
integrated as part of a larger framework involving specialised heuristics or
other graph pruning and state space reduction techniques; for example as
described in \cite{botea04,bjornsson06,pochter10}.}
\item{\textbf{Generalisations to other domains:} I am currently investigating
whether breaking symmetric paths can be applied to more general domains. For
example: weighted grids or road networks. The former often appear in application
areas such as robotics and video games while the latter are particularly
important in GPS navigation.
}
\end{itemize}
