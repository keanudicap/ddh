\section{Symmetries in 4-connected Grid Maps}
\label{sec:currentwork}
I have previously looked at the problem of symmetry breaking in 4-connected
grid maps, which allow straight but not diagonal movement.  Although less popular
than 8-connected grid maps, this domain appears regularly in the literature
\cite{yap02} and is often found in the pathfinding systems of
modern video games; e.g Square Enix's \emph{Children of Mana} (Gameboy Advance)
and Astraware's \emph{My Little Tank} (iPhone).
\par 
In \cite{harabor10} my co-author and I propose the following offline strategy 
for identifying and eliminating symmetric paths in 4-connected grid maps:
\begin{enumerate}
\item{Decompose the grid map into a set of empty rooms, where each empty room is 
rectangular in shape and free of any obstacles. 
The size of the rooms can vary across a map, depending
on the placement of the obstacles.}
\item{Prune all nodes from the interior but not the perimeter of each empty
room.}
\item{Add a series of \emph{macro edges} that connect each node on the perimeter of an empty room
with a node on the directly opposite side of the room 
\footnote{Alternatively, macro edges could be generated on-the-fly during search. 
This obviates the need to store them explicitly.}.
The cost of each edge is equal to the Manhattan distance between its two endpoints.
}
\end{enumerate}

To handle cases where the start or goal location is a node previously pruned, we
use an insertion procedure that re-adds single nodes back into the graph for the
duration of the search.
Each insertion operation can be performed in constant time.
Furthermore, if the start and goal are located in the same empty rectangle
an optimal path is immediately available -- obviating the need for search.
\par
We prove the optimality of this approach and evaluate it empirically by 
comparing the performance of A* when searching on pruned vs. un-pruned grids.
We run experiments on a wide range of realistic game maps including one well
known set from the game \emph{Baldur's Gate II}.  Results show that in many
cases over 50\% of all nodes on a given map can be pruned, resulting in
improvements to average search times by a factor of up to 3.5.

\section{Symmetries in 8-connected Grid Maps}
I am currently studying symmetry breaking methods for the much more 
common 8-connected grid map domain.
To that end, my co-authors and I propose Rectangular Symmetry Reduction (RSR): 
a new algorithm which extends the empty rooms decomposition described in
\cite{harabor10} in several directions: (i) we generalise the method from
4-connected grid maps to the 8-connected case where the branching factor makes
effective symmetry elimination more challenging; (ii) we develop a stronger
offline pruning technique to further reduce the number of nodes which need to 
be explored during search; (iii) we give a novel online pruning strategy which
speeds up node expansion by selectively evaluating either all neighbours
associated with a particular node or only a small subset.  We then prove that 
each proposed extension preserves both optimality and completeness during
search.
\par
We evaluate RSR on a range of synthetic and realistic benchmarks from the
literature and find that it can improve the average performance of A* by up to
38 times.
Compared to Harabor and Botea's method \shortcite{harabor10}, 
we both extend the applicability and improve the speed
on the subset of instances where both methods are applicable.
When compared to the Swamp-based method of 
\citeauthor{pochter10}~\shortcite{pochter10}, we find that the two algorithms
have complementary strengths and identify classes of instances where
either RSR or Swamps is more suitable.
We find that Swamps are better on maps with
small open areas and their effectiveness reduces on maps with larger open areas.
The opposite is true for RSR: larger open areas allow for larger empty rectangles,
leading to a corresponding improvement in performance.
Our results to date are encouraging: we identify a wide range of instances where
RSR is clearly the better choice, dominating convincingly across a range of
benchmarks. 
