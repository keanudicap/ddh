\section{Introduction}
\label{sec:introduction}
Symmetry is a property common in many problem domains studied by the Artificial
Intelligence community e.g. pathfinding, bin packing etc. 
Unless it is handled properly, symmetry can often force search algorithms
to evaluate many equivalent states and hampers real progress towards the goal.
The problem of how best to deal with symmetry has received significant attention 
in areas such as planning \cite{fox99} and constraint programming \cite{gent00} 
but there are very few works that explicitly identify and deal with symmetry in 
pathfinding domains such as grid maps. 
\par
To address this shortcoming I propose two novel techniques for dealing with 
symmetry on grid maps: Rectangular Symmetry Reduction (RSR) and Jump Point
Search (JPS).
RSR proceeds by decomposing an arbitrary grid map into a set of empty
rectangles. RSR then prunes all interior nodes from each rectangle, and possibly
some from the perimeter, replacing them with a set of \emph{macro edges} that
facilitate optimal travel through each rectangle.  
Meanwhile, JPS can be described as an optimality-preserving online graph
pruning method which speeds up pathfinding by selectively expanding only certain
nodes on a grid map called \emph{jump points}. 
\par
Both RSR and JPS are quite different from, and indeed orthogonal to, other graph
pruning techniques which have appeared in the literature.  For example, both the
\emph{dead-end heuristic} of \citeauthor{bjornsson06} ~\shortcite{bjornsson06}
and the \emph{Swamps-based pathfinding} approach of
\citeauthor{pochter10}~\shortcite{pochter10} focus on identifying areas on
a grid map that can be ignored for the search at hand.  
By comparison, RSR and JPS
aim to rapidly explore any given area which the search enters.  Results to date
indicate that both RSR and JPS can speed up A* search by over an order of
magnitude.  Further, when compared to the state-of-the-art method in
\cite{pochter10} we show that there are a wide range of instances where both RSR
and JPS dominate convincingly.
