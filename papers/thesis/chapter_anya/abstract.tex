\chapter{Any-angle Pathfinding}
\label{cha::anya}
In Chapter~\ref{cha::rsr}, Chapter \ref{cha::jps} and Chapter~\ref{cha::jps2}
we examined approaches for
improving the efficiency of pathfinding search on grid maps -- a common
setting for computer games and a domain which commonly appears in the AI
literature. 
In addition to minimising search time, a related problem in such settings
is computing paths that minimise travel distance and are aesthetically
pleasing. For example: characters in a computer game must appear intelligent
when navigating and should therefore prefer short realistic-looking paths.
However, paths which are computed on a grid map, even optimal paths,
necessarily restrict movement to the fixed set of locations defined by the
grid.
\par
In this chapter we describe Anya: a new algorithm which addresses this
problem by computing \emph{any-angle} paths that do not have such constraints.
Anya operates on an input grid map but searches using intervals rather than 
the fixed points of the grid. 
From each such interval we select a representative point for which an
$f$-cost is computed. 
We prove that our approach maintains A{*} expansion order and, unlike other
similar approaches, always returns the shortest possible path.
In the process we resolve an open question in the pathfinding community which
has been standing since at least 2007 and which has been the subject of
studies in the game development community for much longer.

The contributions in this chapter have appeared previously in~\citep{haraborG13}.

%\begin{abstract}
%Any-angle pathfinding is a common problem from robotics and computer games: it
%requires finding a Euclidean shortest path between a pair of points in a grid 
%map.  Prior research has focused on approximate online solutions.  A number of
%exact methods exist but they all require supra-linear space and preprocessing
%time.
%In this paper we describe Anya: a new optimal any-angle pathfinding algorithm 
%which searches over sets of states represented as intervals. Each interval is
%identified online. From each we select a representative point to derive a 
%corresponding $f$-value for the set.
%Anya always returns an optimal path. Moreover it does so entirely 
%online, without any preprocessing or memory overheads. This result answers
%an open question from the areas of Artificial Intelligence and Game Development: 
%is there an any-angle pathfinding algorithm which is online and optimal?
%%and requires only linear space?
%The answer is yes.
%\end{abstract}
\newpage
