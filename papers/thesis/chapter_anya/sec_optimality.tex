\section{Correctness and Optimality}

To prove correctness and optimality, 
we show that the optimal path 
appears in the search space 
and that the $f$ function defined in the previous section 
guarantees that the search node 
corresponding to the optimal path 
will be expanded before any node 
associated with a sub-optimal path to the target.  

%First observe that the optimal path is trivially taut.  
\begin{theorem}
  For any point $p$ that appears on a row, 
  there exists a node in the search tree 
  that corresponds to the optimal path from $s$ to $p$ 
  if such a path exists.  
\end{theorem}

\begin{proof}
  Sketch, by induction.  
  Consider an optimal path 
  $\pi_k = \langle p_1,\dots,p_k\rangle$ 
  where $s = p_1$ and $p_{k-1}$ is either a point 
  one row apart from $p_k$ (similar to a $y_i$ point
  mentionned in Figure~\ref{fig::ex1}) or a corner point on the same row.  
  By induction, there is a node $(I,r)$ in the search tree 
  that represents the optimal path to $p_{k-1} \in I$.  
  Now following Definition~\ref{defi::successors}, 
  there is a node $(I',r')$ 
  that is a successor of $(I,r)$ 
  such that $p_k \in I'$ while $r' = r$ if $p_k$ is visible from $r$ 
  and $r' = p_{k-1}$ otherwise, and the node represents the optimal path.  
\end{proof}

We now assume that the search space is explored 
%using A*, i.e., 
by expanding the queued node with smallest $f$ value 
as defined in Equation~(\ref{eq::f}).  

\begin{theorem}
%  The $f$ function of Equation~(\ref{eq::f}) 
%  is such that 
  The first expanded node 
  that contains the target $t$ 
  corresponds to the optimal path to $t$.  
\end{theorem}

\begin{proof}
  Sketch.  
  First we notice that the $f$ value of a node 
  is indeed the minimal value of all the nodes in the interval, 
  which means that $f$ is an under estimate of the actual cost 
  to the target.  
  Second we notice that, given a search node $(I,r)$ 
  and its successor $(I',r')$, 
  for each point $p' \in I'$, 
  the $f$ value of $p'$ is bigger 
  than the $f$ value of some point $p \in I$ 
  (%trivially 
  $p = r'$ if $r' \neq r$; 
  $p$ is the intersection of $I$ and $(r,p')$ otherwise); 
  the $f$ function is therefore monotonically increasing.  
  Finally, the $f$ function of a search node $(I,r)$ 
  is the length of the path if $t \in I$.  
  Hence the $f$ function of the nodes representing 
  a sub-optimal path to $t$ 
  will eventually exceed the optimal path distance, 
  while the $f$ function of the nodes representing the optimal path 
  will always remain under this value.  
\end{proof}

% EOF
