\section{Introduction}
Any-angle pathfinding is a navigation problem which appears in robotics
and computer video games. It involves finding a shortest path between an 
arbitrary pair of points on a two-dimensional grid map but asks that 
movement along the path is not artifically constrained to the points of 
the grid.  Within the game development community a simple and popular 
solution exists known as \emph{string pulling}~\citep{pinter01,botea04}.
The idea is to compute a grid-optimal path in the first
instance and smooth the result as part of a post-processing step that improves
both its length and aesthetic appeal. String pulling has two disadvantages: 
(i) it requires an additional computation beyond just finding a path (ii)
it can only yield approximately shortest paths.
\\
Theta*~\citep{nash07} and its variant~\citep{%nash09,
nash10} improve things 
by integrating post-processing into node expansion during search. Their 
idea involves updating a node's parent label following a successful line-of-sight
check to a previously expanded ancestor node.
Another approach, Block A*~\citep{yap11}, avoids line-of-sight checks entirely 
by precomputing a database of exact costs between pairs of points in a localised area.
Both Theta* and Block A* improve on the running time and 
solution quality of string pulling but neither guarantees optimality.
Despite this, a number of exact solutions for the problem do exist.
%\\
%In computational geometry a problem similar to any-angle pathfinding exists 
%which has been very well studied: finding Euclidean shortest paths among 
%polygonal obstacles in the plane.
%\\
\\
Visibility Graphs~\citep{lozanoperez79} solve a generalised form of 
any-angle pathfinding in time $O(n^2\log_{2}n)$. Their primary disadvantage
is that storing such a graph requires $O(n^2)$ space as each of the $n$ 
vertices can be adjacent to (i.e. visible from) every other vertex.  
Tangent Graphs~\citep{liu92} are a particularly efficient 
variant of this idea but space requirements remain worst-case quadratic. 
Other exact approaches are based on the 
Continuous Dijkstra~\citep{mitchell87} paradigm.
The most efficient of these algorithms~\citep{hershberger99} involves a
precomputation requiring $O(n \log_{2}{n})$ space and $O(n\log_{2}n)$ time.  The result 
is a planar subdivision of the map that can be used to find a shortest path in 
just $O(\log_{2}n)$ time but only for queries originating at a fixed source.
\\
In this work we discuss a new approach to any-angle pathfinding which
addresses shortcomings associated with existing research.
Our algorithm, Anya, bears some similarity with Continuous Dijkstra: 
instead of searching over individual states from the grid we consider
contiguous sets of states together as an interval. From each interval we select a 
representative point that is used to derive an $f$-value for the set.
Intervals are associated with corner points and projected from one 
row of the grid onto another until the goal is reached.
Anya compares favourably with existing research: (i)
it always finds a Euclidean shortest path between any two vertices on
a grid map (ii) it does not rely on any precomputation (iii) 
it does not introduce any memory overheads beyond those required by a 
pathfinding algorithm such as A*.

%http://www.valvesoftware.com/publications/2009/ai_systems_of_l4d_mike_booth.pdf
%http://digestingduck.blogspot.com.au/2010/03/simple-stupid-funnel-algorithm.html
