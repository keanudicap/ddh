\section{Summary}
\label{cha::conclusion::summary}
We study distance-optimal pathfinding in discrete and static two-dimensional
environments. Our target applications are mobile robot navigation and
pathfinding in computer games, both settings where the world is commonly
represented as a grid. Our objectives throughout this thesis have been to: (i)
find the shortest path; (ii) as quickly as possible; (iii) as economically as
possible with respect to available resources such as memory and
pre-computation time.

We identified two signficant challenges that arise when trying to find optimal
paths on grid domains: (i) grids are dense and finding an optimal path often
means expanding very many nodes; (ii) paths which are grid-optimal are often
not Euclidean-optimal which is undesirable from both an an efficiency
perspective and an aesthetic perspective. We address the first of these
challenges in Chapter~\ref{cha::rsr}, Chapter~\ref{cha::jps} and
Chapter~\ref{cha::jps2}.  We address the second of these challenges in
Chapter~\ref{cha::anya}.

In Chapter~\ref{cha::rsr} we identify one signficiant factor that makes
pathfinding in grids challenging: the existence of equivalent permutations for
each shortest path in a grid. We formalise this idea by defining concretely
what it means for two paths to be symmetric. We then give a description of and
experimental evaluation for an offline symmetry-breaking approach called
Rectangular Symmetry Reduction (RSR). We show that RSR can improve the
performance of the classical A{*} algorithm by several factors in most of our
benchmark domains and up to one order of magnitude in some other cases. We
also compare RSR with two contemporary pathfinding algorithms:
Swamps~\citep{pochter10} and the Portal Heuristic~\citep{goldenberg10}. We
find that RSR has complementary strengths with these approaches and we
identify many instances where RSR dominates convincingly.

In Chapter~\ref{cha::jps} we present Jump Point Search (JPS): an alternative
symmetry breaking procedure which selectively expanding only certain nodes on
a grid map which called \emph{jump points}.  JPS is recursive in nature,
guaranteed optimal and applied entirely online. It requires no pre-processing
and introduces no memory overheads. It can improve the performance of standard
A{*} by one order of magnitude and more in each of our benchmark domains. JPS
compares favourably with and convincingly dominates contemporary optimal
speedup techniques such as Swamps. It is also competitive with the non-optimal
hierarchical pathfinding method HPA{*}.  JPS is unique in the pathfinding
literature in that it has very few disadvantages: it is simple, yet highly
effective; it preserves optimality, yet requires no extra memory;  it is
extremely fast, yet requires no pre-processing.  Further, JPS is completely
orthogonal to and easily combined with competing speedup techniques from the
literature.  We are unaware of any other algorithm which has all these
features.

In Chapter~\ref{cha::jps2} we present several enhancements that further
improve the performance of Jump Point Search. The first enhancement is a set
of improved pruning rules that allow JPS to prune from consideration certain
types of nodes known as intermediate jump points.  The second enhancement
allows JPS to consider ``blocks'' of nodes at one time when breaking
symmetries.  The third enhancement is an offline pre-processing technique that
speeds up search by identifying jump points apriori.  We show in an empirical
evaluation that these improvements have a strong positive effect on Jump Point
Search, improving runtimes by anywhere from several factors to one order of
magnitude. We also compare our approach with two variants of a
SUB~\citep{urasKH13} a recent and very fast approach that breaks path
symmetries during an offline pre-processing step. We find that the two
techniques have complementary strengths and we identify large sets of
instances where JPS is faster.

In Chapter~\ref{cha::anya} we turn our attention to the problem of finding a
Euclidean-optimal path in a grid map. We describe Anya: a novel approach to
this problem which involves considering sets of states together as an
interval. From each interval we select a single representative point which we
use to compute an $f$-value for the entire set.  During search intervals are
projected from one row of the grid to another until the target is reached. We
give theoretical results showing Anya always finds a Euclidean-optimal path if
one exists. Moreover, Anya does in an entirely online fashion. We are unaware
of any other approach with these desirable characteristics.

