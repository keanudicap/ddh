\section{Future Work}
\label{cha::conclusion::future}
We discuss future work in terms of our two main contributions: Symmetry breaking
in pathfinding search and optimal approaches for Any-angle pathfinding.

\subsection{Symmetry Breaking}
One interesting direction for further work involves stronger pruning rules
for Jump Point Search on both weighted and unweighted grid maps.
In the weighted case we have given one simple suggestion for adapting the 
existing algorithm: simply stop whenever any neighbouring tile has a different
terrain type than the current node. 
A more general approach is to explicltly calculate the 
weighted cost of local paths: from the parent node to each neighbouring node. 
The advantage is that we do not necessarily need to stop each time the path
transitions from one weighted region to another. We need to be careful however
to balance the additional time required to explicltly compute the costs of 
each local path vs. the time savings from performing fewer node expansions.

In the case of unweighted grid maps we believe stronger symmetry breaking
rules are possible which will allow us to jump over at least some of the
remaining nodes currently expanded by Jump Point Search. For example we might
consider pruning a node $n$ if all its successors have the same $f$-value as
$n$; i.e.  we can try to keep jumping as long as we are heading in the
direction of the goal.  A stronger variant of this idea is to keep jumping as
long as we are heading in the same direction (or a complementary direction) to
the one used to reach $n$. For example, suppose we reach node $n$ by jumping
in direction $\vec{d}$.  If all the successors of $n$ can be reached by also
jumping in direction $\vec{d}$ or in another direction which is a component of
$\vec{d}$ then we can consider pruning $n$ and continuing to recurse in each
of these directions. It is likely that this procedure will increase in the 
branching factor at $n$ but we posit that fewer node expansions are required 
overall because we do not necessarily need to stop each time the path turns. 
A similar idea is described in~\citep{urasKH13} where the authors also exploit
path symmetries to speed up search -- though their approach is is offline
only.

Another interesting direction for future work is to apply Jump Point Search 
to other types of grids, such as hexagons or texes~\citep{yap02}. We propose
to achieve this by developing  a series of pruning rules analogous to those
given for square grids.  As the branching factor on these domains is lower
than square grids, we posit that jump points could be even more effective 
than we have observed for square grids.

Finally, synthesisng JPS with complementary speedup techniques from the
literature appears to be a promising line of research. For example JPS could
be used to speed up optimal search in an implementation of
Swamps~\citep{pochter08} or to speed up start and target insertion in an
approximate hierarchical pathfinder such as HPA{*}~\citep{botea04}.

\subsection{Any-angle Pathfinding}
An obvious direction for further work is a concrete implementation of Anya
together with an empirical evaluation. This is a topic of current research.

Another interesting direction is to generalise the theoretical results from
Anya to the problem of finding Euclidean-optimal paths in continuous planar
environments with polygonal obstacles -- rather than the square obstacles of
the type found in grid maps. Such a generalisation could be achieved by (i) 
using a grid to tesselate the environment and (ii) testing if any obstacles
intersect the current interval as we move it from one row of the grid to anoter.
Each time we detect an intersection we generate a successor interval whose y-axis
is the same as the obstacle rather than the y-axis of the next row.
