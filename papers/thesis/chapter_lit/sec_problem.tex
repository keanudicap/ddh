\section{The Single-agent Pathfinding Problem}
\label{cha::lit::problem}
Single-agent pathfinding is the problem of navigating a lone entity, for example
a robot or a virtual character, from one point to another in a given operating 
environment.  In the most common setting environments are two-dimensional
Euclidean (i.e.  flat) or three-dimensional Geodesic (i.e.  curved) spaces. They
can take the form of a \emph{spatial network} (i.e. a set of connected points)
or they can be described as a collection of traversable and non-traversable
polytopes; the latter often being called obstacles.  There are many variations
of the single-agent pathfinding problem. These arise by adjusting certain
parameters associated with the problem such as:
\begin{itemize}
\item The objective function. In the canonical case the aim is to minimise
travel distance but other objective functions (including more than one at a
time) could also be used.
\item The type of agent. In the canonical case agents are modeled as points
but they could have arbitrary shapes, sizes and capabilities that 
restrict or enhance movement. % through the environment.
\item The operating environment. In the canonical case agents operate in a 
(i) static (ii) fully observable (iii) discrete environment. 
Depending on the application however, any or all of these assumptions may need 
to be lifted.
\item Solution quality. In the canonical case agents are required to find 
an optimal path.
In some real-time or resource constrained settings however a near-optimal 
or bounded suboptimal path may be preferable.
\item Path constraints. In the canonical case  the agent is simply required 
to move from the start position to the target position without intersecting
any obstacles. In other settings additional constraints may complicate this
task; for example the agent may need to visit certain pre-specified locations 
before arriving at the target.
\end{itemize}


In this thesis we will focus on optimal single-agent pathfinding 
in both discrete and continuous environments. We assume the environment is
static and fully observable and that the agent can be modeled as a point.
We will employ an objective function that minimises travel distance and 
we do not have any path constraints.

\subsection{Search Graphs}
\label{cha::lit::problem::graphs}
Regardless of the problem variation at hand, practitioners typically all begin
by constructing a model of the operating environment $G = (V, E)$ known as a 
\emph{search graph}. Here $V$ is a set of permissible locations that an agent
can occupy; these are often referred to as the nodes or \emph{vertices} of the
graph. Meanwhile $E$ is the set of \emph{edges} that connect adjacent vertices.
Edges can be thought of as roads or corridors that an agent can travel along or
actions that can be executed in order to transition the agent from one location
to another.  The cost associated with each such move is called the \emph{edge
weight}.  Weights often represent distance travelled but they could stand for
other types of metrics as well; e.g.  travel time or fuel consumption.  When the
cost of moving between two vertices $a$ and $b$ can differ to the cost of moving
from $b$ to $a$ the graph is said to be \emph{directed}\footnote{This includes the
case when the cost in one direction is $\infty$}.  Otherwise 
the graph is said to be \emph{undirected}.


\subsection{Paths and Instances}
\label{cha::lit::problem::instance}
A \emph{path} $\pi = \langle v_0, v_1, \ldots, v_{k-1},  v_k \rangle$
can be interpreted as a walk in a search graph $G = (V, E)$.
Each $v_i$ is a vertex in $V$ and each pair of adjacent 
vertices $(v_i, v_{v+1})$ are connected by an associated edge in $E$.  
When searching for a path we designate the start location of the agent $s$ and 
its destination or target location $t$. An \emph{instance}, then, is an
\emph{s-t} pair that concretely define a pathfinding problem. 

Once a path has been found its \emph{quality} is typically evaluated in one of 
two ways: length or cost.
The \emph{length} of a path refers to the number of edges that comprise the path.
The \emph{cost} of a path refers to the total weight of all edges that comprise the path. 
An \emph{optimal} path is the lowest-cost path between vertices $s$ and $t$
which can be found in $G$.

