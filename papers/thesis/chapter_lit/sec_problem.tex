\section{Problem Statement}
\label{cha::lit::problem}
Pathfinding is the problem of navigating between an arbitrary pair of points on
a given \emph{input map}.  In the most common setting input maps are
two-dimensional Euclidean (i.e. flat) or three-dimensional Geodesic (i.e. curved) 
spaces. They can take the form of a \emph{spatial network} (i.e. a set of
connected points) or they can be given as a collection of traversable and
non-traversable polytopes; the latter often being called obstacles.

In order to find an obstacle-free path from one point to another practitioners
begin by constructing from the input map a data structure $G = (V, E)$
known as a \emph{search graph}.  Here $V$ is a set of permissible locations that
an agent can occupy; these are often referred to as the nodes or \emph{vertices}
of the graph.  Meanwhile $E$ is the set of \emph{edges} that connect adjacent
vertices.  Edges can be thought of as roads or corridors that an agent can
travel along or actions that can be executed in order to transition the agent
from one location to another.
The cost associated with each such move is called the \emph{edge weight}.
Weights often represent distance travelled but they could stand for other types
of metrics as well e.g. travel time or fuel consumption.  When the cost of
moving between two vertices $a$ and $b$ can differ to the cost of moving from $b$ to
$a$ the graph is said to be \emph{directed}.  When this is not the case, the
graph is said to be \emph{undirected}.  

Search graphs can be explicit, in which case all vertices and edges are
enumerated apriori, or they can be implicit, in which case the graph is
gradually built during search. Explicit graphs arise in many settings including
computer games~\citep{davis00,tozour02,champandard09}, 
routing~\citep{sanders05,goldberg06} and robot motion
planning~\citep{latombe91,choset05}.  Implicit graphs appear in higher
dimensional pathfinding settings~\citep{lavalle98,bohlin00} and related
application areas such as AI Planning~\citep{russel03}.  
