\section{Symmetry Breaking}
\label{cha::lit::symmetry}
Symmetry is a naturally occurring phenomenon that arises whenever we are forced
to enumerate permutations of elements in a set. Typically
undesirable, symmetry forces search algorithms to waste time evaluating nodes that
are equivalent to each other and prevents real progress toward the goal.
For example: a common problem in AI involves assigning resources to a set of 
functionally identical machines. Systematic search algorithms will proceed by 
generating in this case a unique state for every possible assignment of resource
to machine -- despite the fact that many of these assignments are equivalent and
produce identical results.

In the context of Constraint Programming symmetries arise when interchanging
the order in which values are assigned to variables or when the variables themselves
are interchangeable~\citep{walsh07}. Practitioners in the field deal with symmetries
by posting constraints that restrict the order in which assignments can occur.
In some cases these constraints are produced by a preprocessing step,
 e.g. as in~\citep{crawford96}, or computed dynamically during search,
 e.g. as in~\cite{roney-dougal04}. Each approach involves a compromise between 
effectiveness and efficiency.

A number of symmetry-breaking approaches involving transition systems have been described in the
literature of Model Checking and AI Planning. For example in~\cite{emerson96} and~\cite{pochter11}
an equivalence between states is established by associating each state in the graph with a group of
states that are symmetric. The authors identify a canonical state for each group and derive from
these a quotient transition system that is free of state-symmetry. In~\citep{rintannen03} a similar
idea is applied to propositional planning. In this case however the authors identify symmetric
transitions rather than symmetric states.  Though less powerful than state symmetries, transition
symmetries can be computed in low-order polynomial time; computing state symmetries is $NP$-hard in
general.

A different type of symmetry

\begin{itemize}
\item \cite{fox99} identify interchangeable objects and sets of symmetric actions
in the context of propositional planning with \textsc{GraphPlan}. By removing both
they produce smaller planning graphs and increase the efficiency of seach.
\item Commutativity Pruning~\cite{haslum00} identifies sets of actions that are 
pairwise commutative. To avoid symmetries arising in cases where such actions
need to be applied sequentially, the authors propose a canonical ordering 
\item  

\end{itemize}

Despite a wealth of literature on the topic of symmetry breaking, in areas
as diverse as AI Planning, Constaint Programming, Combinatorial Optimisation
and Model Checking, there have been no works that explicitly identify symmetry 
in pathfinding search.

\subsection{TODO}
Approaches for identifying and eliminating search-space symmetry have been
proposed in areas including planning \cite{fox99}, constraint programming
\cite{gent00}, and combinatorial optimization \cite{fukunaga08}. 
Very few works however explicitly identify and deal with symmetry in pathfinding
domains such as grid maps. 

\begin{enumerate}
\item{Symmetry Breaking in CP and CO}
\item{Isomorphism Detection}
\item{Partial Orderings}
\item{Stubborn Sets}
\item{Duplicate Pruning}
\end{enumerate}

