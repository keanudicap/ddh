\section{Symmetry Breaking}
\label{cha::lit::symmetry}
Symmetry is a naturally occurring phenomenon that arises whenever we are forced
to enumerate permutations of multiple indistinguishable objects. Typically
undesirable, symmetry forces search algorithms to waste time and prevents real
progress toward the goal.  For example: a common problem in AI involves
assigning resources to a set of functionally identical machines. Systematic
search algorithms will proceed by generating in this case a unique state for
every possible assignment of resource to machine -- even though many of these
assignments produce identical results. 

In the area of AI Planning, a number of works deal with the explicit identification
and elimination of symmetries in search:

\begin{itemize}
\item \cite{fox99} describe a static symmetry-breaking technique in a 
a propositional \textsc{GraphPlan} framework. Their
approach identifies sets of symmetric objects and, from those, sets
of symmetric actions. By removing both they are able to reduce the size
of the planning graph and speed up search.
\item Commutativity Pruning~\cite{haslum00} is a static approach for 
identifying sets of actions whose members are all pairwise commutative.
Since permutations of such actions lead to symmetric states, a total
ordering is defined over the members of each set. This eliminates
many partial-order symmetries that would otherwise appear during search.
\item 

\end{itemize}

\subsection{TODO}
Approaches for identifying and eliminating search-space symmetry have been
proposed in areas including planning \cite{fox99}, constraint programming
\cite{gent00}, and combinatorial optimization \cite{fukunaga08}. 
Very few works however explicitly identify and deal with symmetry in pathfinding
domains such as grid maps. 

\begin{enumerate}
\item{Symmetry Breaking in CP and CO}
\item{Isomorphism Detection}
\item{Partial Orderings}
\item{Stubborn Sets}
\item{Duplicate Pruning}
\end{enumerate}

