\section{Discussion}
\label{cha::lit::discussion}
Single-agent Pathfinding is a common topic in many areas of Computer Science. It is studied in the
context of Algorithmics, Artificial Intelligence, Computational Geometry, Computer Graphics and
Robotics.  The most common formulation involves an agent navigating amidst obstacles in a two or
three dimensional operating environment.  Researchers often simplify the problem by discretising the
environment into a fixed set of connected points called a search graph. Such discretisation
approaches include:
\begin{itemize}
\item Grid Maps.
\item Navigation Meshes.
\item Roadmaps.
\item Shortest Path Maps.
\item Visibility Graphs.
\end{itemize}

There exist several popular strategies for finding a path in a search graph: (i)
Blind Search; (ii) Local Search; (iii) Informed Search. Each has its own advantages and 
disadvantages. Blind Search methods, for example, unfold a search graph into a tree and
explore the nodes in a fixed order. Under certain conditions these algorithms are 
guaranteed to find a path (including the optimal path). However, they
require an amount of time, or space, that is exponential in the length of the returned 
solution.
Local Search algorithms, by comparison, use simple rules-of-thumb to find a path.
They are fast and memory efficient but are not guaranteed to find the optimal 
path or indeed any path.
Informed Search algorithms are systematic pathfinding strategies that employ heuristic
lower-bounds during pathfinding search. Such algorithms rank 
nodes by merit and always explore the most promising node first. Informed Search methods
are popular because they are complete, optimal and optimally efficient. 

Despite very attractive theoretical characteristics there exist many domains of practical 
interest where even Informed Search strategies cannot find optimal paths within a specified
time limit or using only a fixed amount of memory. A variety of techniques have been
developed to speed up search in these cases. These can be broadly categorised as follows:
\begin{itemize}
\item Spatial Abstraction.
\item Improved Heuristics.
\item Symmetry Breaking Techniques.
\item Dominance Detection Methods.
\end{itemize}

Within the Artificial Intelligence community pathfinding with spatial abstraction usually involves
creating small approximations of a given input graph.  Though fast and memory efficient such
approaches do not find optimal paths.  A related strategy involves improving the accuracy of 
heuristic functions that guide search. Such methods usually guarantee optimality but often at the 
cost of substative memory overheads.  

Within the Algorithmics community a different line of research has led to the development of 
specialised abstraction hierarchies that, unlike their counterparts in AI, can preserve solution 
optimality. Developed for routing on road networks, these approaches are
very fast but suffer from two drawbacks: (i) they introduce substantive memory overheads 
(compared to the amount of space needed to store the map) and (ii) they can perform poorly on 
graphs that do not have very low degree.

In some domains of practical interest, such as grid maps, it has been suggested that optimal
pathfinding is made difficult by the existence of many equivalent optimal paths. To deal
with such cases researchers in Artificial Intelligence have developed a variety of 
Dominance Detection technques.  Typically preprocessing-based, such algorithms focus on the 
identification of areas on a map that do not need to be considered during search. Ignoring 
these areas can speed up informed search strategies such as A* by many factors; albeit at the 
cost of some memory overhead.

Equivalences in search domains can be due to paths having direct symmetries in the environment (i.e.
automorphisms between states) or they can be the result of interleaving equivalent and
interchangeable actions (i.e. isomorphisms between sequences of states). A large body of work
dealing with both types of symmetries can be found in the literature of AI Planning, Constraint
Programming and Model Checking. Such works are usually deveolped in the context of transitions
systems having compact representations and a small set of applicable operators. By comparison,
pathfinding domains have large explicit representations and transitions that are state-specific.
Thus there is little work applying such ideas to pathfinding.

Until now our discussion has focused on graph-theoretic approaches that rely on discretisation 
to simplify the operating environment. In the areas of Computational Geometry and Computer 
Graphics there exists a line of research devoted to solving the Euclidean Shortest Path
Problem. This is a variation on pathfinding that focuses on exact solutions to navigation 
problems rather than graph-based solutions. A variety of efficient techniques exist in this 
context but all are preprocessing based and require memory overheads. The most performant of 
these, Continuous Dijkstra, requires a fixed starting state; a very strong and often unreasonable 
constraint. 
Any-Angle Pathfinding an instance of the Eucliean Shortest Path Problem which is restricted
to grid-based environments. The problem appears in the context of Computer Games and Robotics.
A number of methods from AI and Game Development exist that can solve the problem but all
are suboptimal. One recent algorithm, Accelerated A*, is conjectured to be optimal but no
proof is given. 

