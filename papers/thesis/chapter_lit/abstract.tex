\chapter{Literature Review}
\label{cha::lit::abstract}
%Pathfinding is the problem of navigating from one location to another on a 
%given two or three dimensional input map.
%The problem is among the oldest in Computer Science and appears in a range
%of common application areas including logistics, robotics and computer games.
In this chapter we discuss two common formulations of the pathfinding problem: 
finding a shortest path in a discrete graph and finding a shortest path in a 
continuous plane.
We consider a range of approaches for constructing search graphs including grid
maps, road maps and navigation meshes.  We then compare and contrast a range of
both classical and more recent approaches for efficiently computing optimal and
near-optimal paths between two points in a discrete graph.  These include:
heuristic methods, abstraction techniques and search space pruning algorithms.
Particular emphasis is given to optimality-preserving methods which speed up
search via symmetry breaking and isomorphism detection.
Finally, we examine known geometric techniques for solving pathfinding problems
in continuous planar environments.  This variation of the pathfinding problem is
particularly challenging as there are no exact methods that can solve the
problem online.  We compare and contrast a range of known methods, both exact
and approximate, and discuss their various tradeoffs.


