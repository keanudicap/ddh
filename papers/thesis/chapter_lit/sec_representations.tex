\section{Input Maps and Search Graphs}
Pathfinding involves navigating from one location to another on a given input
map. In the most common setting input maps are two or three dimensional
Euclidean (i.e. flat) or Geodesic (i.e. curved) spaces which are represented as
a collection of traversable and non-traversable polygons.
To find a path between an arbitrary pair of points on a map practitioners 
create a model of the input map known as a \emph{search graph} $G = (V, E)$. 
Here $V$ is the set of all possible locations, often referred to as the nodes or
\emph{vertices} of the graph, and $E$ is the set of \emph{edges} that connect
vertices.  Edges can be thought of as single step actions that an agent
must perform in order to move from one vertex to another.
The cost associated with each move is called the \emph{edge weight}.
Weights often represent distance traveled but they could stand for other
types of metrics as well e.g. travel time or fuel consumption.
Search graphs can be explicit, in which case all vertices and edges
are enumerated apriori, or they can be implicit, in which case the graph is 
gradually built during search.
Many approaches exist for constructing a search graph from a given input map.

%Implicit graphs are common in settings where the input map is unknown or is too
%large to be modeled explicitly. In this thesis, as in many pathfinding settings
%from robotics and computer games, we will assume the input map is known and that
%all vertices and edges can be enumerated.  
%
%Need to say something about the input map. Usually two and three dimensions
%(can be n dimensions though) it consists of a description 	

\subsection{Grid Maps}

\subsection{Road Maps}

\subsection{Navigation Meshes}

\subsection{Visibility Graphs}

\subsection{Voronoi Diagrams}

