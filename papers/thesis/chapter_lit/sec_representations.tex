\section{Search Graphs}
\label{cha::lit::graphs}
Pathfinding is the problem of navigating between an arbitrary pair of points on
a given \emph{input map}.  In the most common setting input maps are
two-dimensional Euclidean (i.e. flat) or three-dimensional Geodesic (i.e. curved) 
spaces. They can take the form of a \emph{spatial network} (i.e. a set of
connected points) or they can be given as a collection of traversable and
non-traversable polytopes; the latter often being called obstacles.

In order to find an obstacle-free path from one point to another practitioners
begin by constructing from the input map a data structure $G = (V, E)$
known as a \emph{search graph}.  Here $V$ is a set of permissible locations that
an agent can occupy; these are often referred to as the nodes or \emph{vertices}
of the graph.  Meanwhile $E$ is the set of \emph{edges} that connect adjacent
vertices.  Edges can be thought of as roads or corridors that an agent can
travel along or actions that can be executed in order to transition the agent
from one vertex to another.
The cost associated with each such move is called the \emph{edge weight}.
Weights often represent distance traveled but they could stand for other types
of metrics as well e.g. travel time or fuel consumption.  When the cost of
moving between two vertices $a$ and $b$ can differ to the cost of moving from $b$ to
$a$ the graph is said to be \emph{directed}.  When this is not the case, the
graph is said to be \emph{undirected}.  

Search graphs can be explicit, in which case all vertices and edges are
enumerated apriori, or they can be implicit, in which case the graph is
gradually built during search. Explicit graphs arise in many settings including
computer games~\citep{davis00,tozour02,champandard09}, 
routing~\citep{sanders05,goldberg06} and robot motion
planning~\citep{latombe91,choset05}.  Implicit graphs appear in higher
dimensional pathfinding settings~\citep{lavalle98,bohlin00} and related
application areas such as AI Planning~\citep{russel03}.  

Many approaches exist for creating a search graph from a given input map; we
discuss a broad range of popular methods in the remainder of this section.

\subsection{Grid Maps}
\label{cha::lit::graphs::grid}
A grid map is a tesselation of unit squares, often called tiles, which is
applied over a planar input map. Each tile has up to eight adjacent neighbours
and is marked as traversable or non-traversable depending on whether or not it
intersects any obstacles.  Agents are often modeled as unit-size entities that
occupy a single tile and move from the centre of one traversable tile to the
next.  
An alternative approach is to model agents as point-size entities that occupy 
the intersections of the grid and which travel in straight steps along the
explicit edges of grid and diagonally through tile interiors.
In both cases straight steps incur a cost of 1. Diagonal steps, if permitted, 
incur a cost of $\sqrt{2}$.  When diagonal moves are not permitted the grid map 
is said to be \emph{4-connected}; otherwise it is \emph{8-connected}.  

Grid maps are popular for several reasons: (i) they are simple to understand 
and simple to apply (ii) they can be represented as a matrix of bits and stored
efficiently (iii) individual nodes can be identified and queried in constant time.
One significant disadvantage of grid maps is their fixed resolution. In many cases 
grids are too coarse to accurately model the underlying input map. Increasing the
number of tiles is not always possible: there is always a corresponding increase in
memory requirements and searching in a larger grid often makes pathfinding more 
challenging.
Another disadvantage of grid maps is that they produce paths which are
constrained to the points of the grid. Such paths are not only aesthetically
displeasing but they can also be longer than stricictly necessary and may
require post-processing to ``smooth'' them.

\subsection{Road Maps}
\label{cha::lit::graphs::road}
A road map is a spatial network which is derived from a given input map. 
They are analogous to road networks which model automotive transportation systems 
in the physical world.

PRMs.
Disadvantage: only probabilistically complete. Paths can be very suboptimal? Complicated?

A variation of this idea~\citep{geraerts05} yields roadmaps which are representationally complete.
Instead of sampling the input map directly this approach applies a grid tesselation over the
space in the first instance  and then gradually constructs a set of ``guard points''. The process
terminates when every traversable tile is observed by at least one guard point. A roadmap is constructed
by computing a minimum spaning tree over the set of guard points. 

Voronoi diagrams.

Not clear which point on the input maps maps to which point on the road network. Maybe generate all visible?

\subsection{Navigation Meshes}
\label{cha::lit::graphs::nav}

\subsection{Visibility Graphs}
\label{cha::lit::graphs::vis}

\subsection{Voronoi Diagrams}
\label{cha::lit::graphs::voronoi}

