\section{Dominance Detection}
\label{cha::lit::dominance}
In many pathfinding domains, and especially in gridmaps, arbitrary problem instances rarely
have only a single optimal solution. These paths can be symmetric to each other 
(c.f. Section~\ref{cha::lit::symmetry}) or they can simply be equivalent in terms of cost. 
The identification and eliminination of such equivalences is an active topic of research.
Algorithms developed for this purpose are known as \emph{graph pruning} or 
\emph{dominance detection} techniques. A path $\pi$ is said to dominate $\pi$ if 
$\pi \leq \pi'$; here $\leq$ is a relation based on cost. By the same principle $\pi$ is 
said to be strictly dominant if $\pi < \pi'$. 

A number of pathfinding algorithms from the literature of Heuristic Search fit this 
description. The \emph{dead-end heuristic} \cite{bjornsson06} and \emph{Swamps} \cite{pochter10}
are two similar pruning techniques related to our work.
Both decompose grid maps into a series of adjacent areas. Later, this decomposition
is used to identify areas not relevant to optimally solving a particular
pathfinding instance.
This objective is similar yet orthogonal to our work where
the aim is to reduce the effort required to explore any given area in the search
space.

A different method for pruning the search space is to identify \emph{dead} and
\emph{redundant} cells~\cite{sturtevant10b}.  Developed in the context of
learning-based heuristic search, this method speeds up search only after running
multiple iterations of an iterative deepening algorithm.  Further, the
identification of redundant cells requires additional memory overheads which
jump points do not have.

\emph{Fast expansion}~\cite{sun09} is another related work that speeds up
optimal A* search. It avoids unnecessary open list operations when it finds a
successor node just as good (or better) than the best node in the open list.
Jump points are a similar yet fundamentally different idea: they allow us to
identify large sets of nodes that would be ordinarily expanded but which can be
skipped entirely.
