\section{Dominance Detection}
\label{cha::lit::dominance}
In many pathfinding domains, and especially in gridmaps, arbitrary problem
instances rarely have only a single optimal solution. These solutions can be
symmetric to each other (c.f. Section~\ref{cha::lit::symmetry}) or they can
simply be equivalent in terms of cost.  Pathfinding instances that have many
equivalent solutions are more difficult to solve; even for efficient search
algorithms such as A{*}. In such cases, before even one optimal path can be
found, A{*} must usually generate (and often expand) every node on every other
equivalent optimal path.  This is wasteful and prevents A{*} from making real
progress toward the target.  A number of works can be found in the literature
of Heuristic Search that identify and eliminate equivalent paths. Such
algorithms are known as \emph{graph pruning} or \emph{dominance detection}
techniques. We briefly outline them here.

\begin{definition}
A path $\pi$ \emph{dominates} another path $\pi'$ if $cost(\pi) \leq cost(\pi')$. 
If $cost(\pi) < cost(\pi')$ then $\pi$ is \emph{strictly dominant}.
\end{definition}

The Dead-end Heuristic~\citep{bjornsson06} and
Swamps~\citep{pochter09,pochter10} are two graph pruning techniques that can
eliminate from consideration many dominated and strictly dominated paths.
Both rely on an offline preprocessing phase that decomposes a given map into
obstacle-free areas.  During online search each of these two algorithms
consults the decomposed map in order to identify areas that are relevant for
the instance at hand and areas that can be ignored entirely because exploring
them could not improve the cost of any path. Recent work shows that a limited 
type of Swamp detection can also be performed online however that algorithm 
prunes fewer nodes than its offline counterpart~\cite{DBLP:conf/socs/SharonSF13}.
%It can be seen as a technique for detecting areas that don't have to be searched
%in the instance at hand and, as before, is complementary to our work.

MSA{*}~\citep{bolanca09} is another recent pruning technique which is
orthogonal to both Swamps and the Dead-end Heuristic.  Where those algorithms
identify areas that can be ignored during search, MSA{*} generates, on the
fly, macro-edges that facilitate fast travel through areas that do have to be
searched.  It relies on a preprocessing step that decomposes a given grid map
into a series of empty rectangular rooms. The key idea is that each such room
can be traversed in a single macro step. MSA{*} is similar in spirit to
Rectangular Symmetry Reduction; an alternative pruning algorithm that we
introduce in Chapter~\ref{cha::rsr}.  While our work can consistently speed up
A{*} search, MSA{*} is only able to achieve this in some cases.  It is also
worth noting that MSA{*} uses a different empty room decomposition method from
the one described in our work.

A different technique for pruning the search space is to identify \emph{dead}
and \emph{redundant} cells~\citep{sturtevant10b}.  Developed in the context of
learning-based heuristic search, this method speeds up search only after
running multiple iterations of an iterative deepening algorithm.  During each
iteration the algorithm stores the $g$-value associated with each expanded
node and prunes from subsequent iterations all nodes that were found to have
zero $g$-cost-increasing neighbours.

Fast Expansion~\citep{sun09} is another related work that speeds up A{*}
pathfinding search.  This algorithm detects when a successor of the current
node has an $f$-value that dominates the $f$-value of the best node
on the open list. The idea is to expand such nodes immediately and avoid
unnecessary operations on the open list.
