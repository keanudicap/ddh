\section{Improved Heuristics}
\label{cha::lit::heuristics}
%Heuristic functions are employed in search to evaluate states in terms
%of merit. In the case of local search, heuristics are used to choose 
%a successor state from a large set of possible successors. In the case
%of informed search, heuristics provide estimates on the remaining distance 
%to the goal. We discuss a range of representative works for computing such 
%lower-bounds. This is in keeping with the scope of this thesis which is
%concerned with optimality-preserving pathfinding techniques.

%A heuristic, in the context of optimal pathfinding search, is a function that 
%estimates the distance from any given node to the goal. 
%
In the context of optimal search a heuristic $h$ is a lower-bound function defined over states. Its
purpose is to estimate cost-to-go; i.e. to answer questions of the form ``how far is this node from
the target?".  Heuristics are useful if they can be computed efficiently (i.e.  polytime or better)
and if they are accurate.  A popular default when pathfinding in the plane is $h_{SLD}$, Also known
as Straight Line Distance, $h_{SLD}$ is consistent, admissible and runs in constant time (cf.
Section \ref{cha::lit::searc} for a definition of these terms).  In the absence of obstacles
$h_{SLD}$ is perfect but in more complex environments it can dramatically underestimate optimal
distances between two arbitrary points; e.g. as seen in~\citep{goldberg05}.

Many works focus on improving the accuracy of $h_{SLD}$ without negatively impacting its running
time. This approach usually translates into faster search algorithms. However, there are limits.  A
number of theoretical models~\citep{pohl77,helmert08} have shown that optimal search strategies
employing even almost perfect heuristics, i.e. those having only a small additive constant for
error, must expand, in the worst case, an exponential number of nodes before reaching the goal.  A
similar result can also be derived for the average case~\citep{pearl84}.  The models used in these
theoretical works often use certain simplifying assumptions.  They nevertheless align well with
domain models that appear in pathfinding search; e.g. constant branching factor, uniform-cost edges
and a singleton goal state.  Despite such seemingly discouraging results researchers have
demonstrated that, in many pathfinding domains of practical interest (e.g.~\citep{sturtevant12}),
better heuristic estimates can dramatically improve the performance of optimal search.

One idea for improving heuristic accuracy is to take into accout domain specific movement rules.
$h_{MD}$ (Manhattan Distance) and $h_{OD}$ (Diagonal or Octile Distance) are two such heuristics; they
retain all the properties of $h_{SLD}$ but provide better lower-bound estimates when pathfinding in
square grids. Similar domain-specific heuristics have been developed for pathfinding in hexagonal
grids~\citep{yap02} and more broadly for optimally solving large combinatorial puzzles~\citep{korf96}.

Another approach to improving heuristic accuracy involves pre-computing distance databases.  A
large family of related algorithms, sometimes called \emph{memory heuristics}, has been described in
the literature~\citep{goldberg05,bjornsson06,sturtevant07,felner09,goldenberg10,anderson10}.
ALT~\citep{goldberg05} is a typical representative. During a preprocessing step ALT selects from the
map a set of \emph{landmark} nodes and then computes a database of optimal distances from each node
to every landmark. Given such a database it can be shown that admissible estimates between a pair
of nodes can be computed by subtracting the distances from the two nodes at hand to a fixed landmark.
ALT takes as its heuristic estimate the maximum difference over all landmarks.  
Further enhancements to this type of approach involve the combination of distance databases with
spatial abstraction. Such examples include the Gateway Heuristic~\citep{bjornsson06} and the 
Portal Heuristic~\citep{goldenberg10}. Memory heuristics are very fast, typically constant time, 
and can improve the performance of optimal search by anywhere from several factors to an order of
magnitude. Their primary disadvantages are often lengthy pre-processing times and significant space
overheads.  This largely excludes memory heuristics from being applied in many popular pathfinding
settings, including robotics and video games.

A different line of work has investigated the automatic construction of memory heuristics.  In a
recent paper~\cite{rayner11} describe a process for computing entire families of Euclidean
heuristics. Their approach involves embeding a spatial network graph into a manifold. Computing the
embeding requires solving a constrained optimisation problem that minimises heuristic error. The
authors show that their approach produces very accurate results in practice but solving the
optimisation problem can require significant preprocessing time and space.  Compressed Path
Databases (CPDs)~\citep{sanka05,botea11,botea13} are another powerful method for automatically
constructing memory-based heuritics; this time from spatial abstractions.  A CPD can be described as
a highly compressed set of tables that store the first move on the optimal path between any pair of
nodes in a graph. Through a simple process of recursive lookups CPDs can extract any optimal path in
near-linear-time and without employing any state-space search. Though among the fastest pathfinding
techniques today, CPDs nevertheless require substantial amounts of memory and are thus not
applicable in many popular pathfinding settings.

%Pattern Databases (PDBs)~\cite{culberson96} are a powerful method for automatically constructing
%memory-based heuristics from domain abstractions. Though often employed in solving large 
%combinatorial problems, PDBs cannot be feasibly applied in pathfinding as the goal state is 
%not fixed. Block A*~\cite{yap11} is one attempt at applying PDBs 
