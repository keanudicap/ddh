\section{Target Applications and Research Challenges}
\label{cha::intro::challenges}
In this thesis we study optimal pathfinding in the context of two-dimensional
computer games. We focus on real-time strategy games (such as
StarCraft, WarCraft et al.) and real-time role-playing games (Baldur's Gate,
Dragon Age, etc.). In both cases the game map is often represented as a grid of 
traversable and non-traversable cells. Though our principal results
are derived in the context of computer games they are equally applicable to
other related application areas such as two-dimensional mobile robot navigation,
finding shortest paths on printed circuit boards and indeed any setting where the 
operating environment can be described compactly in terms of fixed adjacency
relations between nodes on a map.

%In two-dimensional computer games -- real-time strategy games (such as
%StarCraft, WarCraft et al) and real-time role-playing games (Baldur's Gate,
%Dragon Age etc.) -- the map is often represented as a grid of traversable and
%non-traversable cells.  

%We study a range of fast and resource efficient techniques 
%for the canonical form of the shortest path problem and a related variant.
%Our results are equally applicable to other related target
%applications such as two-dimensional mobile robot navigation.
%

Grid maps are popularly employed in computer games for several reasons: 
(i) they are simple to understand and apply; (ii) they are memory efficient (only one bit of storage is required per grid cell); (iii) 
they facilitate fast spatial reasoning by providing constant-time
access to individual grid cells. At the same time, grids have two significant
disadvantages.
%The main disadvantages of grid maps are two-fold: 
First, there are often many cells in a grid and the distance between them is small.
Many steps may need to be taken before the target can be reached and this 
makes pathfinding surprisingly challenging. Second, computed paths must always pass through the fixed points of the grid (either
the centre of each cell or else the corners). 
This means that characters navigating in a game world behave
unrealistically: they always turn at fixed angles of 45 and 90 degrees and
they follow paths which, although optimal with respect to the points of the
grid, are not optimal with respect to the underlying game environment.

Our first research challenge, relating to the difficulty of pathfinding on a grid,
is a topic that has been previously examined in the academic literature. 
Many techniques have been developed; they can be broadly categorised into
one or more of: spatial abstraction, graph
pruning  and memory heuristics. Each of these approaches involves some
pre-computation before pathfinding can begin and each of these approaches
trades either memory or optimality for speed. Spatial abstraction techniques
for example create a small approximation of the grid map that is faster to
search but computed paths are not guaranteed to be optimal. Graph pruning and 
memory heuristics retain optimality but require significant investment in terms 
of additional memory and up-front pre-processing time: both scarce resources
in computer games.

The second research challenge, relating to the optimality of grid paths,
is known in the academic literature as the Any-Angle Pathfinding Problem. 
Given a grid map and two points upon it, we are asked to compute a shortest path 
that does not necessarily intersect the fixed points of the grid.
A range of approaches exist, both optimal and approximate, 
including solutions to a more general form of the problem known as the 
Euclidean Shortest Path Problem.
Optimal approaches can be typically characterised as very fast but requiring
pre-computation and (sometimes extensive) memory overheads. 
These methods are not applicable in many computer game and robotics contexts as 
each time the map changes the pre-computation must begin anew.
Approximate techniques on the other hand are fast, online and overhead-free but 
by their very nature do not yield distance-optimal paths.

