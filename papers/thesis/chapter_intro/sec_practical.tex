\section{Practical Considerations}
\label{cha::intro::practical}

%There are two common questions which every pathfinding practitioner must address
%at the outset:
%Regardless of the particular variant at hand there exist two common questions that 
%each formulation must address:
% (i) how to construct a map that represents the operating 
%environment in which we want to navigate and (ii) how to actually search the map
%for a path.
The first question which every pathfinding practitioner must address at the 
outset is how to best construct a map that represents the operating
environment in which we want to navigate.
Maps used for pathfinding are exactly analogous to their real-life cartographic 
counterparts: they document the salient features of an environment such as the
%\section{Types of Maps}
%\label{cha::intro::map}
%Before we can start navigating from one location to another -- whether in 
%in a physical or virtual environment -- we need to first construct a map to
%represent the salient features of the environment; features such as the
locations of roads, the lengths of road segments, terrain type and elevation
and the placement of both natural and artifical obstructions such as buildings 
or waterways. 
Depending on the application and setting some of these 
features may be more important or less important; still others we might choose
 to simply ignore. For example:

\begin{itemize}
\item
In computer games the virtual environment is 
often discretised into a grid of traversable and non-traversable cells. Another
popular approach,  known as a navigation mesh, discretises the environment into 
a set of simple polygons. Grids are often favoured for their simplicity while 
meshes are favoured for their typically small size. 
%The main
%advantage of grids is simplicity; their main disadvantage is that at high resolutions
%grids become very large and ineffcient to search. Navigation meshes meanwhile
%require less space but they are complicated to construct and need to be repaired
%if the environment changes.
%In other domains specialised types of maps used for pathfinding have also arisen: 
\item In logistics and personal navigation road maps are ubiquitous while 
in robotics visibility graphs are very popular. Both capture details of the physical 
environment that are most salient for the application at hand. In the case of road maps, it is critial
to accurately describe the features of physical transportation networks but details about the rest
of the world can be discarded. Similarly, visibility graphs only capture information
about which points in the environment can be reached from one another by travelling in a straight
line; everything else is secondary from the perspective of a mobile robot.
\end{itemize}
%Grid maps are simple to construct and very popular
%with researchers and practitiners 
%build information rich topographic models and they are popular with researchers
%and practitioners alike. 
%Their main disadvantage is that at high resolutions, grids become
%very large and they can be time-consuming to search -- a problem further compounded when
%trying to find a path in two or three dimensions.
%They are easy to understand, trivial to apply and 
%A grid map for example is a decomposition of a physical or virtual environment 
%into square cells that are marked as either traversable or non-traversable. 
%Grids are very popular in robotics and computer games because they are easy to 
%understand, trivial to construct and because they can be used to build information 
%rich topographic models. The main disadvantage is that at high-resolution grids
%become very large and can be time-consuming to search -- a problem further compounded
%when trying to find a path in three or more dimensions. 
%
%Navigation meshes are a specialised alternative to grid maps that have arisen
%from the game development community. A navigation mesh can be described as a collection
%of simple polygons that together describe the walkable surfaces of a virtual environment.
%Navigation meshes are popular in many games -- especially three-dimensional games -- because 
%they can be much smaller in size than grid maps are usually faster to search.
%On the other hand navigation meshes are complicated to construct (often requiring hand-tuning) 
%and they are non-trivial to repair of the environment changes.
%%A navigation mesh is the name given to a set of polygons that together
%describe the walkable surfaces in a virtual environment such as a computer game. This type
%of map is a popular alternative to grids because in many cases few polygons can be used 
%to effectively represent large portions of the environment. On the other hand meshes
%are complicated to construct, are non-trivial to repair if the environment changes and
% may require hand tuning.
%In other domains other specialised types of maps have also arisen: in the area of logistics
%and personal navigation road maps are ubiquitous while in robotics visibility graphs 
%are very popular. Both types of maps capture details of the environment that are
%most salient for the application at hand. In the case of road maps, it is critial
%to accurately describe the features of physical transportation networks but details about the rest
%of the environment can be discarded. Visibility graphs meanwhile only capture information
%about points in the environment that can be reached from one another by travelling in a straight
%line; everything else is secondary from the perspective of a mobile robot.
%
Many types of maps exist but for the purposes of pathfinding there is no single ``best'' choice.
Each map emphasises different features of the physical or virtual environment and each has distinct strengths 
and weaknesses -- characteristics which sometines only become apparent in a particular application 
or setting.

The second question which every pathfinding practitioner must address at the outset is how
to actually search the map for a path. The academic literature is rich with works that describe different 
techniques for solving such problems. Some approaches are inspired by the behaviour of real-life
insects; others are systematic techniques developed to automatically solve mazes. Some approaches
come with performance and efficiency guarantees while others do not. 

In this thesis we will build directly on the A{*} algorithm: a milestone technique from the area 
of Artificial Intelligence, A{*} promises to always return the shortest path if one exists. Moreover, 
it promises to do so using the least amount of possible effort. Our contribution is an
enhancement of A{*}

\begin{itemize}
\item Local search algorithms, some sometimes which mimic the behaviour of real-life insects. 
These approaches are often very fast and very memory efficient but they make no guarantees that 
they can actually find a path if one exists.
\item Blind search algorithms. Developed for solving mazes,  these ``classical'' approaches can
often (not always) guarantee that a path will be found and sometimes even that the path
will be optimal -- but they are typically not very efficient.
\item Informed search algorithms. This family of approaches will always find the optimal path if 
one exists. The A{*} algorithm is a notable example; it comes with strong guarantees about
efficiency and optimality and is regarded by many as the gold standard approach for pathfinding. 
\item Near-optimal pathfinding techniques. Such approaches are usually very fast, very memory
efficient and will find a path if one exists. Often such methods also come with guarantees
about the quality of the paths they return.
\end{itemize}


Just as there are many different types of maps there are equally many different types of approaches that have 
been developed to actually find a path between arbitary pairs of start and target locations.
%Road maps are another specialised representation technique widely employed in logistics and personal 
%navigation. They describe the features of physical transportation networks: 
%usually roads but also e.g. public transportation routes or rail lines. Road maps are simple to understand
%and information rich but they can be time consuming to accurately construct and they 
%are not very useful for certain types of pathfinding; e.g. computer games or robotics.

%Visibility graphs are special kinds of maps developed specifically for robotics applications.
%%Their construction requires only a description of the position and size of 
%%obstacles in an environment.
%%The resulting map simply 
%A visibility graph describes which pairs of points in an environment are mutually observable 
%and can thus can be reached from one another by travelling in a straight line. Like road maps, 
%visibility graphs are simple to understand and they can be quickly constructed. 
%The main drawback is that they can be quite dense and thus computationally inefficient 
%to search when trying to find a path. Another disadvantage is that visibility graphs often
%do not record information about the underlying terrain. Such details can be important; e.g.
%certain types of terrain are easier to traverse than others.

\section{Search Techniques}
\label{cha::intro::search}


\begin{itemize}
\item there are many application areas where the problem is interesting
\item there are many variations of the problem
\item solving the problem  involves (i) choosing a domain representation
and (ii) choosing a search strategy. both choices are important and they
can have a dramatic effect on how challenging (or not) pathfinding can be.
\item many domain representations exist; discuss a few of the most popular
including grid maps.
\item many solution strategies exist; discuss a few possibilities and
spell out that e.g. A* is regarded as the gold standard by industry
practitioners: cite gamasutra and mention that many practitioners use A*
-- if not directly then as a basic building block.
\item
Approaches attempting to do better than A* have typically focused on 
exploiting the domain 
\end{itemize}


particular constraints on what the path 

Such activities are central
to a great many application areas. For example: in robotics 

Such activities are
central to a great many applic


of central importance in so much of daily life the problem is studied



are of central importance in a great many application areas.

Since navigation
activities are central to a great many applicaton 

cartographic
representa

