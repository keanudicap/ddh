\chapter*{Abstract}
\addcontentsline{toc}{chapter}{Abstract}
\vspace{-1em}
Pathfinding (or navigating) from A to B is a common problem in Computer
Science with broad practical applications in areas as diverse as digital
entertainment, logistics and robotics. Pathfinding is made difficult when
many variations, or symmetries, of the same path exist.  Symmetry slows down
search by forcing otherwise performant algorithms to waste time considering
many equivalent states.  We speed things up by developing new online and
offline symmetry-breaking techniques that eliminate a large number of
symmetric states.  Our contributions are optimality preserving, memory
efficient and can have a dramatic positive effect on algorithmic performance.
They are especially well suited to speeding up pathfinding search on grid-maps 
of the type widely employed in computer games and robotics.
Moreover, our work is largely orthogonal with a wide range of
efficiency-improving techniques that have been previously described in the
academic literature.
\par 
We investigate a number of novel symmetry breaking approaches.
Rectangular Symmetry Reduction (RSR) identifies symmetric path segments during an 
offline pre-processing step.
This approach is optimal, requires very little overhead (usually a few seconds of
up-front time and a linear amount of memory) and it can improve
the performance of classical pathfinding algorithms such as A{*} by several factors.
Our second contribution, Jump Point
Search (JPS), significantly improves on the performance of RSR and currently 
represents the state of the art for pathfinding on grid-map domains.
In its online form JPS requires zero preprocessing, zero additional memory and
always finds the shortest path. Our experiments show that JPS can consistently
improve the performance of A{*} search by over one order of magnitude and more.
In its offline form JPS reformulates the search space to achieve even better
performance but requires an up-front investment of time. The algorithm has
zero memory overheads when applied to graphs that are stored as an adjacency list.
When applied to graphs stored as an adjacency matrix, the algorithm introduces
a linear-sized memory overhead. 
\par
In addition to RSR and JPS we study the related any-angle pathfinding problem.
Recently formulated but nevertheless well studied, this problem involves
finding a shortest path in a grid-map domain but asks that the path is not
constrained to the points of the grid.  Though a range of approaches have been
suggested there are no effective techniques, to date, that are both optimal
and online.  As a final contribution we give a first algorithm that can
provably compute solutions having both of these desirable characteristics.
