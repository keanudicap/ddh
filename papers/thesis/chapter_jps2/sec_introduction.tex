\section{Introduction}
\label{cha::jps2::introduction}
Grid-based pathfinding is a problem that often
appears in application areas such as robotics and computer games. 
%Many solution approaches exist; the most recent employ ideas from the
%literature of AI Search (e.g. ~\cite{pochter10,goldenberg10,yap11,urasKH13})
%and also Algorithmics (e.g.~\cite{storandt13,antsfeld12}). Such grid-based
Grids are popular with researchers because the encoding is
simple to understand and apply but the process of finding optimal paths
between arbitrary start-target pairs can be surprisingly challenging. At least
one reason for this difficulty can be attributed to the existence of 
symmetries: myriad in grid maps but less common in other domains such as
road networks. A path is considered symmetric when its individual steps 
(or actions) can be permuted in order to derive a new and equivalent path
that has identical cost.
In the presence of symmetry classical algorithms such as A* will waste much 
time looking at permutations of all shortest paths: from the start node to each expanded node.

Jump Point Search (JPS)~\cite{harabor11b} is a recent and very effective 
technique for for identifying and eliminating path symmetries on-the-fly. 
JPS can be described as the combination of A* search with two simple 
neighbour-pruning rules. When applied recursively these rules
can improve the performance of optimal grid-based pathfinding by an order of 
magnitude and more -- all without any pre-processing and without the 
introduction of any memory overheads.


The efficiency of JPS depends on being able to quickly scan many nodes
from the underlying grid map in order to identify jump points.
One the one hand such a procedure can typically save many unnecessary
node expansions. On the other hand the same operation proceeds in a 
step-by-step manner and it can scan the same node multiple times during 
a single search. 
Consider Table~\ref{table::jps2::bottleneck}, where we give a comparative
breakdown of how JPS and A{*} spend their time during search.
The results are obtained by running
a large set of standard instances on three realistic game benchmarks
that appeared in the 2012 Grid-based Path Planning Competition. Observe that
JPS spends $\sim$90\% of its time generating successors (cf. $\sim$40\% for
A{*}) instead of manipulating nodes on the open and closed lists -- i.e.
searching.

In this paper we propose a number of ideas that to improve the
performance of Jump Point Search. We focus on: (i) more efficient online 
symmetry breaking that reduces the time spent scanning the grid; 
(ii) more effective online pruning strategies that avoid expanding some jump
points; (iii) pre-computation strategies for breaking symmetries offline.
We evaluate our ideas on three realistic grid-based benchmarks and
find that our enhancements can improve the performance of Jump Point Search
by anywhere from several factors to over one order of magnitude. 
%In the case of 
%our first two contributions, we achieve this without compromising optimality, without
%preprocessing and without any memory overheads. In the case of our third
%contribution, some preprocessing and additional memory are introduced 
%in order to increase the performance of search.

% latex table generated in R 3.0.2 by xtable 1.7-1 package
% Thu Oct 10 11:58:29 2013
{\setlength{\tabcolsep}{0.5em}
\begin{table}[b!]
\vspace{-1em}
\small
\centering
\begin{tabular}{l|rr|rr}
  \hline
& \multicolumn{2}{c|}{A{*}} & \multicolumn{2}{c}{JPS} \\
& L.Time & G.Time & L.Time &  G.Time \\ \hline
D. Age: Origins & 58\% & 42\% & 14\% & 86\% \\ 
D. Age 2 & 58\% & 42\% & 14\% & 86\% \\ 
StarCraft & 61\% & 39\% & 11\% & 89\% \\ 
   \hline
\end{tabular}
\caption{\small A comparative breakdown of total search time on three realistic
video game benchmarks.
L.Time is the time spent manipulating nodes on open or closed.
G.Time is the time spent generating successors (i.e. accessing the grid).}
\label{table::bottleneck}
\end{table}
}


%\newpage
