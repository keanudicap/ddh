\section{Discussion}
\label{cha::jps2::discussion}
We study several techniques for improving Jump Point Search (JPS):
a recent symmetry-breaking algorithm that facilitates fast pathfinding 
on grid maps such as those commonly found in computer games.
%: a recent grid-based pathfinding technique which employs online symmetry breaking 
%to speed up search by up to one order of magnitude. 
Our first improvement allows us to detect jump points 
more efficiently by considering sets of nodes at one time (cf. one at a time). 
Our second improvement is a pre-processing strategy which computes and stores jump point
successors for every node on the map. Our third improvement is a 
pruning strategy that avoids many node expansion operations.
We speed up JPS by anywhere from several factors 
(in the purely online case) to over one order of magnitude (using preprocessing).
%To put this result into context consider that JPS itself is often one order faster
%than traditional online pathfinding techniques such as A{*}~\cite{harabor11b}.
Experiments on maps drawn from real computer games show that our 
work is competitive with and often faster than some very recent 
state-of-the-art grid-based pathfinding techniques.
