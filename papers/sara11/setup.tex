\section{Experimental Setup}
We evaluate the performance of RSR on three benchmarks taken from the freely
available pathfinding library Hierarchical Open Graph
(HOG)\footnote{\url{http://www.googlecode.com/p/hog2}}: {\textbf{Adaptive Depth}
is a set of 12 maps of size 100$\times$100 in which approximately $\frac{1}{3}$
of each map is divided into rectangular rooms of varying size and a large
open area interspersed with large randomly placed obstacles.} {\textbf{Baldur's
Gate} is a set of 120 maps taken from BioWare's popular roleplaying game
\emph{Baldur's Gate II: Shadows of Amn}.  Often appearing as a standard
benchmark in the literature \cite{bjornsson06,harabor10,pochter10} these maps
range in size from 50$\times$50 to 320$\times$320 and have a distinctive
45-degree orientation.} {\textbf{Rooms} is a set of 300 maps of size
256$\times$256 which are divided into symmetric rows of small rectangular areas
($7\times7$), connected by randomly placed entrances. This benchmark has
previously appeared in \cite{sturtevant09,pochter10,goldenberg10}.}
As discussed later, we also use a variant of each benchmark where every map is
scaled up by a factor of 3. In effect, our input data contains 864 maps in
total, with sizes up to $960\times960$.  
\par
Since our work is applicable to both 4
and 8 connected grid maps we used two copies each map: one in which diagonal
transitions are allowed and another in which they are not.  
For each map we generated 100 valid problem instances, checking that every
instance could be solved both with and without the use of diagonal transitions.
Our test machine had a 2.93GHz Intel Core 2 Duo processor, 4GB RAM and ran OSX
10.6.2.  Our implementation of A* is based on one provided in HOG, which we
adapted to facilitate our online node pruning enhancement.
