\begin{abstract}
TRANSIT~\cite{bast06} is a fast and optimal technique for computing shortest
path costs in road networks.
It is attractive for its usually modest memory
requirements and impressive running times.  In this paper we give a first
analysis of TRANSIT routing on a set of popular grid-based video-game benchmarks
taken from the AI pathfinding literature.
We show that in the presence of path symmetries, which
are inherent to most grids but normally not road networks, TRANSIT
is strongly and negatively impacted, both in terms of performance and memory
requirements.  We address this problem by
developing a new general symmetry breaking technique which adds
small random $\epsilon$-values to edges in the search graph, reducing the
size of the TRANSIT network by up to 4 times while preserving optimality.
Using our enhancements TRANSIT achieves \emph{up to four orders of magnitude}
speed improvement vs. A* search and uses in many cases only a small
($\leq$ 10MB) or  modest ($\leq$ 50MB) amount of memory.
We also compare TRANSIT with CPDs, a recent and very fast database-driven
pathfinding approach. We find the algorithms have complementary strengths but
also identify a class of problems for which TRANSIT is \emph{up to two orders of magnitude}
faster than CPDs using a comparable amount of memory.
 \end{abstract}
