\subsection{Algorithmic Description}

\input alg_jumpexpansion

We begin by making precise the concept of a jump point.  
%To do so, we use the notation $\vec{d}$ 
%to refer to one of the eight unit moves that can be taken from a node.  
%We write $y = x + k_{x}\vec{d}$ when node~$y$ can be reached 
%by taking $k_{x}$ unit moves from node~$x$ in direction $\vec{d}$.
%According to the pruning rules defined above, 
%if move~$\vec{d}$ from node~$p(x)$ to node~$x$ is diagonal, 
%then two straight moves are enabled from node~$x$ 
%(see Figure~\ref{fig:pruning}(c)); 
%these two moves are denoted by $\vec{d_1}$ and $\vec{d_2}$.  

\begin{definition}
\label{def:jump}
Node~$y$ is the \emph{jump point} from node~$x$, heading in direction~$\vec{d}$,
if $y$ minimizes the value $k$ such that $y = x + k \vec{d}$ and one of the
following conditions holds:
\begin{enumerate}
\item{Node~$y$ is the goal node.}
\item{Node~$y$ has at least one neighbour whose evaluation is forced according 
to Definition \ref{def:forced}.}
\item{$\vec{d}$ is a diagonal move and there exists a node $z = y +
k_i\vec{d_{i}}$ which lies $k_i \in \mathbf{N}$ steps in direction $\vec{d_i} \in
\{\vec{d_1},\vec{d_2}\}$ such that $z$ is a jump point from $y$ by condition 1 or condition~2.}
\end{enumerate}
\end{definition}

Figure \ref{fig:jumppoints}(b) shows an example of a jump point which is
identified by way of condition 3.
Here we start at $x$ and travel diagonally until encountering node $y$. 
From $y$, node~$z$ can be reached with $k_i = 2$ horizontal moves.
Thus $z$ is a jump point successor of $y$ (by condition~2) and this in turn
identifies $y$ as a jump point successor of $x$ 

%We identify $y$ as a jump point of $x$ by applying condition 3 of 
%Definition \ref{def:jump}. This in turn invokes condition 2 to identify $z$ 
%as a jump point from $y$. 


The process by which individual jump point successors are identified is given in
Algorithm \ref{alg:successors}.  We start with the pruned set of neighbours
immediately adjacent to the current node $x$ (line 2).  Then, instead of adding each
neighbour $n$ to the set of successors for $x$, we try to ``jump'' to a node
that is further away but which lies in the same relative direction to $x$ as $n$
(lines 3:5).  For example, if the edge $(x, n)$ constitutes a straight move
travelling \emph{right} from $x$, we look for a jump point among the nodes
immediately to the right of $x$.  If we find such a node, we add it to the set
of successors instead of $n$.  If we fail to find a jump point,
we add nothing.  The process continues until the set of neighbours is
exhausted and we return the set of successors for $x$ (line 6).

In order to identify individual jump point successors we will apply Algorithm
\ref{alg:jump}. It requires an initial node $x$, a direction of travel
$\vec{d}$, and the identities of 
the start node $s$ and the goal node $g$.
In rough overview, the algorithm attempts to establish whether $x$ has any 
jump point successors by stepping in the direction $\vec{d}$ (line 1) and testing
if the node $n$ at that location satisfies Definition \ref{def:jump}.
When this is the case, $n$ is designated a jump point and returned (lines 5, 7
and 11).
When $n$ is not a jump point the algorithm recurses and steps again in direction
$\vec{d}$ but this time $n$ is the new initial node (line 12).
The recursion terminates when an obstacle is encountered and no further
steps can be taken (line 3).
Note that before each diagonal step the algorithm must first 
fail to detect any straight jump points (lines 9:11). 
This check corresponds to the third condition of Definition \ref{def:jump} 
and is essential for preserving optimality.
