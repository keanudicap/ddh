\section{Introduction}
Single-agent path planning is a well known and extensively studied problem in computer science.
It has many applications such as logistics, robotics, and more recently, computer games. 
Despite the large amount of progress that has been made in this area, to date, very little work has focused specifically on addressing planning for diverse-size agents in heterogeneous terrain environments. 
\par \indent
The problem is interesting because such diversity introduces much additional complexity when solving path planning problems.
Modern real-time strategy or role-playing games (for example, EA's popular \emph{Red Alert 3} or Relic Entertainment's \emph{Company of Heroes}) often feature a wide array of units of differing shapes and abilities that must contend with navigating across environments with complex topographical features; many terrains, different elevations etc. 
Thus, a route which might be valid for an infantry-solider may not be valid for a heavily armoured tank. 
Likewise, a car and an off-road vehicle may be similar in size and shape but the paths preferred by each one could differ greatly. 
\par \indent
Unfortunately, the majority of current path planners, including recent hierarchical planners (\cite{botea04,sturtevant05,demyen07,geraerts07}), only perform well under certain ideal conditions. 
They assume, for example, that all agents are equally capable of reaching most areas on a given map and any terrain which is not traversable by one agent is not traversable by any. 
Further assumptions are often made about the size of each respective agent; a path computed for one is equally valid for any other because all agents are typically of uniform size. 
Such assumptions limit the applicability of these techniques to solving a very narrow set of problems: homogeneous agents in a homogeneous environment. 
We address the opposite case and show how efficient solutions can be calculated in situations where both the agent's size and terrain traversal capability are variable. 
\par \indent
First, we extend recent work on clearance-based pathfinding \cite{geraerts07} in order to measure obstacle distance at key locations in the environment (Section \ref{aha:computingclearance}) and show how this information helps agents plan appropriate paths (Section \ref{aha:aastar}). 
Next, we draw on a successful cluster-based abstraction technique \cite{botea04} in order to produce compact yet information rich search abstractions (Section \ref{aha:mapabstraction}). 
Finally, we introduce HAA*, a new clearance-based hierarchical path planner (Section \ref{aha:aha}) and provide a detailed empirical analysis of its performance on a wide range of problems involving multi-size agents in heterogeneous multi-terrain environments (Section \ref{aha:results}).
