\section{JPS+}
\label{sec:pre}
%JPS+ is a previously unpublished search strategy
%which we introduce for the 2012 Grid-based Pathfinding Competition.
Jumping from one point to another in the grid
avoids many unnecessary A* open list operations. However, identifying 
these jump points becomes the new bottleneck of the JPS algorithm. 
We therefore propose JPS+: an symmetry-breaking method which replaces
each adjacent neighbour of a node with a jump point that lies in the 
same relative direction.

Figure~\ref{fig:preproc}(a) illustrates our graph reformulation idea for a 
single node $x$. We simply search for a jump point in the direction
of each grid neighbour of $x$. In JPS, we discard all nodes along a failed
path. By comparison, JPS+ must store the last node along a failed path.
These \emph{sterile jump points} are required 
to guarantee optimality during search but are never added to the A* open list.
To see why they are necessary, consider Figure~\ref{fig:preproc}(b).
Here we reach $x$ from $p$ and try to jump from $x$ to $y$. 
Notice that each such jump may cross the goal or column of the target node
$t$. In JPS the diagonal recursion would have terminated at node $y'$, having detected
the goal $t$ along a non-failed straight jump.
JPS+ simulates this behaviour by explicitly inserting an intermediate node $y'$ 
at the point where the jump to $y$ crosses the column of $t$.
This condition is sufficient to preserve optimality during search. The proof
involves showing that JPS+ simulates exactly the behaviour of JPS. We omit it 
for brevity.

\begin{figure}[tb]
       \begin{center}
		   \includegraphics[width=0.95\columnwidth]
			{diagrams/preproc.png}
       \end{center}
	\vspace{-3pt}
       \caption{(a) A jump point is computed in place of each grid neighbour of node $x$.
		(b) When jumping from $x$ to $y$ we may cross the row or column of the target $t$ (here, both). 
To avoid jumping over $t$ we insert an intermediate successor $y'$ on the row or column of $t$ (whichever is closest to $x$).}

       \label{fig:preproc}
\end{figure}

%\textbf{JPS vs JPS+:} 
%Where JPS performs symmetry breaking online, JPS+ differs by breaking
%symmetries during an offline preprocessing step. 
%


