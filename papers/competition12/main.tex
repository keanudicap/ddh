\documentclass{article}

\begin{document}

\title{The Jump Point Search Entry for 2012 GPPC}
\author{Daniel Harabor \& Alban Grastien}
\maketitle

%\section{Introduction}

\begin{abstract}
  This document presents the Jump Point Search entry (WHAT NAME?) 
  for the 2012 Grid-based path planning competition.  

  Our entry is based on JPS \cite{harabor11b} 
  (presented Section~\ref{sec::jps}) 
  with an improvement based on a preprocessing 
  (presented Section~\ref{sec::pre}).  
\end{abstract}

\section{Jump Point Search}
\label{sec::jps}

Jump Point Search (JPS) is an improvement of the $A^*$ algorithm 
based on the reduction of the number of successors of the search nodes 
in order to ``jump'' over a number of nodes.  

Given a node $x$ reached from the parent node $p$, 
we remove the neighbour $n$ of $x$ 
from the set of successors of $x$ 
if one of the following rule holds: 
\begin{enumerate}
\item 
  there exists a path $\pi'$ from $p$ to $n$ 
  that is strictly shorter than path $\pi = [p,x,n]$; 
\item 
  there exists a path $\pi'$ from $p$ to $n$ 
  that is as long as $\pi$ but that includes a diagonal move earlier than $\pi$.  
\end{enumerate}
Observe that these rules can be tested by simply looking 
at the neighbours of the current node $x$ 
as we now illustrate.  

On Figure~\ref{fig::pruning}, 
the trimed neighbours are given in grey.  
The remaining neighbours are called the \emph{natural} successors of node $x$.  
The Figure~\ref{fig::forced} shows 
that the obstacles can modify the list of successors; 
we call \emph{forced} these additional successors.  

\begin{figure}[ht]
  INCLUDE FIGURES WITHOUT OBSTACLE
  \caption{Pruning rules in the absence of obstacle.}
  \label{fig::pruning}
\end{figure}

\begin{figure}[ht]
  INCLUDE FIGURES WITH OBSTACLE
  \caption{Pruning rules in the presence of obstacle.}
  \label{fig::forced}
\end{figure}

As illustrated in Figure~\ref{fig::pruning}, 
the number of successors of a node reached from a straight move 
is often one.  
JPS uses this property when exploring $x$ 
to immediately and recursively explore this single successor.  
If the recursive procedure leads to an obstacle (which is the most frequent situation), 
the path is ignored; 
if the procedure leads to a node $x$ with a forced neighbour 
(or to the goal), then JPS ``jumps'' from the current node to $x$.  

Consider now the diagonal move as presented in Figure~\ref{fig::pruning}.  
A diagonal move generally leads to three successors, 
two of which are straight moves.  
Therefore, when a diagonal move to node $x$ is considered by JPS, 
JPS first looks at the two straight moves from $x$ 
and, if they both lead to obstacles, recursively considers the next diagonal move.  
If one of the straight move cannot be ignored, 
then JPS jumps to the current node $x$.  
This is illustrated on Figure~\ref{fig::jump}.  

\begin{figure}[ht]
  INCLUDE FIGURE WITH A JUMP
  \caption{Jumping in JPS.}
  \label{fig::jump}
\end{figure}

By jumping, JPS is able to move quickly in the map 
without inserting nodes in the queue list of $A^*$.  
This is doubly beneficial as i) it reduces the number of operations 
and ii) it reduces the number of nodes in the queue, 
hence making each other operation cheaper.  
In addition, there is no memory overhead.  

In the original JPS \cite{harabor11b}, 
we assumed possible a diagonal move
even if one of the opposite squares was blocked 
(see the move from ??? to ??? on Figure~\ref{fig::pruning}).  
In the competition, it is impossible to ``cut'' the corners.  
Therefore, we redefined the forced moves to match the competition settings.  

\section{Preprocessing}
\label{sec::pre}

The main addition to JPS for GPPC is a (unpublished) preprocessing step 
which reduces considerably the runtime of JPS.  
We mentioned that JPS, compared to $A^*$, 
avoids many queue list operations 
by jumping from one jumping point to another in the grid.  
However, the identification of these jump points 
then becomes the bottleneck of the algorithm.  

Therefore, we propose a preprocessing of the map 
where we compute the jump point 
for each node and each direction.  
These off-line jump points are illustrated on Figure~\ref{fig::ojp}.  

\begin{figure}[ht]
  INCLUDE FIGURE WITH THE OFFLINE JUMP POINTS
  \caption{Off-line jump points in all eight directions for a given node.}
  \label{fig::ojp}
\end{figure}

The off-line jump points are computed similarly to the on-line jump points 
with the difference that a jump point is defined before each obstacle is hit.  
This addition is necessary 
because the goal state is unknown at preprocessing time.  

At runtime, the algorithm does not need to search for the next jump points 
and can simply find them in the precomputed table.  
Additional tests are however necessary to account for the goal.  

If the current move is a straight move 
and passes over the goal, 
then the jump is stopped at the goal.  

If the current move is a diagonal move 
and crosses the row or the column that contain the goal, 
then a test is performed to check 
whether the goal is visible from the point 
where the jump crosses the line.  
If the goal is indeed visible, 
then the jump is stopped at the crossing point; 
otherwise, the jump continues to the precomputed point.  

On the other hand, the off-line jump point are more numerous 
that the on-line jump point, 
since a jump point is defined even when an obstacle is hit.  
However, it is possible to identify these spurious jump points 
and to avoid inserting them in the queue list.  
DANIEL: YOU PROBABLY WANT TO WRITE SOMETHING ABOUT THIS PARAGRAPH.  
(HOW) DO YOU IDENTIFY THESE SPURIOUS JUMP POINT?  
ARE YOU SURE YOUR IDENTIFICATION PROCEDURE IS CORRECT?  

The benefit of this approach is to remove the computation of the jump point 
and replace it with a simple table lookup.  
The cost of the operation is the additional checks 
presented in the previous paragraphs, 
but it is very limited (much smaller than the gain).  
The preprocessed table induces a limited (linear) memory overhead 
(eight distances for each node in the graph).  
Finally, the preprocessing itself is very low, 
typically in the order of a few seconds.  

I DON'T KNOW WHETHER WE SHOULD MENTION POTENTIAL IMPROVEMENTS HERE.  
FOR INSTANCE, STORING THE TABLE ONLY FROM A JUMP POINT 
(I.E., SEARCH REQUIRED ONLY FOR THE START NODE); 
FILLING THE TABLE DURING THE SEARCH, 
UPDATING THE TABLE WHEN OBSTACLE ARE ADDED/REMOVED.  

\section{Conclusion}

???

\section*{Acknowledgments}
\section{Acknowledgements}
\label{sec::ack}
We thank Patrik Haslum for taking the time to read and comment on 
an early revision of this paper.
NICTA is funded by the Australian Government through the Department of
Communications and the Australian Research Council through the ICT Centre of
Excellence Program.


\bibliographystyle{alpha}
\bibliography{references}

\end{document}
