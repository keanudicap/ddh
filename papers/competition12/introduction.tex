\section{Introduction}
%Each grid path can be equivalently described as an ordered sequence of moves $D = [\vec{d_1}, \vec{d_2},\ldots,\vec{d_{k-1}}]$
%that begins at $s$, ends at $t$ and in which each move 
%where each $\vec{d_i}$ is a vector that is associated with one ....
\label{sec:introduction}

Symmetry in grid-based pathfinding manifests itself when we consider a
path, traditionally defined as an ordered sequence of nodes, as consisting
instead of an ordered sequence of vectors.
Under this alternative definition each vector $\vec{d}$ is associated with 
one of the eight allowable movement directions (up, down, left, right etc.)
and has a magnitude of either 1 or $\approx\sqrt 2$, depending on whether 
it represents a straight move or a diagonal move.
Such a formulation is useful because it allows us to see that 
many paths on a grid map, which share the same start and end node but which pass through 
different intermediate nodes, are often just symmetric permutations of each other; i.e.
they are identical save for the order in which the individual moves occur.

In the presence of symmetry, popular pathfinding techniques such as 
A*~\cite{hart68} unnecessarily consider permutations 
of all shortest paths: from the start node to 
\emph{every expanded node}.
Jump Point Search~\cite{harabor11b} is a simple but highly effective strategy that
identifies and eliminates many such symmetries. 
JPS is fast, optimal, requires zero preprocessing, has zero memory overhead
and appears orthogonal to recent pathfinding algorithms; for example~\cite{bjornsson06,pochter09,goldenberg10}. 
In this paper we adapt JPS to grid domains where ``corner-cutting'' movement is not allowed.
We also introduce JPS+pre, a new variant search strategy which involves
reformulating an input graph into a symmetry-reduced equivalent that can be 
searched much faster.
