\section{Introduction}
\label{sec:introduction}
Symmetry in grid-based pathfinding manifests itself when we consider a path as an ordered 
sequence of vectors $\vec{d}$ where each such vector
is associated with one of the eight allowable movement directions (up, down, left, right etc.)
and has a magnitude of either 1 or $\approx\sqrt 2$, depending on whether 
it represents a straight move or a diagonal move.
Let $\pi_{s,t} = \langle \vec{n_1}, \vec{n_2}, \ldots, \vec{n_k} \rangle$ be a path in a grid
map which begins at node $s$, ends at node $t$ and involves $k$ distinct moves, each represented
by one of the vectors $\vec{n_i}$.
Informally, we will say that $\pi_{s,t}$ is symmetric to another path $\pi'_{s,t}$ if we can 
derive $\pi'_{s,t}$ by simply permuting the sequence of vectors associated with $\pi_{s,t}$
\footnote{Both paths need to be valid for such an operation to succeed.}.

In the presence of symmetry popular pathfinding techniques such as 
A*~\cite{hart68} unnecessarily consider permutations 
of all shortest paths: from the start node to 
\emph{every expanded node}.
Jump Point Search~\cite{harabor11b} is a simple but highly effective strategy that
identifies and eliminates many such symmetries. 
JPS appears orthogonal to recent pathfinding algorithms 
(e.g.~\cite{bjornsson06,pochter09,goldenberg10}) and has few disadvantages:
it is fast, optimal, requires zero preprocessing and has zero memory overhead.
In this paper we adapt JPS to grid domains where ``corner-cutting'' movement is not allowed.
We also introduce JPS+pre, a new variant search strategy which involves
reformulating an input graph into a symmetry-reduced equivalent that can be 
searched much faster.
