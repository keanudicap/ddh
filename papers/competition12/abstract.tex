\begin{abstract}
%Jump Point Search (JPS) is a method for extracting shortest paths in grid maps
%and other regular domains commonly found in computer games, robotics and even
%warehouse logistics.  It speeds up search by pruning any nodes on a solution
%path which is symmetric to another path already under consideration.  In its
%basic form, JPS can be described as a recursive pruning operator that
%is applied online and which can be combined with any existing search algorith
%(for example A*).  In this paper we describe a pathfinding system that
%implements JPS and JPS+pre: a new derivative search strategy that performs
%symmetry breaking as an offline preprocessing step.  The result of the
%precomputation is a reformulated symmetry-reduced search space which can be
%searched much more efficiently than the original input graph.
%Both JPS and JPS+pre were submitted to the 2012 Grid-based Path Planning 
%Competition.
In this paper we describe a pathfinding system based on Jump Point Search (JPS): 
a recent and very successful search strategy that performs symmetry
breaking in order to speed up optimal pathfinding on grid maps.
Our first contribution is a simple modification of JPS which 
allows online symmetry breaking on maps where corner-cutting moves are not allowed.
Our second contribution is JPS+pre: a new derivative search strategy that performs
symmetry breaking as an offline preprocessing step.  The result of the
precomputation is a reformulated symmetry-reduced graph which can be
searched much more efficiently than the original input graph.
Both JPS and JPS+pre were submitted to the 2012 Grid-based Path Planning 
Competition.
\end{abstract}
