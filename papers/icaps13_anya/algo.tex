\section{Algorithm}
%The best online approaches for any-angle pathfinding consider 
%only the set of discrete points that define the grid.
%Since this strategy is insufficient to guarantee optimality 
%we will consider sets of discrete and intermediate points
%by taking them together as an interval.

\begin{defi}
\label{defi::interval}
A \emph{grid interval} $I$ is a set of contiguous pairwise visible points 
from any row of the grid.  Each interval is defined in terms 
of its endpoints $a$ and $b$. 
With the possible exception of $a$ and $b$, each interval
contains only intermediate and discrete non-corner points.
\end{defi}
Identifying intervals is simple: any row of the grid can be naturally divided
into maximally contiguous sets of traversable and non-traversable points.
Each traversable set forms a tentative interval which we split, repeatedly,
until the only corner points are found at $a$ or $b$.
\\
A significant advantage of Anya is that we construct intervals on-the-fly.
This allows us to start answering queries immediately and for any discrete
start-target pair. Similar algorithms (e.g. Continuous Dijkstra) require
a preprocessing step before any queries can be answered and then only
from a single fixed start point. 

\begin{defi}
\label{defi:searchnode}
A \emph{search node} $(I, r)$ is a tuple where $I$ is an interval and 
$r \not \in I$ is a \emph{root} point chosen s.t. each $p \in I$  
is visible from $r$. The identity of $r$ is always the most
recent turning point on a path from the start point $s$ to any $p \in I$.
To represent the start node, set $I = [s]$ and assume $r$ is a point located 
off the plane and visible only from $s$. The cost from $r$ to $s$ in this case
is zero.
\end{defi}
The successors of a search node $n$ are identified by computing intervals
over sets of traversable points from the same row of the grid as $n$ and from rows
immediately adjacent. We want to guarantee that each point in such
a set can be reached from the root of $n$ via a local path which is \emph{taut}.
Taut simply means that if we ``pull'' on the endpoints of the path we cannot
make it any shorter.

\begin{figure}[tb]
  \begin{center}
    \begin{tikzpicture}

\creategrid{6}{6}
\drawobstacle{0}{2}
\drawobstacle{4}{2}

\draw[dashed, line width=2] (1,1) -- (1,4);
\path[name path=p1] (1,1) -- +(6,4);
\path[name path=p2] (0,4) -- (6,4);
\path[name path=p3] (0,3) -- (6,3);

\draw[name intersections={of=p1 and p3,by=x2}];
\draw[dashed, line width=2] (1,1) -- (x2);
\draw[red,line width=2] (1,3) -- (x2);

\draw[name intersections={of=p1 and p2,by=x1}];
\draw[dashed, line width=2] (1,1) -- (x1);
\draw[red,line width=2] (1,4) -- (x1);

\end{tikzpicture}

  \end{center}
%  \caption{$(I, r)$ has three successors: $(I'_1, r)$ which is observable and $(I'_2, r')$ and $(I'_3, r')$ which are not. Notice that intervals of traversable points exist left of $I$ but the local path through $I$ to each such point is not taut.}
  \caption{$(I, r)$ has four successors: $(I'_1, r)$ and $(I'_2,r)$ which are observable and $(I'_3, r')$ and $(I'_4, r')$ which are not. Notice that intervals of traversable points exist left of $I$ but the local path through $I$ to each such point is not taut.}
\label{fig::successors}
\end{figure}

\begin{defi}
\label{defi::successors}
$(I', r')$ is a \emph{successor} of
$(I, r)$ if each $p' \in I'$ is reached
by a taut path $\langle r, p,  p' \rangle$ that begins
at $r$ and passes through some $p \in I$, 
and $r'$ is the last common point of these paths.  Additionally, 
the subpath $\langle p, p' \rangle$ must not intersect any 
interval $J \neq I'$.
\end{defi}
We begin with the set of traversable points that are 
visible from $r$ through $I$ and divide this set into $0 \leq k$
adjacent closed grid intervals.
We will say that each such interval is \emph{observable} and 
generate for each a corresponding successor node 
$(I', r')$ with root $r' = r$.
\\
Not all successors are observable.
For example, the taut path from $r$ can intersect 
$I$ at an endpoint $b$ which is also a corner point.
In this case we reach a set of traversable points that 
are either adjacent to $I$ or adjacent to the set of 
observable successors.
Each such point is visible from $p = b$ but not 
from $r$.  From this set of non-visible points we build a 
single half-open interval $I' = [a', b')$ s.t. $I'$ is open at the 
endpoint closest to $b$.
We will say $I'$ is \emph{non-observable} and generate a 
corresponding successor $(I', r')$ with root $r' = b$.  
Figure~\ref{fig::successors} shows examples of both
observable and non-observable successors.
%
%Notice that if $b$ is a non-corner endpoint of $I$ there 
%can also exist a set of traversable points that are not visible from $r$.
%In this case however the path from $r$ to each such point $p'$ is 
%not taut so we do not generate any successor.

To evaluate a search node $n = (I, r$) we will select a point $p \in I$ 
which has a minimum $f$-value with respect to a target point $t$.
We compute: 
\begin{equation}
\label{eq::f}
f(p) = g(r) + d(r, p) + h(p, t)
\end{equation}
where $g(r)$ is the length of the optimal path from the start point to 
the root, $d(r, p)$ is the straight line distance from $r$ to $p$
and $h(p, t)$ is an admissible heuristic function that lower-bounds the cost of reaching $t$ from~$p$.  

\begin{figure}[tb]
  \begin{center}
    \begin{tikzpicture}

\creategrid{6}{6}
%\drawobstacle{2}{3}

\draw[red,line width=2] (2,4) -- (4,4);

\coordinate (root) at (2,1);
\coordinate (g1)   at (3,5);
\coordinate (g2)   at (1,5);
\coordinate (g3)   at (5,5);

\foreach \g in {g1, g2, g3}
  \draw (root) -- (\g);

\path[name path=direct] (root) -- (g1);
\path[name path=interval] (2,4) -- (4,4);
\draw[name intersections={of=direct and interval,by=zsmile}];
\draw (zsmile) -- ++ (-0.2,0.2) + (-0.1,0.1) node {\textit{\u z}};

\draw[dashed,line width=2pt] (root) -- (g1);
\draw[dashed,line width=2pt] (root) -- (2,4) -- (g2);
\draw[dashed,line width=2pt] (root) -- (4,4) -- (g3);

\draw (g1)   + (-0.2,0.2) node {$g_1$};
\draw (g2)   + (-0.2,0.2) node {$g_2$};
\draw (g3)   + (-0.2,0.2) node {$g_3$};
\draw (root) + (-0.2,-0.2) node {$p$};

\coordinate (g4)   at (4,2);
\draw (g4)   + (+0.2,0.2) node {$g_4$};
\coordinate (g4mirror) at (4,6);
\draw (g4mirror) + (+0.2,-0.2) node {$g'_4$};
\path[name path=directtwo] (root) -- (g4mirror);
\draw[name intersections={of=directtwo and interval,by=zsmiletwo}];
\draw[dashed,line width=2pt] (root) -- (zsmiletwo) -- (g4);
\draw[dashed,line width=2pt] (root) -- (g4mirror);

\end{tikzpicture}

  \end{center}
  \caption{An illustration of Lemmas~\ref{lemm::choosingp}~and~\ref{lemm::mirror}.  The points $t_1$ and $t'_4$ correspond to the case 
where the line intersects the interval; 
$t_2$ and $t_3$ where it does not; 
$t_4$ where the mirrored target $t'_4$ must be used.
%The point $t_1$ corresponds 
%to Case (i); $t_2$ and $t_3$ correspond to Case (ii); $t_4$ and $t'_4$ correspond
%to Case (iii);
}
\label{fig::minf}
\end{figure}

Finding this point $p$ %that minimizes the value of $f$ 
is not trivial in general.  
For certain heuristics however, 
it is easy.  
Assume for instance 
that the heuristics is the straight-line distance 
between $p$ and $t$ (ignoring obstacles): $h(p,t) = d(p,t)$.  
The point $p$ can be identified 
thanks to the following two lemmas.  

\begin{lemm}\label{lemm::choosingp}
  Let $t$ and $r$ be two points 
  and let $I$ be an interval 
  such that the row of $I$ is between the rows of $t$ and $r$.  
  Then the point $p$ of $I$ 
  that minimizes the $f$ value 
  is the closest point of $I$ to the intersection of $(t,r)$ 
  with the row of $I$.  
\end{lemm}

\begin{proof}
  If $(t,r)$ intersects $I$ in $p_\mathrm{i}$, 
  then the minimum value of $d(r,p) + h(p,t)$ 
  is $d(r,t)$ which is obtained 
  by choosing $p = p_{\mathrm{i}}$ (by the triangle inequality).  
  Otherwise choose $p$ as the end point of $I$ 
  on the side where $(r,t)$ intersects the row of $I$.  
\end{proof}

If the precondition of Lemma~\ref{lemm::choosingp} is not satisfied, 
it is possible to replace $t$ by its mirrored version $t'$ through $I$ 
as their distance $d(p,t)$ is the same 
while $t'$ does satisfy the precondition.  

\begin{lemm}
\label{lemm::mirror}
  The mirrored point $t'$ of target $t$ 
  through interval $I$ is such that   
  $d(p,t) = d(p,t')$ for all $p \in I$.  
\end{lemm}

Lemma~\ref{lemm::mirror} is a trivial geometrical result.  
Both lemmas are illustrated on Figure~\ref{fig::minf}.  

%simply project a straight line $r
%\rightarrow t$ from the root point $r$ to the target point $t$ and choose the
%point $p \in I$ which lies at their intersection.  If $r \rightarrow t$ does not
%intersect $I$, choose $p$ as one of the two endpoints of $I$.

%\begin{lemm}
%\label{lemm::minf}
%For any tuple $(I, r)$ and target point $t$: 
%\textbf{if} the line $r \rightarrow t$ intersects $I$ at a point
%$p$ \textbf{then} $\forall p' \in I: f(p) \leq f(p')$;
%\textbf{else}
%%If $r \rightarrow t$ does not intersect $I$ then 
%$p$ is one of the two endpoints of $I$ and 
%$\forall p' \in I: f(p) \leq f(p')$.
%\end{lemm}
%\begin{proof}
%%By construction. 
%There are three cases to consider:
%%(i) $r \rightarrow t$ intersects $I$; (ii) $r \rightarrow t$ intersects
% the grid row containing $I$ but not $I$; (iii) $r \rightarrow t$ does 
% not intersect the grid row containing $I$. Each
% case is illustrated in Figure~\ref{fig::minf}.
% \\
% \textbf{Case (i):} $r \rightarrow t$ is a minimum bound on the cost of any
% path from $r$ through $p \in I$ to $t$. Thus $f(p)$ is minimum.
% \\
% \textbf{Case (ii):} $r \rightarrow t$ crosses the row of $I$ 
% either to the left of $I$, in which
% case $p = a$, or to the right of $I$, in which case $p = b$.
% In either case $f(p)$ is again a minimum bound on the cost of any
% path from $r$ to $t$ through $I$.
% \\
% \textbf{Case (iii):} $r \rightarrow t$ does not intersect $I$ 
% or even the grid row $R$ containing $I$. In this case we mirror
% $t$ through $R$ and derive a point $t'$ which we use in
% conjunction with Case (i) or Case (ii) instead of $t$.
% To see that $t$ and $t'$ are equivalent we identify 
% a point $x$ at the intersection of $t \rightarrow t'$ with $R$. 
% For any $p \in I$ we can now form two adjacent right-angle
% triangles: $p, x, t$ and $p, x, t'$. The opposide and adjacent 
% sides of both triangles have the same length and in both cases 
% meet at an angle of 90 degrees. We conclude the triangles are
% congruent and that 
% $\forall p \in I: p \rightarrow t \equiv p \rightarrow t'$.
% \end{proof}

The algorithm terminates when we expand a node $(I, r)$ s.t.
$t \in I$. By Lemma~\ref{lemm::choosingp} and \ref{lemm::mirror} 
the $f$-value of this
interval is guaranteed to be minimum with respect to $t$.
To extract a path we simply follow parent pointers until we
reach the node containing the start point.
The root points associated with the search nodes we encounter
are the turning points on the optimal any-angle path from 
$s$ to $t$. 

