\documentclass{article}
\usepackage{subfigure}
\usepackage{ecai2012/ecai2012}
\usepackage{times} 
\usepackage{latexsym}
\usepackage{url}
\usepackage{graphicx}
\usepackage{latexsym}
\usepackage{amsmath, amsthm, amssymb, amsthm}
\usepackage{algorithm}
\usepackage{multirow}
\usepackage{array}
\usepackage[noend]{algorithmic}

\newtheorem{theorem}{Theorem}
\newtheorem{lemma}{Lemma}
\newtheorem{proposition}{Proposition}
\newtheorem{corollary}{Corollary}
\newtheorem{definition}{Definition}

\renewcommand{\dblfloatpagefraction}{0.9}
\newcommand{\figref}[1]{\figurename~\ref{#1}}
\newenvironment{pth}{\langle}{\rangle}
\sloppy

\begin{document}

\title{New Jump Point Identification Techniques For Breaking Grid Symmetries}
\author{Paperid: XX}
%\author{
%Daniel Harabor 
%\and 
%Alban Grastien \\
%NICTA and The Australian National University \\
%Email: firstname.lastname@nicta.com.au
%}

\maketitle

\input abstract
\input introduction
\input relatedwork
\input notation
\input multijump
\input preprocess
\input setup
\input results
\input conclusion
%\input acknowledgements

\bibliographystyle{ecai2012/ecai2012}
\bibliography{references}
\end{document}
