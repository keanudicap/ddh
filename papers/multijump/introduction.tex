\section{Introduction}

Path planning in grid is a critical problem for many applications 
ranging from robotics to video-games, 
where the optimal path 
between a unit location and its target must be found quickly.  

The optimal preserving algorithms are based on $A^*$, 
which develops several paths in parallel 
and uses a heuristic to determine 
which path is the most promising 
and should be developed.  
We extend a recent approach called Jump Point Search (JPS) 
where symmetry is used to ignore numerous paths 
and to quickly move in the grid.  

JPS's main advantage is that much fewer nodes are expanded, 
but each expansion is more expensive.  
In our first contribution, we show 
how the jump points can be pre-processed 
so as to reduce the expansion time.  
This operation is not trivial, 
as the ``on-line'' computation of the jump points 
automatically eliminates a number of potential points 
that cannot be disregarded when the target is unknown.  
The pre-computation requires only limited time 
and the pre-computed table is linear 
which makes it quite attractive.  

% EOF
