\section{Memory Requirements}
\label{sec:memory}
In the most straightforward implementation, RSR requires a memory overhead which
is linear in the number of nodes in the search graph: i.e. $O(|V|)$.
In our implementation we stored the id of the parent rectangle for each node in 
the original grid. We also stored, for each identified rectangle, its height, width 
and the coordinates of its origin in the grid. 
Further overheads, such as storing the set of macro-edges for each perimeter
node, can be avoided by exploiting the simple geometric nature of empty
rectangles. 
During node expansion we calculate, on-the-fly, and in constant time, the exact position of 
the two nodes at the edge of each ``fan'' of neighbours from the opposite side of the perimeter. 
We can then simply travel along the perimeter, from one edge of
the fan to the other, and generate each non-pruned node we encounter.
Other neighbours, from the same side of the perimeter as the current node, or an orthogonal side, 
can be similarly identified on-the-fly and in constant time.
