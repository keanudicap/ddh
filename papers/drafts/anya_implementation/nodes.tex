\section{Search Nodes}
The traditional way to approach grid-based pathfinding is to map the grid to a
search graph. Often this means that every discrete point is a node and every edge
between two adjacent points is a transition.  As we have seen Anya uses a different 
model involving sets of points modeled as intervals.
In particular Anya models each search node as a tuple $(I, r)$. Here $I$ is once more a 
contiguous (i.e. no gaps) interval of traversable points drawn from a single row of the grid. Meanwhile
$r$ is a distinguished point called the \emph{root}. 

The identity of the root point in each case is always the most recent corner
point on a path that begins at the starting location and ends with some point
$p \in I$. 
In the case of the starting location $s$ we construct an tuple where
$I = [s]$ and $r = (\infty, \infty)$ is a special root point located
off the map and visible only from $s$.
Figure~\ref{fig::anya_example} (Left) shows seven different search nodes. Four 
of these nodes share the common root point $r = (0, 0)$. These nodes are defined
by the tuples $(I_2, (0, 0))$, $(I_3, (0, 0))$
$(I_4, (0, 0))$ and $(I_5, (0, 0))$. A fifth search node is formed by the
tuple comprising interval $I_6$ and the root point $r = (3, 3)$. 
The sixth search node is formed by the interval $I_1$ and the root point
$r = (0 + \epsilon, 0)$ where $\epsilon$ is a very small number. The seventh
search node is the starting point which we have defined above.

We now outline some special properties that need to exist between each pair 
of $I$ and $r$:
\begin{itemize}
\item{Every point $p \in I$ is visibile from $r$.}
\item{The root point $r$ and the interval $I$ are disjoint; i.e. $r \not\in I$.}
\item{The path from the root point $r$ to every point $p \in I$ is \emph{taut}.
As discussed in Example~\ref{ex::anya_example}, taut simply means that if we
``pull'' on the ends of the path (in this case from $r$ to any $p \in I$) we cannot 
make the path any shorter.}
\end{itemize}
We will revisit these properties in more detail later. For now it is
enough to be aware of them.

