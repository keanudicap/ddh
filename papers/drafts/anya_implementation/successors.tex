\section{Generating Successors}
\label{sec::successors}
The successors of a search node $(I, r)$ are found by computing intervals from
the set of traversable points located on the same row of the grid as $I$ and
on the rows immediately adjacent.  
Whenever we generate a successor $(I', r')$ we want to guarantee that:
\begin{enumerate}
\item The local path from $r$ to $r'$ and then to each point $p' \in I'$ is taut.
\item The local path from $r$ to $r'$ and then to each point $p' \in I'$  does not pass
through any interval except $I$ and $I'$.
\end{enumerate}

We will distinguish between two kinds of successors: observable and non-observable. 

\subsection{Observable Successors}
A search node $(I', r')$ is an observable successor of $(I, r)$ if every 
$p' \in I'$ is visible from $r$ through $I$.
We compute observable successors as follows:

\begin{itemize}
\item If $I$ and $r$ are located on the same row of the grid, we construct a
a new \emph{closed interval} $I'$ that begins at the endpoint of $I$ farthest from $r$.
We extend $I'$ away from $r$ using the procedure described in in Section~\ref{sec::intervals}.
The new successor node is formed by the tuple $I'$ and the root point $r' = r$.
Figure~\ref{fig::anya_example} (Left) shows an example.
\item If $I$ and $r$ are located on different rows of the grid, we project $r$ through $I$
and construct a new \emph{closed interval} on the next row of the grid (moving away from $r$).
The new interval, which we call $I'$, contains only points that (i) are visible from $r$ through $I$
and that (ii) that can be reached from $r$ via a local path that is taut. 
The new successor node is formed by the tuple $I'$ and the root point $r' = r$.
\end{itemize}

There are many ways to implement the projection of the root point $r$ though the interval $I$
and onto the next row of the grid.
One simple method is to compute the endpoints of the new interval $I'$ by extending the
vectors $\vec{ra}$ and $\vec{rb}$ until they intersect the next row of the grid. 
Here $r$ is once more the root point and $a$ and $b$ are the endpoints of $I$.
The result of this operation is one or more observable successors, depending on whether or
not the interval $I'$ needs to split due to internal corner points.
Figure~\ref{fig::observable} (Middle) shows an example. 
In some cases the extension of one 
vector may be impeded by obstacles. When this occurs we can simply follow the contour of
the obstacle up to the next row of the grid. Figure~\ref{fig::observable} (Right) shows
an example.  


\begin{figure}[tb]
\center
		   \includegraphics[width=\columnwidth]
			{images/observable.pdf}
	\vspace{-3pt}
\caption{\small
Examples of observable successors. In each case the current node is formed by the tuple $(I, r)$; the observable
successor is formed by the tuple $(I', r' = r)$. \textbf{(Left)} The root and interval of the current node $(I, r)$ are on
the same row. \textbf{(Middle)} Projecting $r$ through $I$ onto the next row of the grid. We extend the vectors $\vec{ra}$ and $\vec{rb}$
until they intersect the next row. The points of intersection become the endpoints of the closed interval $I'$. 
\textbf{(Right)} Alternative projection of $r$ through $I$ (due to obstacles).
}
\label{fig::observable}
\end{figure}

\subsection{Non-observable successors}


Two cases: (i) the root is on the same row as the interval $I$. (ii) the root is on a different row to the interval $I$.
