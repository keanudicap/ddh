\section{Constructing Intervals}
An interval $I$ is a contiguous (i.e. no gaps) set of pairwise visible points
chosen s.t. all points are drawn from a single row of the grid.
At the extremes of each interval are two endpoints: $a$ (the leftmost point in $I$) 
and $b$ (the rightmost point in $I$).
Each such interval has two properties:
(i) it cannot contain any non-traversable points; (ii)
it cannot contain any corner points except possibly $a$ or $b$.  
Constructing an interval is easy: 
\begin{enumerate}
\item Begin with an empty interval $I$.
\item Add to $I$ a single discrete or intermediate point $p$. We must ensure
that $p$ is both traversable and drawn from one of the rows of the grid
(i.e. it cannot be a point from the interior of a square cell).
\item Extend the interval $I$ until its size is maximum. To achieve this we
 simply scan the current row of the grid -- either to the left, to the right,
or in both directions. Scanning means we move horizontally and away from $p$, 
stepping from one discrete point of the row to the next until a termination 
condition is reached.
\item We stop scanning when (i) we reach a corner point or (ii) when the next
discrete point is not visible from the previous (refer once more to
the examples shown in Figure~\ref{fig::visibility}).
The point where we stop is the (left or right) endpoint of the 
interval. 
\item If necessary, we repeat the scanning process in the remaining direction
(either left or right).
The procedure terminates when both endpoints of the maximal interval have been 
identified.
\end{enumerate}

In many cases the search process will fix one of the endpoints of the candidate interval 
at hand (either $a$ or $b$). Constructing a maximal interval in such cases only requires 
scanning the grid in the direction of the remaining endpoint. For example if the left 
endpoint $a$ is fixed we need only scan to the right of $a$ in order to find the location 
of the right endpoint $b$.
It is important to notice that all of the intervals we construct will be either (i) 
closed at one endpoint or (ii) closed at both endpoints. 
