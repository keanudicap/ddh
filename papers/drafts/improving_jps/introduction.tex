\section{Introduction}
\label{sec::introduction}
Finding a shortest path between two points on a grid is a problem that often 
appears in application areas such as robotics and computer games. 
Many solution approaches exist; the most recent employ ideas from the
literature of AI Search (e.g. ~\cite{pochter10,goldenberg10,yap11,urasKH13})
and also Algorithmics (e.g.~\cite{storandt13,antsfeld12}). Such grid-based
pathfinding problems are popular with researchers because the formulation is
simple to understand and apply but the process of finding optimal paths
between arbitrary start-target pairs can be surprisingly challenging. At least
one reason for this difficulty can be attributed to the existence of path
symmetries; myriad in many standard grid-based benchmarks but seldom
found in other types of spatial networks such as road maps.  Symmetry in
pathfinding arises when the individual steps (or actions) that are represented
by the edges of a path can be permuted in order to derive a new equivalent
path that has identical cost.  Symmetries are problematic in pathfinding (and
in search settings more generally) because standard algorithms have difficulty
distinguishing between equivalent states.  In the presence of symmetry the A*
algorithm, for example, will waste much time looking at permutations of all
shortest paths: from the start node to each expanded node.

Jump Point Search (JPS)~\cite{harabor11b} is a recent and very effective 
technique for for identifying and eliminating path symmetries on-the-fly. 
JPS can be described as the combination of A* search with two simple 
neighbour-pruning rules. When applied recursively during search these
rules allow JPS to identify and avoid entirely many nodes that are 
equivalent and that would otherwise need to be generated
and expanded before the target is reached.
JPS can improve the performance of optimal grid-based pathfinding search 
by an order of magnitude and more. Moreover, JPS achieves this without 
any pre-processing and without the introduction of any memory overheads.

In this paper we propose a number of ideas that can further improve the
performance of Jump Point Search. Our first contribution is an enhanced
set of neighbour pruning rules which can further reduce the number of 
nodes expanded during search. Our second contribution is an enhancement
to JPS that allows it to consider sets of nodes in a single operation 
rather than considering node one at a time.
Our third contribution is a variation of JPS that breaks symmetries
during an offline pre-computation step.
We evaluate our ideas on a large range of grid-based benchmarks and
find that our enhancements can improve the performance of Jump Point Search
by (some impressive amount). In the case of our first
two contributions, we achieve this without compromising optimality, without
preprocessing and without any memory overheads. In the case of our third
contribution, some preprocessing and additional memory are introduced 
in order to increase the performance of search.

