\section{Preprocessing}
\label{sec::preprocessing}
We outline two different preprocessing approaches:

\begin{enumerate}
\item{We replace all the neighbours of every node
with jump points that lie in the same relative direction}.
\item{We construct a graph that contains only jump points
and insert the start and goal as necessary.}
\end{enumerate}

One straightfoward approach to speeding up JPS is to preprocess
the entire map and replace the neighbours of each node
with the first jump point that can be reached by travelling in the
direction of each neighbour. A sketch of this idea has been 
outlined already~\cite{harabor12}.
Preprocessing eliminates entirely the need to search for jump
points but it has two disadvantages (i) jump points need to be recomputed
if the map changes (some local repair seems enough) and (ii) it introduces 
a substantive memory overhead: we need to keep for each node 8 distinct identifiers. 
The overhead can be as large as $4\times8$ bytes per node.
