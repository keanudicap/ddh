\section{Conclusion}
\label{sec::conclusion}
We study several optimisation techniques applicable to Jump Point Search (JPS).
%: a recent grid-based pathfinding technique which employs online symmetry breaking 
%to speed up search by up to one order of magnitude. 
Our first improvement, block-based jumping, involves simple bitwise
operations that allow us to consider sets of nodes at one time (c.f. one
at a time) in order to detect jump points more efficiently. Our second
improvement is a pre-processing technique which computes and stores jump point
successors for every node on the map. Our third improvement is a simple node
pruning strategy that helps JPS avoid many unnecessary node expansion operations.

On a variety of realistic video-game domains we show that we can improve 
the performance of online JPS by several factors while maintainig
completeness, optimality and adding little-to-no extra memory. With the addition
of a pre-processing step and a small amount of extra memory (< 10MB in most cases) 
we can increase the performance of JPS by up to one order of magnitude.  This is
remarkable as JPS itself is often one order faster than traditional online 
pathfinding techniques such as A*.

We also compare our approach with SUB-S (which is guaranteed optimal) and SUB-TL 
(which is not). Both approaches are very recent and very fast pathfinding techniques from the 
literature~\cite{uras13}. We find that in most cases our enhanced variants of online 
JPS are competitive with and often faster than SUB-S. Meanwhile, we find that JPS with
preprocessing and node pruning has complementary strengths to SUB-TL; each method having
certain advantages and disadvantages for different classes of test instances.

There are several interestting directions for further work. One possibility involves
updating JPS' pre-computed database of jump points when the envrionment changes; another 
is to investigate stronger pruning rules for JPS, for example in the style of SUB-TL;
