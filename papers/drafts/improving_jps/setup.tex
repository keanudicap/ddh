\input table_expansion

\section{Experimental Setup}
\label{sec::setup}
We implemented Jump Point Search and a number of variants as described
in this this paper. All source code is written from scratch in C++.
%and is available from {\small \url{http://ddh.googlecode.com}}.
For all our algorithms we have applied a number of simple optimisations 
that help to effectively stride through memory and reduce the effect of 
cache misses.
This means that (i) we store the input map as a vector of bits, one bit 
for each node; (ii) we store the map twice, once in row major order and 
once in column major order; (iii) we pre-allocate memory in 256KB chunks.

We run experiments on a 2010 iMac running OSX 10.6.4. Our machine has 
a 2.93GHz Intel Core 2 Duo processor with 6MB of L2 cache and 4GB of RAM.
For test data we selected three benchmark problem sets taken from
real video games: 
\begin{itemize}
\item \textbf{Dragon Age: Origins}; 44,414 instances across 27 grids of 
sizes ranging 665 to 1.39M nodes.
\item \textbf{Dragon Age 2}; 68,150 instances across 67 grids of sizes 
ranging 1369 to 593K nodes.
\item \textbf{StarCraft}; 29,970 instances across 11 grids of sizes 
ranging 262K to 786K nodes.
\end{itemize}
Instances are sampled from across all possible problem lengths on each map.
All have appeared in the 2012 Grid-based Path Planning Competition. 

In keeping with the rules of the competition we disallow diagonal
corner-cutting transitions in all our experiments.  This change requires a
slight modification to the JPS algorithm.  Principally it means that we no
longer test for forced neighbours when jumping diagonally.
Jumping straight is also simplified. In the terminolgy of our block-based
jumping rules this change means we stop at location $B_N[b_S]$ rather than
$B_N[b_S-1]$.  These modifications do not affect the optimality or 
correctness of the algorithm. The argument is identical to the one
in~\cite{harabor11b}.
%All maps and test instances are availble from 
%~{\small \url{http://www.movingai.com/benchmarks/}}.
