
\begin{figure*}[tb]
 \begin{minipage}[t]{.47\linewidth}
  \begin{center}
    \begin{tikzpicture}

\creategrid{6}{6}

\drawobstacle{2}{5}
\drawobstacle{2}{4}
\drawobstacle{2}{3}
\drawobstacle{3}{3}
\drawobstacle{3}{2}
\drawobstacle{3}{1}
\drawobstacle{4}{3}
\drawobstacle{4}{2}
\drawobstacle{4}{1}

\coordinate (plm1)   at (5,0);
\coordinate (pl)     at (2,1);
\coordinate (plp1)   at (1,5);
\coordinate (redpathdirection) at (0,2.4); % Actually, further away
\coordinate (actualcorner) at (3,1); 
\path[name path=uppath] (pl) -- (plp1);
\path[name path=shortcut] (plm1) -- (redpathdirection);
\draw[name intersections={of=uppath and shortcut, by=plp'}];

\draw[red, line width=2pt]   (plm1) -- (plp');
\draw[blue, line width=2pt]  (plm1) -- (pl) -- (plp');
\draw[green, line width=2pt] (plp1) -- (plp');

\draw (plm1) +(0.4,0.2) node {$p_{l-1}$};
\draw (pl) +(-0.3,-0.2)   node {$p_{l}$};
\draw (plp1) +(-0.3,0.2) node {$p_{l+1}$};
\draw (plp') +(-0.3,0) node {$p'_l$};
%\draw (actualcorner) +(-0.3,0.3) node {$c$};

\end{tikzpicture}

  \end{center}
  \caption{Illustration of Lemma~\ref{lemma::corner}.
  A similar observation has been previously made for geodesic paths \cite{mitchell87}.  }
  \label{fig::corner}
\end{minipage} 
\hfill 
 \begin{minipage}[t]{0.47\linewidth}
  \begin{center}
    \begin{tikzpicture}

\creategrid{6}{6}

\drawobstacle{3}{3}

\coordinate (n1) at (2,0);
\coordinate (n2) at (3,4);

\path[name path=direct] (n1) -- (n2);
\path[name path=row 1] (0,1) -- (6,1);
\path[name path=row 2] (0,2) -- (6,2);
\path[name path=row 3] (0,3) -- (6,3);
\draw[name intersections={of=direct and row 1,by=y1}];
\draw[name intersections={of=direct and row 2,by=y2}];
\draw[name intersections={of=direct and row 3,by=y3}];

\draw[red] (n1) -- (n2);

\draw (n1) ++(-0.4,0.4) node {$n_1$} + (0.1,-0.1) -- (n1);
\draw (n2) ++(-0.4,0.4) node {$n_2$} + (0.1,-0.1) -- (n2);
\draw (y1) ++(-0.4,0.4) node {$y_1$} + (0.1,-0.1) -- (y1);
\draw (y2) ++(-0.4,0.4) node {$y_2$} + (0.1,-0.1) -- (y2);
\draw (y3) ++(-0.4,0.4) node {$y_3$} + (0.1,-0.1) -- (y3);

\end{tikzpicture}

  \end{center}
  \caption{When pathfinding from $n_1$ to $n_2$ online algorithms such as Theta* only consider the discrete points of 
the grid and never any points $y_i$. 
%This strategy is not monotonic and does not guarantee optimality.
}
  \label{fig::ex1}
  \end{minipage}
\end{figure*}

\section{Preliminaries}
A \emph{grid} is a planar subdivision consisting of $W \times H$ square cells.
%It is common in such settings to simply mark each cell as traversable or
%non-traversable and treat each one as a unique location during pathfinding search.
%This simple model has an unfortunate sideeffect: any computed path is forced
%to pass through the centre point of any cell it intersects.
%However we would like to compute optimal paths that can intersect traversable cells 
%at any point and which may not pass through the centre of any cell.
We model each cell as an open set of \emph{interior} points which are 
either all \emph{traversable} or all \emph{non-traversable}.
%\footnote{Such cells are sometimes called obstacles.}.
We refer to the set of locations that define the perimeter of each cell as the 
\emph{discrete} points of the grid. Each such point $p$ has a unique coordinate $(x, y)$
where $x \in \{0,\dots,W\}$ and $y \in \{0,\dots,H\}$.
% refers to an explicit vertex 
Edges in the grid can be interpreted as open intervals of \emph{intermediate}
points; each one representing a transition between a pair of discrete points. 
\par
A discrete (resp. intermediate) point is traversable if it is adjacent to 
at least one traversable cell. Otherwise it is non-traversable.
%We now define some terminology for discussing discrete points:
A discrete point which is common to exactly four adjacent cells is called an \emph{intersection}.
Any intersection where three of the adjacent cells are traversable and one is not
is called a \emph{corner}.
We will say that two points are \emph{visible} from one another if there exists a 
straight-line path connecting them which does not pass through any non-traversable point.
%A \emph{corner} is an intersection 
%such that only one of its four squares is non-traversable.  
%Intersection $i'$ is \emph{visible} from intersection $i$ 
%iff the segment $(i,i')$ crosses only traversable tiles 
%or borders non-traversable tiles and traversable tiles.  
%Without loss of generality, 
%we assume that if the segment $(i,i')$ crosses a corner 
%(besides $i$ and $i'$), 
%then intersection $i'$ is not visible from intersection $i$.  

An \emph{any-angle path} $\pi$ as a sequence of discrete points 
$\langle p_1,\dots,p_k \rangle$ where each $p_{i}$ is visible from both $p_{i-1}$
and $p_{i+1}$.
The \emph{length} of path $\pi$ 
is the cumulative straight-line distance between every successive
pair of discrete points along the path, 
i.e., $d(p_1,p_2) + d(p_2,p_3) + \dots + d(p_{k-1},p_k)$, 
where $d((x,y), (x',y'))= \sqrt{(x-x')^2 + (y-y')^2}$ 
is a uniform Euclidean distance metric.
We will say $p_i \in \pi$ is a \emph{turning point} if the segments
$(p_{i-1}, p_i)$ and $(p_i, p_{i+1})$ form an angle.
\\
\begin{lemm}
\label{lemma::corner}
  Given two discrete points $p$ and $p'$, 
  any turning point in the optimal any-angle path between $p$ and $p'$ 
  is also a corner point.
\end{lemm}

\begin{proof}
{
  Assume an optimal any-angle path $\pi = \langle p_1,\dots,p_k \rangle$ 
  that includes a turning point $p_l$ ($l \not\in \{1,k\}$) 
  which is not a corner.  
  We will prove that $\pi$ is suboptimal 
  which, by contradiction, will prove the lemma.  
  If $p_{l+1}$ is visible from $p_{l-1}$, 
  then  $\pi \setminus p_l$ is a path 
  which is strictly shorter than $\pi$.  
  Hence $\pi$ is suboptimal.  
  If $p_{l+1}$ is not visible from $p_{l-1}$, 
  then let $p'_{l}$ be a point from the segment $\langle p_l,p_{l+1} \rangle$ 
  that (i) is visible from $p_{l-1}$ 
  and (ii) is different from $p_l$.  
  Because $p_l$ is not a corner, 
  then such a $p'_l$ exists.  
  Moreover, the subpath $\langle p_{l-1}, p'_{l}, p_{l+1} \rangle$ is 
  strictly shorter than $\langle p_{l-1}, p_{l}, p_{l+1}\rangle$. 
  Hence $\pi$ is suboptimal.
  This case is is illustrated in Figure~\ref{fig::corner}  
  where the point $p'_l$ is chosen as close as possible from the corner $c$.  
}
\end{proof}
% EOF
