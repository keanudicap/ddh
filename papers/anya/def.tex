\section{Preliminaries}
A \emph{grid} $G$ is a $W \times H$ matrix of tiles in which each tile is either
\emph{traversable} or \emph{non-traversable}\footnote{Such tiles are sometimes
called obstacles.}.
%It is common
%in such settings for each tile to represent a unique location in the grid and to
%assume there is an implicit vertex located at its centre~\cite{yap02}.  In this
%paper we follow~\cite{nash07} and adopt a slightly different model: 
Each location in the grid refers to an explicit point $p = (x, y)$
($x \in \{0,\dots,W\}$ and $y \in \{0,\dots,H\}$)
% refers to an explicit vertex 
which is at the intersection of two or more grid edges. 
An edge between two neighbouring points $(p_i, p_j)$ 
is considered traversable if at least one of the two tiles adjacent to it is also 
traversable.
We now give some terminology for referring to points:
\begin{itemize}
\item An \emph{intersection} is a point common to four adjacent tiles.
\item A \emph{corner} is an intersection where three of the adjacent tiles are traversable
and one is not.
\item Two points are \emph{visibile} from one another if there exists a 
straight-line path connecting them which does not pass through the interior of any
 non-traversable tiles.
\end{itemize}

%A \emph{corner} is an intersection 
%such that only one of its four squares is non-traversable.  
%Intersection $i'$ is \emph{visible} from intersection $i$ 
%iff the segment $(i,i')$ crosses only traversable tiles 
%or borders non-traversable tiles and traversable tiles.  
%Without loss of generality, 
%we assume that if the segment $(i,i')$ crosses a corner 
%(besides $i$ and $i'$), 
%then intersection $i'$ is not visible from intersection $i$.  

An \emph{any-angle path} $\pi$ is a sequence of points 
$\langle p_1,\dots,p_k \rangle$ where each $p_{i}$ is visible from both $p_{i-1}$
and $p_{i+1}$.
The \emph{length} of path $\pi$ 
is the cumulative straight-line distance between every successive
pair of points along the path, 
i.e., $d(p_1,p_2) + d(p_2,p_3) + \dots + d(p_{k-1},p_k)$, 
where $d((x,y), (x',y'))= \sqrt{(x-x')^2 + (y-y')^2}$ 
is the (uniform) Euclidean distance between the two points.  
We will say $p_i \in \pi$ is a \emph{turning point} if the segments
$(p_{i-1}, p_i)$ and $(p_i, p_{i+1})$ form an angle.
\\
\begin{lemm}
\label{lemma::corner}
  Given two points $p$ and $p'$, 
  any turning point in the optimal any-angle path between $p$ and $p'$ 
  is also a corner point.
\end{lemm}

\begin{proof}
{
  Assume an optimal any-angle path $\pi = (p_1,\dots,p_k)$ 
  that includes a turning point $p_l$ ($l \not\in \{0,k\}$) 
  which is not a corner.  
  %We will prove that $\pi$ is suboptimal 
  %which, by contradiction, will prove the lemma.  
  If $p_{l+1}$ is visible from $p_{l-1}$, 
  then  $\pi \setminus p_l$ is a path 
  which is strictly shorter than $\pi$.  
  Hence $\pi$ is suboptimal.  
  If $p_{l+1}$ is not visible from $p_{l-1}$, 
  then let $p'_{l}$ be a point from the segment $(p_l,p_{l+1})$ 
  that (i) is visible from $p_{l-1}$ 
  and (ii) is different from $p_l$.  
  Because $p_l$ is not a corner, 
  then such a $p'_l$ exists.  
  Moreover, the subpath $\langle p_{l-1}, p'_{l}, p_{l+1} \rangle$ is 
  strictly shorter than $\langle p_{l-1}, p_{l}, p_{l+1}\rangle$. 
  Hence $\pi$ is suboptimal.
  This case is is illustrated in Figure~\ref{fig::corner}  
  where the point $p'_l$ is chosen as close as possible from the corner $c$.  
}
\end{proof}


\begin{figure}[tb]
  \begin{center}
    \begin{tikzpicture}

\creategrid{6}{6}

\drawobstacle{2}{5}
\drawobstacle{2}{4}
\drawobstacle{2}{3}
\drawobstacle{3}{3}
\drawobstacle{3}{2}
\drawobstacle{3}{1}
\drawobstacle{4}{3}
\drawobstacle{4}{2}
\drawobstacle{4}{1}

\coordinate (plm1)   at (5,0);
\coordinate (pl)     at (2,1);
\coordinate (plp1)   at (1,5);
\coordinate (redpathdirection) at (0,2.4); % Actually, further away
\coordinate (actualcorner) at (3,1); 
\path[name path=uppath] (pl) -- (plp1);
\path[name path=shortcut] (plm1) -- (redpathdirection);
\draw[name intersections={of=uppath and shortcut, by=plp'}];

\draw[red, line width=2pt]   (plm1) -- (plp');
\draw[blue, line width=2pt]  (plm1) -- (pl) -- (plp');
\draw[green, line width=2pt] (plp1) -- (plp');

\draw (plm1) +(0.4,0.2) node {$p_{l-1}$};
\draw (pl) +(-0.3,-0.2)   node {$p_{l}$};
\draw (plp1) +(-0.3,0.2) node {$p_{l+1}$};
\draw (plp') +(-0.3,0) node {$p'_l$};
%\draw (actualcorner) +(-0.3,0.3) node {$c$};

\end{tikzpicture}

  \end{center}
  \caption{Illustration of Lemma~\ref{lemma::corner}.
  A similar observation has been previously made for geodesic paths \cite{mitchell87}.  }
  \label{fig::corner}
\end{figure}


% EOF
