\section{Principle Of Anya}
Consider the example presented on Figure~\ref{fig::ex1} 
where the path requires to travel from $(2,0)$ to $(3,4)$.  

%\begin{figure}[ht]
%  \begin{center}
%    \includegraphics[scale=0.3]{images/ex1}
%  \end{center}
%  \caption{Example of an any-angle path}
%  \label{fig::ex1}
%\end{figure}

\begin{figure}[ht]
  \begin{center}
    \begin{tikzpicture}

\creategrid{6}{6}

\drawobstacle{3}{3}

\coordinate (n1) at (2,0);
\coordinate (n2) at (3,4);

\path[name path=direct] (n1) -- (n2);
\path[name path=row 1] (0,1) -- (6,1);
\path[name path=row 2] (0,2) -- (6,2);
\path[name path=row 3] (0,3) -- (6,3);
\draw[name intersections={of=direct and row 1,by=y1}];
\draw[name intersections={of=direct and row 2,by=y2}];
\draw[name intersections={of=direct and row 3,by=y3}];

\draw[red] (n1) -- (n2);

\draw (n1) ++(-0.4,0.4) node {$n_1$} + (0.1,-0.1) -- (n1);
\draw (n2) ++(-0.4,0.4) node {$n_2$} + (0.1,-0.1) -- (n2);
\draw (y1) ++(-0.4,0.4) node {$y_1$} + (0.1,-0.1) -- (y1);
\draw (y2) ++(-0.4,0.4) node {$y_2$} + (0.1,-0.1) -- (y2);
\draw (y3) ++(-0.4,0.4) node {$y_3$} + (0.1,-0.1) -- (y3);

\end{tikzpicture}

  \end{center}
  \caption{Example of an any-angle path}
  \label{fig::ex1}
\end{figure}

To find a path between node $n_1 = (2,0)$ and node $n_2 = (3,4)$, 
the Theta$*$ any angle pathfinding algorithm \cite{nash10}
successively explores nodes $(2,1)$ to $(2,4)$ and $(3,0)$ to $(3,3)$ 
(thereafter called the intermediate nodes) until reaching $n_2$.  
Notice that the actual shortest path between $n_1$ and $n_2$ 
does not go through these intermediate nodes, 
and Theta* accounts to this fact by ``pulling'' the string 
when a new node is explored.  
Presented differently, when $n_2$ is generated, 
the $g$ value is computed for the direct path $(n_1,n_2)$ 
and not based on a path that would go through e.g. $(2,1)$ or $(3,0)$.  

The problem with this algorithm 
is however that the $A^*$ $f$ value of the intermediate nodes 
is based on the assumption that the path would go through these nodes.  
For instance, assuming $n_1$ is the start node 
and $n_2$ is the target node, 
although node $n' = (2,1)$ needs to be expanded before $n_2$, 
the $f$ value of $n'$ will be $1 + \sqrt{10} \simeq 4.16$ 
while the $f$ value of $n_2$ is $\sqrt{17} \simeq 4.12$.  
Consequently, the $f$ value computed 
for the intermediate nodes may be higher 
than the actual optimal path length, 
although the intermediate nodes must be expanded 
before reaching the goal.  
In order to avoid this mistake, 
we would need to explore not only the intermediate nodes, 
but also the $y_i$ nodes presented on the figure.  
However, the number of such $y_i$ nodes is a priori very large, 
since the search engine does not know a priori 
where the next interesting node (here $n_2$) will be; 
for instance, $y_1$ has here position $(2 + \frac{1}{4},1)$, 
but if $n_2$ was in $(3,3)$, 
then $y_1$ would be at position $(2 + \frac{1}{3},1)$.  
Assuming a map of size $W\times H$, 
the relevant nodes could have any position 
that is a fraction $\frac{w}{h}$ 
where $w$ (resp. $h$) is an integer 
in $\{0,\dots,W\}$ (resp. $\{0,\dots,H\}$).  

We therefore propose to consider all the potential $y$ nodes together.  
From $n_1$, all the nodes that can be reached with a one step North 
(and potentially any fraction of West/East steps) will be considered.  
We call ``interval'' this set of nodes that are considered together.  

How should we then compute the successors of an interval?  
Look at Figure~\ref{fig::succ1} (left) 
where a North step is taken from the interval $[(1,3)-(4,3)]$
reached from node $(1,1)$.  
The (single) successor of the interval 
is simply computed by projecting the visibility cone 
(represented with dashed lines)
from $(1,1)$ to the next row.  
This leads to the interval $[(1,4)-(5+\frac{1}{2},4)]$.  
Obverse that, as a consequence, the parent of the interval 
(here $(1,1)$) must be kept together with the interval.  

\begin{figure}[ht]
  \begin{minipage}{0.5\linewidth}
  \begin{center}
    \begin{tikzpicture}

\creategrid{6}{6}
\drawobstacle{0}{2}
\drawobstacle{4}{2}

\draw[dashed, line width=2] (1,1) -- (1,4);
\path[name path=p1] (1,1) -- +(6,4);
\path[name path=p2] (0,4) -- (6,4);
\path[name path=p3] (0,3) -- (6,3);

\draw[name intersections={of=p1 and p3,by=x2}];
\draw[dashed, line width=2] (1,1) -- (x2);
\draw[red,line width=2] (1,3) -- (x2);

\draw[name intersections={of=p1 and p2,by=x1}];
\draw[dashed, line width=2] (1,1) -- (x1);
\draw[red,line width=2] (1,4) -- (x1);

\end{tikzpicture}

  \end{center}
  \end{minipage}
  \begin{minipage}{0.5\linewidth}
  \begin{center}
    \begin{tikzpicture}

\creategrid{6}{6}
\drawobstacle{0}{2}
\drawobstacle{2}{3}

\draw[dashed, line width=2] (1,1) -- (1,4);
\path[name path=p1] (1,1) -- +(6,4);
\path[name path=p2] (0,4) -- (6,4);
\path[name path=p3] (0,3) -- (6,3);
\path[name path=p4] (1,1) -- +(4,4);

\draw[name intersections={of=p1 and p3,by=x2}];
\draw[dashed, line width=2] (1,1) -- (x2);
\draw[red,line width=2] (1,3) -- (x2);

\draw[name intersections={of=p1 and p2,by=x1}];
\draw[dashed, line width=2] (1,1) -- (x1);
\draw[red,line width=2] (1,4) -- (2,4);
\draw[dashed, line width=2] (1,1) -- (2,4);
\draw[name intersections={of=p4 and p2,by=x3}];
\draw[dashed, line width=2] (1,1) -- (x3);
\draw[red,line width=2] (x3) -- (x1);

\end{tikzpicture}

  \end{center}
  \end{minipage}
  \caption{Computing the successors of an interval}
  \label{fig::succ1}
\end{figure}

There are more complicated situations, 
as presented in Figure~\ref{fig::succ1} (right).  
Here, we can see that an obstacle splits the interval 
in two separate intervals.  
We could define disjunct intervals, 
but it seems more natural to consider them independently.  

Furthermore, the path might turn at node $(3,3)$
(e.g., to reach node $(3,4)$
which is not directly accessible from $(1,1)$).  
The corner $(3,3)$ is therefore a successor of node $(1,1)$ 
and intervals should be built from this node.  

%Let us present more precisely how the successors are splitted.  
%From the interval $I = [(1,3)-(4,3)]$, 
%we assume that an epsilon step 
%(i.e., an infinitely small step North) is taken.  
%This step reveals how the interval will be splitted.  
%Therefore, we generate interval 
%$[(1,3+\varepsilon)-(4+\frac{3}{2}\varepsilon,3+\varepsilon)]$; 
%we then remove the parts that belong to the obstacle, 
%which produces the two intervals 
%$[(1,3+\varepsilon)-(2,3+\varepsilon)]$ 
%$[(3,3+\varepsilon)-(4+\frac{3}{2}\varepsilon,3+\varepsilon)]$; 
%finally, we split the part that is not visible from $(1,1)$, 
%which leads to three intervals: 
%$[(1,3+\varepsilon)-(2,3+\varepsilon)]$, 
%$[(3,3+\varepsilon)-(3+\varepsilon,3+\varepsilon)]$, 
%and
%$](3+\varepsilon,3+\varepsilon)-
%(4+\frac{3}{2}\varepsilon,3+\varepsilon)]$. 
%If we project back on row 3, 
%we obtain three intervals: 
%$[(1,3)-(2,3)]$, 
%$[(3,3)-(3,3)]$, 
%and
%$](3,3)-(4,3)]$. 
%We say that these intervals are ``consistent''%
%\footnote{We need a better word here.} 
%with the obstacles.  
%The first and last intervals can be delt with similarly 
%to the example in Figure~\ref{fig::succ1} (left).  
%The second interval, being a single point, 
%requires to generate new intervals.  

The remaining question is what $f$ value 
should be associated with these intervals.  
For $A^*$ to return the optimal solution, 
we need $f$ to be an underestimate of the actual value 
(except for the final node).  
The interval has the following semantics: 
the path that is currently being constructed 
should start from $s$, cross the parent $P$, 
then cross the interval, and finally reached the goal.  
The shortest path that belongs to this set 
has length $g^*(P) + min_{x \in I}(d(P,x) + h^*(x))$.  
The value $g^*(P)$ should be known at this stage 
(this is one advantage of $A^*$).  
However, the minimum factor is unknown, 
but it can be lower-bounded by $min_{x \in I}(d(P,x) + h(x))$.  
This formula is actually pretty simple to solve, as we now demonstrate.  

Let $z_1$ and $z_2$ be the two extreme points of the interval.  
Draw a direct line between the goal and the parent $P$ of the interval, 
as shown in Figure~\ref{fig::fvalue}.  
If the line crosses the interval in \textit{\u z}, then the value is h(P).  If the line passes on the left, 
then the value is $d(P,z_1)+h(z_1)$; 
otherwise, it is $d(P,z_2)+h(z_2)$.  
If the goal is between the parent and the interval, 
as is the case with $g_4$, 
then one needs to consider the mirrored version of $g_4$ 
(here $g'_4$).  

\begin{figure}[ht]
  \begin{center}
%    \includegraphics[scale=0.3]{images/fvalue}
    \begin{tikzpicture}

\creategrid{6}{6}
%\drawobstacle{2}{3}

\draw[red,line width=2] (2,4) -- (4,4);

\coordinate (root) at (2,1);
\coordinate (g1)   at (3,5);
\coordinate (g2)   at (1,5);
\coordinate (g3)   at (5,5);

\foreach \g in {g1, g2, g3}
  \draw (root) -- (\g);

\path[name path=direct] (root) -- (g1);
\path[name path=interval] (2,4) -- (4,4);
\draw[name intersections={of=direct and interval,by=zsmile}];
\draw (zsmile) -- ++ (-0.2,0.2) + (-0.1,0.1) node {\textit{\u z}};

\draw[dashed,line width=2pt] (root) -- (g1);
\draw[dashed,line width=2pt] (root) -- (2,4) -- (g2);
\draw[dashed,line width=2pt] (root) -- (4,4) -- (g3);

\draw (g1)   + (-0.2,0.2) node {$g_1$};
\draw (g2)   + (-0.2,0.2) node {$g_2$};
\draw (g3)   + (-0.2,0.2) node {$g_3$};
\draw (root) + (-0.2,-0.2) node {$p$};

\coordinate (g4)   at (4,2);
\draw (g4)   + (+0.2,0.2) node {$g_4$};
\coordinate (g4mirror) at (4,6);
\draw (g4mirror) + (+0.2,-0.2) node {$g'_4$};
\path[name path=directtwo] (root) -- (g4mirror);
\draw[name intersections={of=directtwo and interval,by=zsmiletwo}];
\draw[dashed,line width=2pt] (root) -- (zsmiletwo) -- (g4);
\draw[dashed,line width=2pt] (root) -- (g4mirror);

\end{tikzpicture}

  \end{center}
  \caption{Computing the minimum distance 
    to go from $p$ through the interval to the goal; 
    if the goal is $g_1$, the direct distance between $p$ and $g_1$; 
    if the goal is $g_2$ or $g_3$, 
    the shortest distance is the distance represented by the dashed lines.}
  \label{fig::fvalue}
\end{figure}
