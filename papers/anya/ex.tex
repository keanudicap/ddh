\section{Principle Of Anya}
Consider the any-angle instance depicted in Figure~\ref{fig::ex1}. 
The start point is $n_1 = (2, 0)$ and the target
point is $n_2 = (3, 4)$.  The best online algorithm for solving such
problems, to date, is Theta*~\cite{nash07}; a method which computes
an any-angle path by only considering the set of discrete points from 
the grid. 
Each time such a point is reached Theta* ``pulls the string''.  
Thus when node $n_2$ is generated its $g$-value is not
the length of the grid-constrained path from $n_1$ to $n_2$
but rather the length of the direct path $\langle n_1, n_2 \rangle$.

%\begin{figure}[tb]
% \begin{minipage}[b]{\0.45\linewidth}
%  \begin{center}
%    \begin{tikzpicture}

\creategrid{6}{6}

\drawobstacle{3}{3}

\coordinate (n1) at (2,0);
\coordinate (n2) at (3,4);

\path[name path=direct] (n1) -- (n2);
\path[name path=row 1] (0,1) -- (6,1);
\path[name path=row 2] (0,2) -- (6,2);
\path[name path=row 3] (0,3) -- (6,3);
\draw[name intersections={of=direct and row 1,by=y1}];
\draw[name intersections={of=direct and row 2,by=y2}];
\draw[name intersections={of=direct and row 3,by=y3}];

\draw[red] (n1) -- (n2);

\draw (n1) ++(-0.4,0.4) node {$n_1$} + (0.1,-0.1) -- (n1);
\draw (n2) ++(-0.4,0.4) node {$n_2$} + (0.1,-0.1) -- (n2);
\draw (y1) ++(-0.4,0.4) node {$y_1$} + (0.1,-0.1) -- (y1);
\draw (y2) ++(-0.4,0.4) node {$y_2$} + (0.1,-0.1) -- (y2);
\draw (y3) ++(-0.4,0.4) node {$y_3$} + (0.1,-0.1) -- (y3);

\end{tikzpicture}

%  \end{center}
%  \caption{Example of an any-angle path}
%  \label{fig::ex1}
%  \end{minipage}
%\end{figure}

The problem with this approach is that the solution-cost estimate
(or $f$-value), from a parent node to each of its successors, may 
not be monotonically increasing.  The monotone condition is
necessary to guarantee that an optimal solution, if one exists, is always found.
For instance: Theta* can generate $n_2$ from the intermediate point $p = (3,3)$.
When $p$ is expanded we have $f(p) = d(n_1, p) + h(p, n_2) = 4.16$. 
To satisfy the monotone condition we require that $f(n_2) \geq 4.16$. However 
Theta* computes $f(n_2) = d(n_1, n_2) + h(n_2, n_2) = 4.12$.
Clearly $p$ should be expanded after $n_2$ but in this case the opposite occurs.  
In order to avoid this mistake we would need to consider, in addition to the
set of discrete points from the grid, all the points $y_i$ shown in Figure~\ref{fig::ex1}.
The problem is that the number of such points can be very large:
each edge of the grid, together with its discrete endpoints, 
forms a $[0, 1]$ interval that can be intersected by the optimal
path at any point $0 \leq \frac{w}{h} \leq 1$; here $w$ (resp. $h$) is an integer in
$\{0,\dots,W\}$ (resp.  $\{0,\dots,H\}$).
This is a set whose members are reducible to a Farey Sequence.
For any given $n$ (in our case $n = \max(W, H)$) the cardinality of the corresponding 
set of elements is known to be quadratic in $n$~\cite{concrete89}(Ch. 9).
We are therefore motivated to consider an alternative approach: instead
of evaluating each $y_i$ node individually we will evaluate together
all the nodes from the corresponding interval in which each $y_i$ appears.


