Consider the example presented on Figure~\ref{fig::ex1} 
where the path requires to travel from C7 to D3.

\begin{figure}[ht]
  \begin{center}
    \includegraphics[scale=0.3]{images/ex1}
  \end{center}
  \caption{Example of an any-angle path}
  \label{fig::ex1}
\end{figure}

To find a path between node C7 and node D3, 
the pathfinding algorithm proposed in [CITE] 
successively explores nodes C6--C3 and D7--D4 
(thereafter called the intermediate nodes) until reaching D3.  
Notice that the actual shortest path between C7 and D3 
does not go through these intermediate nodes, 
and Theta* accounts to this fact by ``pulling'' the string 
when a new node is explored.  
Presented differently, when D3 is generated, 
the $g$ value is computed for the direct path C7--D3 
and not based on a path that would go through C6 or D7.  

The problem with this algorithm 
is however that the $f$ value of the intermediate nodes 
is based on the assumption that the path would go through these nodes.  
Consequently, the $A^*$ $f$ value computed 
for the intermediate nodes may be higher 
than the actual optimal path length, 
although the intermediate nodes must be expanded 
before reaching the goal.  
In order to avoid this mistake, 
we would need to explore not only the intermediate nodes, 
but also the Y$i$ nodes presented on the figure.  
However, the number of such Y$i$ nodes is a priori very large, 
since the search engine does not know a priori 
where the next interesting node (here C3) will be.  
Assuming a map of size $W\times H$, 
the relevant nodes could have any position 
that is a fraction $\frac{w}{h}$ 
where $w$ (resp. $h$) is an integer 
in $\{0,\dots,W\}$ (resp. $\{0,\dots,H\}$).  

We therefore propose to consider all the potential Y6 nodes together.  
From C7, all the nodes that can be reached with a one step North 
(and potentially any fraction of East steps) will be considered.  
We call ``interval'' this set of nodes that are considered together.  

How should we then compute the successors of an interval?  
Look at Figure~\ref{fig::succ1} (left) 
where a North step is taken from the interval [B4--E4] 
reached from node B6.  
The (single) successor of the interval 
is simply computed by projecting the visibility cone 
(represented with dashed lines)
from B6 to the next row.  
This leads to the interval in row 3.  
(As a consequence, the parent B6 of the nodes in the interval 
must be kept together with the interval.)  

\begin{figure}[ht]
  \begin{minipage}{0.5\linewidth}
  \begin{center}
    \includegraphics[scale=0.3]{images/ex2}
  \end{center}
  \end{minipage}
  \begin{minipage}{0.5\linewidth}
  \begin{center}
    \includegraphics[scale=0.3]{images/ex3}
  \end{center}
  \end{minipage}
  \caption{Computing the successors of an interval}
  \label{fig::succ1}
\end{figure}

There are more complicated situations, 
as presented in Figure~\ref{fig::succ1} (right).  
Here, we can see that an obstacle splits the interval 
in two separate intervals.  
We could define disjunct intervals, 
but it seems more natural to consider them independently.  
Furthermore, the path might turn on mode D4 
and go, e.g., to node D3 which is not directly accessible from B6.  
At this stage, we just assume that two intervals 
are built from D4; 
both intervals stretch as far as possible left and right, 
and the first one moves northward 
while the second is headed south.  

Let us present more precisely how the successors are splitted.  
From the interval I = [B4,E6], 
we assume that an epsilon step 
(i.e., an infinitely small step North) is taken.  
This step reveals how the interval will be splitted.  
Therefore, we generate interval [B(4-e),(E+$\frac{3}{2}$e)(4-e)]; 
we then remove the parts that belong to the obstacle, 
which produces the intervals [B(4-e),C(4-e)] 
and [D(4-e),(E+$\frac{3}{2}$e)(4-e)]; 
finally, we split the part that is not visible from B6, 
which leads to three intervals: [B(4-e),C(4-e)], 
[D(4-e),(D+e)(4-e)],](D+e)(4-e),(E+$\frac{3}{2}$e)(4-e)].  
If we project back on row 4, 
we obtain three intervals: 
[B4,C4], [D4,D4], and ]D4,E4].  
We say that these intervals are ``consistent''%
\footnote{We need a better word here.} 
with the obstacles.  
The first and last intervals can be delt with similarly 
to the example in Figure~\ref{fig::succ1} (left).  
The second interval, being a single point, 
requires to generate new intervals.  

The remaining question is what $f$ value 
should be associated with these intervals.  
For $A^*$ to return the optimal solution, 
we need $f$ to be an underestimate of the actual value 
(except for the final node).  
The interval has the following semantics: 
the path that is currently being constructed 
should start from $s$, cross the parent P, 
then cross the interval, and finally reached the goal.  
The shortest path that belongs to this set 
has length $g^*(P) + min_{x \in I}(d(P,x) + h^*(x))$.  
The value $g^*(P)$ should be known at this stage 
(this is one advantage of $A^*$).  
However, the minimum factor is unknown, 
but it can be lower-bounded by $min_{x \in I}(d(P,x) + h(x))$.  
This formula is actually pretty simple to solve, as we now demonstrate.  

Let $z_1$ and $z_2$ be the two extreme points of the interval.  
Draw a direct line between the goal and the parent $P$ of the interval, 
as shown in Figure~\ref{fig::fvalue}.  
If the line crosses the interval in \textit{\u z}, then the value is h(P).  If the line passes on the left, 
then the value is $d(P,z_1)+h(z_1)$; 
otherwise, it is $d(P,z_2)+h(z_2)$.  

\begin{figure}[ht]
  \begin{center}
    \includegraphics[scale=0.3]{images/fvalue}
  \end{center}
  \caption{Computing the minimum distance 
    to go from $P$ through the interval to the goal; 
    if the goal is G1, the direct distance between $P$ and $G1$; 
    if the goal is G2 or G3, 
    the shortest distance is the purple distance.}
  \label{fig::fvalue}
\end{figure}
