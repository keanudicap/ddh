\section{Conclusion}
We study the problem of symmetric path elimination in 4-connected grid maps.
Though less popular than the 8-connected variant, 4-connected grid maps appear 
regularly in modern video games and academic literature.
\par
We presented a novel offline method for breaking path symmetries which is simple
to understand and requires no significant extra memory. 
Our method involves decomposing a map into empty rectangular rooms, pruning all nodes
appearing in the interior and replacing them with a set of \emph{macro edges}
that facilitate optimal traversal from the perimeter of any room to the perimeter
of any other.
We also give an online node insertion technique that extends these guarantees
 to arbitrary pairs of locations appearing in the original unmodified map.
\par
We evaluate the performance of our algorithm by running A* on a wide
range of realistic game maps including one well known set from the game
\emph{Baldur's Gate II}. 
In many cases we are able to prune over 50\% of all nodes on a given map
and improve the average search time performance of A* by a factor of up to 3.5.
\par
The performance of our method depends on the topography of individual maps: 
in the presence of large rooms or wide open areas (both commonly seen in video games\footnote{For 
example, Blizzard's popular multi-player game \emph{World of Warcraft}})
we can often compute optimal paths much faster than searching on the original map. 
On less favourable map topographies we achieve more modest improvements.
However, since our method is orthogonal to existing search techniques, it could be integrated
as part of a larger framework involving specialised heuristics or other speedup techniques; 
for example as described in \cite{botea04,bjornsson05,bjornsson06}. 
\par
One direction for further work is to study breaking path symmetries
in 8-connected grid maps. 
This domain also exhibits a high degree of path symmetry but the problem is more
challenging because each tile has a higher branching factor. 
%%One approach involving macro edges has already been tried 
%%\cite{bolanca09} but we believe a different method, that does not involve
%%exploring the interior of empty rooms, could be promising.
Another direction for future work is to investigate alternative decomposition
algorithms which produce bigger rooms and improve the performance of the current
method.
