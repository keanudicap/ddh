\section{Experimental Setup}
We evaluate the performance of jump point pruning on four benchmarks taken from the freely
available pathfinding library Hierarchical Open Graph
(HOG, \url{http://www.googlecode.com/p/hog2}): 
\begin{itemize}

\item{\textbf{Adaptive Depth}
is a set of 12 maps of size 100$\times$100 in which approximately $\frac{1}{3}$
of each map is divided into rectangular rooms of varying size and a large
open area interspersed with large randomly placed obstacles.
For this benchmark we randomly generated 100 valid problems per map for a 
total of 1200 instances.
} 

\item{\textbf{Baldur's Gate} is a set of 120 maps taken from BioWare's popular
roleplaying game \emph{Baldur's Gate II: Shadows of Amn}; it 
appears regularly as a standard benchmark in the literature
\cite{bjornsson06,harabor10,pochter10}.
We use the variaton due to Nathan Sturtevant where all maps have been scaled to size
$512\times512$ to more accurately represent modern pathfinding environments.
Maps and all instances are available from \url{http://movingai.com}
}

\item{\textbf{Dragon Age} is another realistic benchmark; this time taken from
BioWare's recent roleplaying game \emph{Dragon Age: Origins}.
It consists of 156 maps ranging in size 
from $30\times21$ to $1104\times1260$.
For this benchmark we used a large set of randomly generated instances,
again due to Nathan Sturtevant and available from \url{http://movingai.com}.
}

\item{\textbf{Rooms} is a set of 300 maps of size
256$\times$256 which are divided into symmetric rows of small rectangular areas
($7\times7$), connected by randomly placed entrances. This benchmark has
previously appeared in \cite{pochter10}.
For this benchmark we randomly generated 100 valid problems per map for a 
total of 30000 instances.
}
\end{itemize}

Our test machine is a 2.93GHz Intel Core 2 Duo processor with 4GB RAM running OSX 
10.6.4.  
