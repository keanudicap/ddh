\section{Experimental Setup}
We evaluate the performance of RSR on three benchmarks taken from the freely
available pathfinding library Hierarchical Open Graph
(HOG)\footnote{\url{http://www.googlecode.com/p/hog2}}: {\textbf{Adaptive Depth}
is a set of 12 maps of size 100$\times$100 in which are composed of rectangular
rooms and open areas interspersed with obstacles.} 
{\textbf{Baldur's
Gate} is a set of 120 maps from BioWare's popular roleplaying game
\emph{Baldur's Gate II: Shadows of Amn}.  They range in size from $50\times50$
to $320\times320$ and often appear as a standard
benchmark in the literature \cite{bjornsson06,harabor10,pochter10}.}
{\textbf{Rooms} is a set of 300 maps of size
256$\times$256, each divided into small rectangular areas
($7\times7$), connected by randomly placed entrances. Rooms has
previously appeared in \cite{sturtevant09,pochter10,goldenberg10}.}
\par
We used two copies each map: one in which diagonal transitions are allowed and 
another in which they are not.  
For each map we generated 100 valid problem instances, checking that every
instance could be solved both with and without the use of diagonal transitions.
Our test machine had a 2.93GHz Intel Core 2 Duo processor, 4GB RAM and ran OSX
10.6.2.  Our implementation of A* is based on one provided in HOG, which we
adapted to facilitate our online node pruning enhancement.
